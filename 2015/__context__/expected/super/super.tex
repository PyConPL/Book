\usemodule[pycon-2015]
\starttext

\Title{Sztuczki z metodą super()}
\Author{Dominik \quotation{Socek} Długajczyk}
\MakeTitlePage

Jest jedna funkcjonalność Pythona, której wszyscy używają bardzo często,
acz szczątkowo, gdyż panuje opinia iż nadużywanie jej może doprowadzić
do nieokreślonego działania aplikacji. Tą funkcjonalnością jest
wielodziedziczenie. Postaram się w kilku słowach pokazać jak działa
\quotation{Method Resolution Order} w Pythonie (wybaczcie, nie znalazłem
dobrego tłumaczenia tej nazwy). Postaram się wam pokazać jak można użyć
\quotation{super()}, aby zamockować i przetestować przykładową klasę.

Zacznijmy jednak od teorii. W programowaniu panuje przekonanie, że
wielodziedziczenie jest złe, gdyż może występować \quotation{the diamond
problem}, czyli sytuacja, w której klasa dziedziczy po dwóch klasach, a
te dwie klasy dziedziczą po jednej i tej samej. Gdybyśmy w takiej
sytuacji (powiedzmy w C++) we wszystkich klasach zaimplementowali taką
samą metodę, to język nie wiedziałby w jakiej kolejności wykonać dane
metody. W Pythonie ten problem nie występuje dzięki algorytmowi
\quotation{C3 linearization}, który został wykorzystany do implementacji
MRO. Wprowadzono go w wersji 2.3 (new style classes) oraz w językach
Perl 5 i Parrot. Niestety nie są znane mi inne języki, które by
implementowały ten algorytm.

Dodatkowym atutem Pythona jest fakt, iż wywołanie metody rodzica, jest
jawne w metodzie potomka. Dzięki temu, można kod rodzica uruchomić nie
tylko przed wywołaniem własnej metody, ale na przykład w środku. Daje
nam to większą władzę oraz naprawdę duże pole do manewru.
{\externalfigure[graph.new.png]}

Weźmy pod uwagę tą specyficzną sytuację (wyciągniętą z jednego z moich
projektów). Jest to hierarchia dziedziczenia jednej klasy i jej
rodziców. Liczby w nawiasach określają numerek w liście wywołania, który
dostajemy dzięki MRO.

OrdersListController dziedziczy po Controller. Tutaj jest jeszcze
wszystko zrozumiałe. Jednak Controller dziedziczy już po 4 klasach (w
kolejności): * Requestable * FanstaticController * FormskitController *
FlashMessageController

Tutaj wprawdzie jest już zastosowane dziedziczenie po wielu klasach
bazowych, jednak MRO w takiej sytuacji jeszcze nie pokazuje swoich
możliwości. To co dzieje się głębiej jest już sednem użycia
\quotation{C3 linearization}. FlashMessageController jest 5. w
kolejności, co kończy listę bezpośrednich rodziców klasy Controller. To,
że BaseController ma numerek 6 wymaga już niestety trochę większego
zastanowienia. Nie będę się nawet starał tłumaczyć czemu akurat tak to
zostało obliczone (gdyż ten temat bardzo dobrze opisał na swojej
prelekcji Raymond Hettinger, twórca implementacji MRO. Prelekcja nazywa
się \quotation{Super considered super!}*).
{\externalfigure[graph.mocked.png]}

Tym diagramem chciałem natomiast pokazać, iż w Pythonie nie występuje
\quotation{the diamond problem}. Jest jeszcze jedna bardzo ważna cecha
tego algorytmu: dziecko nigdy nie wie która metoda zostanie wykonana,
gdy użyjemy super(). Zwróćcie proszę uwagę na klasę FanstaticController
(z numerkiem 3). Dziedziczy ona jedynie po klasie BaseController, a nic
nie wie o klasie FormskitController (z numerkiem 4), która będzie
wykonana zaraz po niej. Jest tak tylko dlatego, że klasa Controller
dziedziczy po tych dwóch klasach.

Ta konkretna funkcjonalność daje nam bardzo ciekawą możliwość. Mieliście
kiedyś problem o nazwie \quotation{jak zmockować funkcję super()?}.
Czyli w testach chcecie wykonać kod z jednej klasy, ale nie chcecie
wykonywać kodu rodziców. Można to zrobić bardzo prosto: wybierzmy klasę
do testów, w tym przypadku \quotation{OrdersListController}. Następnie
tworzymy klasę MockedOrderListController, która będzie dziedziczyłą po
tych samych klasach co OrderListController. A następnie robimy pustą
klasę OrdersListControllerEx, która dziedziczy po OrdersListController
oraz MockedOrderListController (w tej kolejności). Dzięki czemu najpierw
zostanie wykonana metoda z klasy OrdersListController, a potem
MockedOrderListController. W samym MockedOrderListController po prostu
nie wywołujemy metody super(), dzięki czemu nie zostanie wykonany żaden
kod rodziców.

Niestety taka sztuczka ma sens tylko w testach, gdyż wtedy wiemy
dokładnie jaką klasę blokujemy i jak MRO się zachowa (oraz jak struktura
się zmieni). Jeśli chcielibyśmy jeszcze raz podziedziczyć po wielu
klasach, to hierarchia MRO może ulec zmianie i nasz kod przestanie
działać. Tak samo nie możemy zablokować kodu konkretnej klasy w
hierarchii, już nie mówiąc o tym, że trzeba by przepisać wszystkie te
klasy, po których nasza klasa będzie dziedziczona, aby zmienić
strukturę~dziedziczenia (co jest niewykonalne, gdyż nigdy nie wiemy co
będzie dziedziczyło po naszej klasie).

Wiedząc to wszystko bardzo łatwo będzie nam teraz zrozumieć po co w
argumentach metody super trzeba podać~aktualną~klasę~oraz aktualny
obiekt. Czyli:

\starttyping
super(OrdersListController, self)
\stoptyping

Pierwszym argument jest miejsce w liście MRO, drugim - obiekt, od której
lista się~zaczyna. Na przykład: \quotation{self} może być typu
OrdersListController, jeśli stworzyliśmy obiekt tej klasy, natomiast
jeśli zrobiliśmy klasę~która tylko dziedziczy po OrdersListController,
to self będzie typu tej nowej klasy. Dlatego użycie super w tej formie
nie ma sensu:

\starttyping
super(self.__class__, OrderListController)
\stoptyping

Zauważyłem jakiś czas temu, że bardzo często używam super() na początku
każdej metody. Chciałem zatem zrobić dekorator @superme, który by
wykonywał super za mnie. Nie udało mi się to, gdyż dekoratory działają w
dość prosty sposób. Dekorator to funkcja, która bierze funkcję i zwraca
inną funkcję. Jeśli zrobimy dekorator na metodzie, to w klasie
wyciągniemy tę metodę i wystawimy nową. Jeśli spróbujemy w tej nowej
funkcji użyć~super, to wszystko powinno grać. Jeśli natomiast spróbujemy
użyć~super na metodzie, której rodzic ma metodę pod dekoratorem, to
używając super nie dostaniemy naszej upragnionej metody, lecz funkcję
zwróconą przez dekorator. Czyli mając coś takiego:

\starttyping
def superme(fun):
    def wrapper(*args, **kwargs):
        return fun(*args, **kwargs)
    return wrapper


class One(object):

    @superme
    def method(self):
        pass


class Two(One):

    def method(self):
        print(super(Two, self).method)

Two().method()
\stoptyping

W klasie Two.method, pod super().method nie dostaniemy One.method, tylko
coś w rodzaju: \quotation{} Mając Two.method albo One.method, możemy
dostać~się~do klasy, w której ta metoda została zdefiniowana. Mając już
klasę, możemy użyć metody super(), gdyż mamy pełną listę MRO (dzięki
self), oraz miejsce w hierarchii (dzięki klasie). Mając funkcję wrapper
nie jesteśmy w stanie (a przynajmniej ja nie znam sposobu) dostać~się do
klasy, na której ten wrapper został założony.

Po przejrzeniu tego artykułu przez kolegę~Nihilifer'a okazało się, że
jest możliwość zrobienia takiego dekoratora. Wystarczy użyć
functools.wraps jako dekorator funkcji wrapper. Dzięki temu dekorator
zwróci nam metodę, którą udekorowaliśmy. Chwila przeszukiwania
stackoverflow da nam funkcję, która zwróci nam klasę, w której dana
metoda była zdefiniowana.

\starttyping
import inspect
from functools import wraps


def get_class_that_defined_method(meth):
    if inspect.ismethod(meth):
        for cls in inspect.getmro(meth.__self__.__class__):
            if cls.__dict__.get(meth.__name__) is meth:
                return cls
        meth = meth.__func__  # fallback to __qualname__ parsing
    if inspect.isfunction(meth):
        module = inspect.getmodule(meth)
        data = meth.__qualname__.split('.<locals>', 1)[0].rsplit('.', 1)[0]
        cls = getattr(module, data)
        if isinstance(cls, type):
            return cls
    return None


def superme(fun):
    @wraps(fun)
    def wrapper(*args, **kwargs):
        return fun(*args, **kwargs)
    return wrapper


class One(object):

    @superme
    def method(self):
        pass


class Two(One):

    def method(self):
        method = super(Two, self).method
        cls = get_class_that_defined_method(method)
        print(method, cls)

Two().method()
\stoptyping

I to nam da taki wynik:

\starttyping
<bound method Two.method of <__main__.Two object at 0x7fc8f2c2e160>> <class '__main__.One'>
\stoptyping

Jak widzicie tutaj, funkcjonalność metody super() nie jest taka prosta
jak może się wydawać, jednak trzeba ją poznać i zrozumieć, aby nie
strzelić sobie w stopę przy jej użyciu. Mam nadzieję, że tym artykułem
zachęciłem was do głębszego poznania wielokrotnego dziedziczenia i
eksperymentowania z tą częścią języka Python.

\subsection[źródła]{Źródła}

\startitemize
\item
  https://en.wikipedia.org/wiki/C3\letterunderscore{}linearization
\item
  https://www.youtube.com/watch?v=EiOglTERPEo
\item
  https://github.com/nihilifer
\item
  https://docs.python.org/2/library/functools.html\#functools.wraps
\item
  http://goo.gl/4x6Ap1
\stopitemize


\stoptext
