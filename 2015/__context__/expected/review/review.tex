\usemodule[pycon-2015]
\starttext

\Title{Why Do You Need Code Review Tools?}
\Author{Radomir Dopieralski}
\MakeTitlePage

There is a simple and straightforward way to improve the quality of your
code, pay of some of your technical debt, teach the new members of your
team and keep everyone up to date about what is happening in the
project. As you might have guessed, you just have to start doing regular
code reviews. It is that easy.

As outlined in multiple publications, of which I will just mention here
the \quotation{Best Kept Secrets of Peer Code Review} by J. Cohen, S.
Teleki and E. Brown, code review can greatly lower the overall cost of
development, mostly by decreasing the number of defects that make their
way into later stages of development, by catching them early. Of course
that is just a single benefit, one that is relatively easy to measure,
and one that speaks to the management best. But there are multiple other
benefits for the team and the clients. In short, you really want to have
code reviews.

However, those benefits do come at a cost. Properly performing a formal
code review (or code inspection, how it is often called) requires
preparation, skill and training. The time spent on review has to come
from somewhere, and it usually comes from the time normally used to
write that code in the first place. And you can't just have a dedicated,
less skilled person hired specially to do it -- it has to be the same
programmers who are actually writing that code in the first place --
otherwise you do not get those benefits.

Another problem is that many programmers don't like doing it, and will
complain loudly when forced. After all, they are here to write code, not
to sit in endless meetings. And having your code dissected and publicly
analyzed is not necessarily very pleasant for people with large egos, of
whom there are plenty in this profession. So you will have some
resistance here too.

In open source projects that are being developed by volunteers it's even
worse than that. If you force the contributors to jump through too many
hoops, they will just go to some other project that makes better (in
their opinion) use of their time, or do something entirely different
than programming.

Fortunately, as many social problems do, this one also has a simple
technical solution. You can minimize the time spent on preparation to
code review, guide the programmers through the review process, so that
they don't need special training, and still keep the experience engaging
and pleasant, by using specialized software. In addition to that, in
open source projects, you may even attract users who don't feel
confident enough to write code, but are happy to review the submitted
patches and gradually learn more about the project and become regular
contributors.

A typical code review software lets you submit a patch or a series of
patches with a simple command, usually added as a plugin to your version
control system. Once submitted, the patch appears on a list, and
optionally the people responsible for the project receive notifications.
You can also explicitly add certain people to the review, if you believe
they should look at the code before it is accepted. Programmers can then
review the patch by looking at the changes in context of the whole
program, and can also check the modified version of the program to
analyze and test it at run time. They can then easily comment on the
whole patch, individual files or even individual lines of code. It is
common for those comments to grow into whole discussions, especially
when the change being discussed is not particularly important. New
versions of the change can be uploaded for review as the comments are
addressed -- either by the original author, or by anyone else. Finally,
after set criteria are met (enough of the right people voice their
approval), the change is accepted. The criteria may differ depending on
the process used in the given project, and the code may even be merged
into the project already or only after acceptance. Different projects
have different styles and the tools are usually flexible enough to
account for that.

The review process is also an excellent place for plugging all sorts of
code analysis and continuous integration tools. If you care about the
style of the code being written, add a linter in there. If you have
automated tests, have them run on each submitted change and attach the
result to the patch. Finally, if you care about certain metrics, this is
the place where it's the easiest to measure them. You can, for instance,
tell exactly how much time the programmers are spending on code review
and whether it's worth the increased code quality.

This place is also perfect for adding other sorts of automation.
Greeting new contributors and pointing them to guidelines. Informing
programmers about code freezes. Linking the patches to the information
in the bugtracker, and vice versa. Informing translators and
documentation writers about important changes. And so on -- the
possibilities are countless, and each of them saves someone on your team
some time and makes somebody's life easier -- perhaps even enough time
is saved to account for the extra time spent on reviewing the code.

Finally, reviewing the code becomes a much easier and smaller task. You
can review a simple patch in a matter of minutes or even seconds.
Certain classes of bugs are immediately visible even to the original
author -- it's common to re-submit a fixed patch even before anybody
else looked at it. Of course, large, complex, distributed changes are
still hard do review, in particular when they require complicated and
specialized environment to test. This didn't change, and that's why you
still need to have your QA team and tests.

For open source project there is an additional benefit. If you make the
code review tool public (and there is no reason why you shouldn't,
especially when you handle security-sensitive patches separately), you
suddenly have a great way to attract, guide and teach newcomers to your
project. The tool guides them through the process, the programmers guide
them through the good practices used in that particular language or
project, and they even learn by reading reviews of patches submitted by
other people.

It is possible to achieve some part of that using other tools, or just
by manually following an agreed upon process, but that has considerable
drawbacks. The main drawback is the greater effort required to get
through all the hoops. While experienced members of the team who are
used to the process might not even notice the wasted time, it does
become a big problem for beginners and sporadic contributors. For
instance, the Linux kernel community is known for using e-mail to
exchange the patches for code review. What they don't tell you is that
the reviewers themselves have their own tools and script that automate
large part of the process, but are not easily accessible for anyone
else.

The CPython project still uses a bug tracker to collect and review the
contributions. At least officially, because in reality a lot of the
review process actually happens out-of-band, through alternate means,
such as discussions on the IRC channels or face-to-face, e-mails,
pastebins, etc. In fact, it's almost impossible to have one's patch
merged by just following the official process. That is a huge barrier to
potential new contributors, especially if they represent minorities or
find it hard to socialize with the development team for other reasons.

On the other hand, the OpenStack project uses a transparent, public code
review system for its sub-projects. Anybody can jump in any time and
start reading the submitted patches, their comments, and add comments of
their own, potentially catching errors that other reviewers missed. They
also become more familiar with the project and more likely to start
contributing themselves. A huge win for the project.

And what about your project? Do you think it would use some improvements
in the quality of its code, reduction in the number of bugs that get to
production and overall improvement and growth of its community? If you
do, please start using a code review tool, it may be as simple as
creating a repository in one of the popular code hosting services!


\stoptext
