\usemodule[pycon-2015]
\starttext

\Title{Aplikacje desktopowe z PyQt}
\Author{Piotr Maliński}
\MakeTitlePage

\subsection[o-co-chodzi-z-qt-i-pyqt]{O co chodzi z Qt i PyQt?}

Qt to potężna platforma do tworzenia aplikacji z graficznym interfejsem
użytkownika. Możliwości wykorzystania Qt są bardzo duże. Możemy tworzyć
aplikacje desktopowe na komputery działające pod kontrolą MS Windows,
Linuksa, czy OSX. Możemy też tworzyć aplikacje na niektóre urządzenia
mobilne (czy bardziej - systemy wbudowane), automaty z ekranami
(dotykowymi), czy systemy digital signage i wiele więcej.

Programista Pythona ma do dyspozycji PyQt - interfejs Qt udostępniony w
Pythonie. Dzięki temu możemy tworzyć aplikacje desktopowe w Pythonie bez
konieczności użycia C++. Jako że pisanie prostych aplikacji w PyQt jest
bardzo szybkie i łatwe, biblioteka ta może bardzo się przydać w
codziennej pracy programisty jako narzędzie do tworzenia aplikacji
pomocniczych. W prezentacji zajmę się zaprezentowaniem możliwych
zastosowań PyQt - nie tylko dla twórców aplikacji desktopowych.

Qt i PyQt dostępne są na dwóch licencjach - GPL dla aplikacji na
licencji GPL oraz płatnej, komercyjnej licencji dla aplikacji
komercyjnych. Na licencji LGPL dostępna jest biblioteka PySide -
interfejs biblioteki Qt zapoczątkowany przez Nokię. Niestety rozwój
PySide pozostaje nieco w tyle za PyQt. W chwili pisania artykułu nie ma
nadal wersji dla Qt 5 (gdzie Qt 4 nie jest już za bardzo wspierana).

\subsection[skąd-wziąć-pyqt]{Skąd wziąć PyQt?}

Na Linuksie pakiety Qt i PyQt będą dostępne w repozytorium dystrybucji.
W zależności od dystrybucji pakiety mogą być mocno rozbite. W przypadku
MS Windows i OSX możemy zainstalować całe SDK Qt, a następnie pasującą
{[}paczkę PyQt{]}{[}1{]} (dostępne są dla różnych wersji Pythona) ze
strony projektu. PyQt działa zarówno z Pythonem 2 jak i 3.

Oprócz samej biblioteki mamy do dyspozycji szereg aplikacji, które mogą
być przydatne. Qt Creator to IDE dla programistów Qt. Nie wspiera ono
bezpośrednio Pythona i jeżeli interesuje Cię tylko Python, to Qt Creator
nie jest potrzebny. Qt Designer natomiast będzie potrzebny, ta aplikacja
służy do tworzenia interfejsu - rozmieszczania poszczególnych widżetów,
nazywania ich i konfigurowania wyglądu. Qt Designer będzie jako
oddzielny pakiet, lub razem z paczką SDK/Qt Creatorem.

Od strony Pythona potrzebne będą: \type{pyuic} i \type{pyrcc}. W
przypadku rodziny Ubuntu, czy Debiana aplikacje te dostępne są w
pakiecie - pyqt4-dev-tools, czy pyqt5-dev-tools. \type{pyuic} używamy do
kompilowania interfejsów z Qt Designera do klas Pythona, natomiast
\type{pyrcc} odpowiada za kompilowanie plików-zasobników plików
statycznych (np. własnych ikon używanych w interfejsie aplikacji). To,
plus sama biblioteka PyQt, wystarczy do tworzenia aplikacji.

\subsection[jak-zabrać-się-za-programowanie-w-pyqt]{Jak zabrać się za
programowanie w PyQt?}

Tworząc aplikację z PyQt zaczynamy zazwyczaj od narysowania interfejsu w
Qt Designerze. W bardzo prostych aplikacjach interfejs można stworzyć z
poziomu kodu aplikacji. Mając interfejs bierzemy się za łączenie
sygnałów wysyłanych przez widżety (np. kliknięto, strona została
załadowała) ze slotami - metodami w naszej klasie, które będą
implementować logikę aplikacji (jak kliknięto przycisk, to zrób coś).

Podstawą programowania z PyQt4 czy PyQt5 jest {[}Class
Reference{]}{[}1{]} - jest to lista wszystkich klas widżetów i innych
obiektów z biblioteki Qt. Pokazuje ona, w jakim module znajduje się dana
klasa i, co ważniejsze, opisuje wszystkie gettery, settery i sygnały
używane przez widżet. Cała logika widżetu jest ładnie widoczna, np.
zwykły przycisk QPushButton ma metodę text(), do pobrania tekstu
znajdującego się na przycisku. Do zmiany tego tekstu możemy użyć metody
setText(). Gdy przycisk zostanie kliknięty, wyemituje on sygnał
\quotation{clicked}. Pod taki sygnał podpinamy własną logikę i gotowe (w
pewnym uproszczeniu).

Na rynku znajdziemy też co najmniej kilka książek (w tym kilka
dostępnych już otwarcie w sieci {[}3{]}, {[}4{]}), szczególnie tych
anglojęzycznych. Większość z nich opisuje starszą wersję biblioteki -
PyQt4. W porównaniu do PyQt5, na chwilę obecną, dużych zmian nie ma.
Zmieniły się moduły, z których importujemy klasy, a także dodano np.
prostszą metodę łączenia sygnałów ze slotami. Tak więc mając książkę do
PyQt4 można sporo z niej wykorzystać programując z PyQt5. Do tego
znajdziemy w sieci sporo artykułów i poradników poświęconych PyQt. Ta
biblioteka jest dość popularna i nie ma problemu ze znalezieniem osoby,
która jej używa.

\subsection[pierwsza-aplikacja]{Pierwsza aplikacja}

Zacznijmy od aplikacji bez oddzielnie rysowanego interfejsu. Oto prosty
przykład z widżetem QWebView - widżetem okna przeglądarki. Widżet ten
posiada metodę \type{load()}, która przyjmuje adres URL i ładuje go tak,
jak w normalnej przeglądarce:

\starttyping
import sys

from PyQt5.QtCore import QUrl
from PyQt5.QtWidgets import QApplication
from PyQt5.QtWebKitWidgets import QWebView

app = QApplication(sys.argv)

web = QWebView()
web.load(QUrl("http://pl.pycon.org/2015/agenda.html"))
web.show()

sys.exit(app.exec_())
\stoptyping

Wszystko poza QWebView można uznać za boilerplate takiej prostej
aplikacji - importy i zainicjalizowanie QApplication. Odpalenie tego
kodu otworzy okno, w którym załaduje się agenda PyCon PL.

W bardziej praktycznych przypadkach będziemy mieć kilka widżetów,
będziemy chcieli je rozmieścić, ponazywać, czy upiększyć. Wtedy Qt
Designer będzie nieodzowny. Jako prosty przykład - otwórz Qt Designer,
wybierz Widget jako bazę twojego interfejsu, a następnie przeciągnij na
niego przycisk (Push Button). Gdy zaznaczysz przycisk po prawej stronie,
pojawią się jego ustawienia - nazwa (objectName) i ustawienia wyglądu
tekstu. Ja swój nazwałem \type{closeButton} i zapisałem całość jako
\type{close.ui}. Następnie skompilowałem interfejs do klasy Pythona:

\starttyping
pyuic5 close.ui > close.py
\stoptyping

Mając klasę interfejsu można ją wykorzystać. Oto boilerplate dla takiego
przypadku:

\starttyping
import sys
from PyQt5 import QtWidgets

from close import Ui_Form


class MyForm(QtWidgets.QWidget):
    def __init__(self, parent=None):
        QtWidgets.QWidget.__init__(self, parent)
        self.ui = Ui_Form()
        self.ui.setupUi(self)


if __name__ == "__main__":
    app = QtWidgets.QApplication(sys.argv)
    myapp = MyForm()
    myapp.show()
    sys.exit(app.exec_())
\stoptyping

Z pliku close.py importujemy nasz widget. Nie zmieniałem jego nazwy,
więc domyślna to Ui\letterunderscore{}Form. Ten kod odpali nasz
interfejs, ale nic się nie stanie, gdy klikniemy w przycisk. Musimy
połączyć sygnał (kliknięcia) ze slotem:

\starttyping
class MyForm(QtWidgets.QWidget):
    def __init__(self, parent=None):
        QtWidgets.QWidget.__init__(self, parent)
        self.ui = Ui_Form()
        self.ui.setupUi(self)

        self.ui.closeButton.clicked.connect(self._close)

    def _close(self):
        print('zamykam')
        self.close()
\stoptyping

Oto prosty przykład obsługi sygnałów i slotów. Po kliknięciu przycisku
\type{closeButton} odpali się nasz slot - \type{_close}, który wykona
swoją logikę.

\subsection[dystrybucja-aplikacji-pyqt]{Dystrybucja aplikacji PyQt}

Aplikacje napisane z pomocą PyQt mogą bez problemu działać na
popularnych desktopowych systemach operacyjnych, ale będą też wymagać
zainstalowania całego środowiska developerskiego. Na szczęście da się
\quotation{zamrozić} nasze aplikacje do wersji niezależnej - czy to za
pomocą py2exe, czy py2app. W przypadku systemu Windows nieduża aplikacja
wraz z dołączonymi bibliotekami Pythona i PyQt da co najmniej 10-15 MB
aplikację.

Od pewnego czasu istnieje także pyqtdeploy - aplikacja do dystrybucji
aplikacji PyQt na Windows, Linuksa, OSX, a także Androida i iOS. Jeżeli
interesuje was Tizen (Maemo), czy Sailfish OS, to PyQt jest też tam
obecne. Także Windows RT / Windows Phone 8, czy Blackberry 10 / QNX są
listowane jako wspierane. Niemniej w przypadku systemów mobilnych nie
wszystko musi być dostępne, czy działać tak samo jak w wersji
desktopowej (szczególnie wygląd interfejsu).

PyQt dostępna jest też na Raspberry Pi i praktycznie każdym innym mini
komputerze na płytce. Takie zestawienie może przydać się, gdy tworzymy
rozwiązania digital signage, czy automaty z graficznym interfejsem
wystawianym użytkownikowi.

\subsection[konkurencja]{Konkurencja}

PyQt, czy wspomniane PySide nie są jedynymi bibliotekami do tworzenia
aplikacji z graficznym interfejsem. W bibliotece standardowej Pythona
znajdziemy \type{tk} - znacznie prostszą bibliotekę, której interfejs
graficzny może wyglądać nie za ciekawie, niemniej jest dostępna i
działa. Z zewnętrznych bibliotek mamy też PyGTK, wxPython i Pythoncard,
czy biblioteki bardziej wyspecjalizowane jak PyGame, czy Kivy.

Jak dla mnie, PyQt jest bardzo dobrą biblioteką i dlatego jakiś czas
temu wybrałem ją zamiast innych. Przenośność między platformami,
czytelny sposób programowania, czy spora grupa programistów pracujących
nad rozwojem Qt to duże atuty. Niemniej Twój projekt może mieć
wymagania, które lepiej spełnić może inna biblioteka.

\subsection[co-dalej]{Co dalej?}

Qt i PyQt implementują znacznie więcej niż widżety interfejsu. Dostajemy
np. obsługę wątków, dostęp do baz danych, do usług systemowych jak
drukowanie i wiele więcej. Do tego kod powinien być przenośny pomiędzy
różnymi systemami operacyjnymi, więc możliwości są duże - czy to w
ramach tworzenia prostych aplikacji pomocniczych, czy po większe
projekty desktopowych aplikacji. Zobacz, jakie klasy są dostępne na
Class Reference, a przekonasz się, jaki zbiór funkcjonalności ukryty
jest w PyQt.

Jeżeli biblioteka ta zainteresowała Ciebie, to zacznij od przerobienia
kilku prostych aplikacji, zapoznaj się z podstawowymi widżetami i
sposobem ich działania. Zrób prosty edytor tekstowy, prostą przeglądarkę
i temu podobne. W sieci znajdziesz sporo pomocnych materiałów, a w
księgarniach kilka książek. Lista dyskusyjna PyQt też oferuje szybką i
skuteczną pomoc. Na {[}mojej stronie{]}{[}5{]} znajdziesz wiele
poradników do PyQt4 (w miarę możliwości będę aktualizował je do PyQt5).

\startitemize
\item
  {[}1{]} Strona PyQt
  https://www.riverbankcomputing.com/software/pyqt/intro
\item
  {[}2{]} Class Reference PyQt5
  http://pyqt.sourceforge.net/Docs/PyQt5/\crlf
class\_reference.html
\item
  {[}3{]} Rapid GUI Programming with Python and Qt
  http://www.qtrac.eu/\crlf
pyqtbook.html
\item
  {[}4{]} GUI Programming with Python: QT Edition\crlf
https://www.commandprompt.com/community/pyqt/
\item
  {[}5{]} Poradniki PyQt4 http://www.python.rk.edu.pl/w/p/pyqt/
\stopitemize


\stoptext
