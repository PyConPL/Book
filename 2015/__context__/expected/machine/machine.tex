\usemodule[pycon-2015]
\starttext

\Title{Building better machine learning models: Power of ensembles}
\Author{Bargava Subramanian}
\MakeTitlePage

For a data platform, the quality of the machine learning model's output
is one of the most critical pieces for its success.

Data, more often than not, is messy. But once one gets past the
preliminary data clean-up, building an initial machine learning model
isn't difficult. But how far is that from the optimal model? One of the
core tenets in machine learning is that there is no substitute for
domain expertise; that a simpler model with the right features will
invariably outperform a complex model without the right features.

When solving a typical machine learning problem, once most of the
necessary data are collated and preliminary data cleansing work is done,
figuring out the right features, feature transformations and models are
as much art as science. This is so because the solution space to search
and prune grows exponentially as you start considering all feature
transformations and combinations of features along with multitude of
model parameters.

But what if there's a clever algorithmic way to search the solution
space? This article covers one such important toolkit: ensemble
learning.

\section[success-stories]{Success Stories}

Before going into what ensemble models are, it has to be mentioned that
they are widely used in practice and are often instrumental in breaking
into new thresholds. A couple of their successful use cases are:

\startitemize
\item
  Ensemble models improved the baseline movie recommendation by 10\% and
  won the Netflix million-dollar prize
\item
  Almost all recent
  \useURL[url1][https://www.kaggle.com/][][Kaggle]\from[url1]
  competitions are won by employing ensemble models. For example, see
  \useURL[url2][https://github.com/ChenglongChen/Kaggle_CrowdFlower][][here]\from[url2]
\stopitemize

\section[what-are-ensemble-models]{What are Ensemble Models?}

\startblockquote
This is how you win ML competitions: you take other peoples' work and
ensemble them together.”
\useURL[url3][http://cims.nyu.edu/~vitaly/][][Vitaly
Kuznetsov]\from[url3] NIPS2014
\stopblockquote

The theoretical underpinning is that consensus information from diverse
model outputs are more reliable than a single model. At its core, an
ensemble model structure looks like figure 1.

\placefigure[here,nonumber]{Figure 1: Ensemble model
structure}{\externalfigure[ensemble_1.png]}

Multiple model outputs are combined to create the final prediction. The
individual(single) models are called base models. How the various base
models are combined(ensembled) to create the final prediction determines
the quality/accuracy of ensemble learning.

The key ingredient to ensemble is to select the base models with
significant diversity in their predictions. If the individual base
learners are similar, they add little information. Worse, they tend to
reinforce the same error present within them.

\section[motivation-example]{Motivation Example}

Before going into the various ways of ensembling, let's take a step back
and realize that ensembles do work with a simple example.

Let's consider an example of classifying a credit card transaction as
fraudulent or not. The prediction has to be made on 10 transactions.
Let's take the case where we already know those 10 cases to be
fraudulent. Encoding 1 as fraudulent and 0 as non-fraudulent, let there
be three base model outputs, one each from Logistic Regression, Random
Forest and Gradient Boosting, each having 70\% accuracy.

While individual models have 70\% accuracy, consider a simple way of
ensembling them: Majority voting. Using majority voting scheme as the
ensembling function, the prediction accuracy is boosted from 70\% to
90\%. Case 6 alone will remain mis-classified, as shown in Table 1.

{\em Table 1:Example output from three models}

\bTABLE
\bTR\bTD RandomForest \eTD\bTD xgboost \eTD\bTD LogisticRegression
\eTD\bTD EnsembledOutput\eTD\eTR
\bTR\bTD 0 \eTD\bTD 1 \eTD\bTD      1 \eTD\bTD 1 \eTD\eTR
\bTR\bTD 1 \eTD\bTD 1 \eTD\bTD      0 \eTD\bTD 1 \eTD\eTR
\bTR\bTD 1 \eTD\bTD 0 \eTD\bTD      1 \eTD\bTD 1 \eTD\eTR
\bTR\bTD 1 \eTD\bTD 1 \eTD\bTD      0 \eTD\bTD 1 \eTD\eTR
\bTR\bTD 1 \eTD\bTD 1 \eTD\bTD      1 \eTD\bTD 1 \eTD\eTR
\bTR\bTD 0 \eTD\bTD 0 \eTD\bTD      1 \eTD\bTD 0 \eTD\eTR
\bTR\bTD 1 \eTD\bTD 1 \eTD\bTD      0 \eTD\bTD 1 \eTD\eTR
\bTR\bTD 1 \eTD\bTD 1 \eTD\bTD      1 \eTD\bTD 1 \eTD\eTR
\bTR\bTD 1 \eTD\bTD 0 \eTD\bTD      1 \eTD\bTD 1 \eTD\eTR
\bTR\bTD 0 \eTD\bTD 1 \eTD\bTD      1 \eTD\bTD 1 \eTD\eTR
\eTABLE

A key point to be stressed again is that model diversity is important.
If the models were similar, the output would've been similar and there
would've been no improvement in the output.

\section[advantages]{Advantages}

Some of the advantages of using ensemble models are:

\startitemize
\item
  {\bf Improved accuracy}: Dozens of models, even of mediocre quality,
  may produce top-notch predictions as a team.
\item
  {\bf Robustness}: more robust than a single model
\item
  {\bf Efficiency}: A divide-and-conquer paradigm. Complex problem is
  decomposed into many sub-problems and then solved
\item
  {\bf Parallelization}: It naturally lends itself to parallelization.
\item
  {\bf Wider search of solution space}: When ensembling, different
  solution spaces searched by individual models are combined, leading to
  a wider search of solution space. Since the combination step is not
  computationally intensive, the search is more efficient than a single
  model searching the entire solution space.
\item
  {\bf Reduces overfitting}
\stopitemize

\section[different-ways-to-create-ensemble-models]{Different ways
to create ensemble models}

\subsection[base-models]{Base Models}

Let's now discuss some of the ways to create the base models, to ensure
that they have enough diversity

\startitemize[n][stopper=.]
\item
  {\bf Different training sets}: If there's enough data, the same model
  can be run on different training examples. In practice, transformation
  of original training set is carried out to create multiple training
  sets. ({\em Widely used})
\item
  {\bf Different algorithms/algorithm diversity}: Choose diverse
  algorithms. \type{LinearRegression}, \type{LogisticRegression} are
  diverse from \type{DecisionTree}, \type{RandomForest} etc (Regression
  Vs Tree-based). \type{SVM} is another class of algorithms. Also,
  \type{Regression} could be L1 or L2. The basis function for \type{SVM}
  could be radial or linear. ({\em Widely used})
\item
  {\bf Different parameter setups}: Called the hyperparameters, the
  parameters of the model can be varied to obtain different model
  outputs. An example would be to build \type{RandomForest} with shallow
  trees versus building it with denser trees (less trees vs more trees)
  ({\em Widely used})
\item
  {\bf Algorithm randomization}: A lot of the models have randomness
  playing a key part and the model can be run with different random
  initialization. ({\em Not so widely used})
\item
  {\bf Feature Sampling} Using subset of columns for different model
  runs, produce different model outputs and can help in estimating and
  containing variance of model predictions({\em Widely used})
\stopitemize

\subsection[model-aggregation]{Model Aggregation}

Some of the ways to aggregate the models are

\startitemize[n][stopper=.]
\item
  {\bf Voting/Averaging}: A simple way to aggregate output is to either
  take the candidate with maximum outputs or the average of outputs (if
  regression or probability is the desired output). For simple
  averaging, all models are assumed equal.
\item
  {\bf Weighted voting/averaging}: The model is given a weight and the
  output is weighted based on that. An example would be: A model with
  higher accuracy has a higher weight than a model with lower accuracy.
\item
  {\bf Using as attributes}: The model outputs are used as inputs for a
  second-stage model. See {\em stacking}
\item
  {\bf Bagging}: {\bf B}ootstrap {\bf Agg}regat{\bf ing} -Base model is
  created using bootstrap samples of training set and are combined by
  plain voting. \type{RandomForest} uses bagging.
\item
  {\bf Stacking}: Using different algorithms as base learners, the model
  outputs are used as meta attributes for another model. Also called
  {\em Stacked Generalization}. Currently, these are widely used.
  {\em An example of stacking}: Divide the dataset into two. Using the
  first half, build \type{LogisticRegression}, \type{SVM},
  \type{LinearRegression}, \type{xgboost}, \type{RandomForest} models.
  Predict the output for the second half of the dataset. Using these
  meta attributes as input features, build a \type{LogisticRegression}
  model on the second half of the dataset.
\stopitemize

\section[disadvantages-of-ensemble-models]{Disadvantages of
ensemble models}

\startitemize[n][stopper=.]
\item
  Model human readability is not great. While there are ways to show
  variable importance, they are not as interpretable as a simple
  \type{LinearRegression} model or a \type{DecisionTree} model.
\item
  For some business cases, the tradeoff in time/effort it takes to build
  complex ensemble models may not be offset by the improvements in
  accuracy and hence may not be justified.
\stopitemize

\section[some-python-libraries-to-aid-efficient-ensembling]{Some
Python libraries to aid efficient ensembling}

\startitemize[n][stopper=.]
\item
  \type{Pipeline}: This creates a pipeline of transforms with a final
  estimator. An example of pipeline could be: pre-processing(capping,
  standardizing, scaling) -\lettermore{} feature extraction(principal
  components, transformation) -\lettermore{} model creation. The process
  can be made elegant and efficient using this.
\item
  \type{hyperopt}: It uses a form of bayesian optimization called
  {\em tree of Parzen estimators}. For weighted averaging of the base
  models, this is an efficient package for optimally identifying the
  models and their weights. In many cases, it may not be possible to
  specify the required objective function for the base models, but
  \type{hyperopt} allows custom objective function to be specified and
  so, ensures that the final ensembled model is optimized on the
  required objective function.
\item
  \type{RandomizedSearchCV}: While we didn't talk about hyper-parameter
  optimization, this helps in efficient searching of hyper-parameters
  for the model. When running dozens of models as base models, it may
  not be possible to do an exhaustive grid search of hyper parameters
  (and in some cases, not tractable too). Randomized search of the
  hyper-parameter space helps in producing better model parameters, and
  hence better base model outputs.
\item
  \type{joblib}: For parallel computing. Different models can be run in
  parallel.
\stopitemize

\section[references]{References}

\startitemize[n][stopper=.]
\item
  http://ews.uiuc.edu/\lettertilde{}jinggao3/sdm10ensemble.htm
\item
  http://www.overkillanalytics.net/\crlf
  more-is-always-better-the-power-of-simple-ensembles/
\item
  http://www.rms.com/blog/2013/10/08/a-weight-on-your-mind/
\item
  http://mlwave.com/kaggle-ensembling-guide/
\item
  http://fastml.com/optimizing-hyperparams-with-hyperopt/
\stopitemize


\stoptext
