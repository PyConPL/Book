\usemodule[pycon-2015]
\starttext

\Title{How to re, czyli jak wyrażenia regularne stały się tym, czym są obecnie}
\Author{Dariusz Śmigiel}
\MakeTitlePage

\subsection[historia]{1. Historia}

Wyrażenia regularne (powszechnie znane jako regular expressions, bądź w
skrócie regex/regexp) są to wzorce, które opisują łańcuchy symboli. Jak
podaje wikipedia: \quotation{Wyrażenia regularne mogą określać zbiór
pasujących łańcuchów, mogą również wyszczególniać istotne części
łańcucha.} {[}1{]} Obecnie wyrażenia regularne znajdują zastosowanie w
programowaniu, jednakże ich początki datowane są~na lata 40. XX wieku, a
z informatyką miały wtedy niewiele wspólnego.

\section[a-biologia-na-ratunek-informatyce]{a) Biologia na ratunek
informatyce}

Za podstawę~współczesnej informatyki uznawane są automaty skończone,
które wywodzą się z prac Alana Turinga. {[}2{]} Maszyna Turinga to
abstrakcyjny model komputera służącego do wykonywania algorytmów,
składającego się z nieskończenie długiej taśmy podzielonej na pola, w
których zapisuje się dane {[}3{]}. Zainspirowani pracą Turinga, dwaj
neurolodzy Warren McCulloch i Walter Pitts zbudowali sztuczny neuron,
który jest uproszczonym matematycznym modelem biologicznego neuronu.
Służy on jako podstawowy blok budulcowy sztucznych sieci neuronowych.
{[}4{]} W 1943 wyniki swojej pracy opublikowali w artykule naukowym w
Bulletin of Mathematical Biophysics 2:115-133 zatytułowanym:
\quotation{A logical calculus of the ideas immanent in nervous
activity}. Okazało się, że ten artykuł stał się podwaliną pod rozwój
współczesnej informatyki, chociaż nie taki był jego cel. {[}5{]}

\section[b-co-tam-panie-w-matematyce-słychać]{b) Co tam panie, w
matematyce słychać}

Kilkanaście lat później, w 1956 roku, matematyk Stephen Kleene poszedł o
krok dalej i, bazując na neuronie McCullocha-Pittsa {[}4{]},
zaproponował prostą algebrę, swoje przemyślenia publikując w artykule
\quotation{Representation of events in nerve nets and finite automata}.
{[}6{]} Jego celem było opisanie mózgu jako rachunku logicznego.
Kleene'a nie interesowało to, czy model McCullocha-Pittsa był dokładny;
skupił się na tym, czym ten model jest. W przeciwieństwie do podejścia
McCullocha-Pittsa, Kleene udowodnił, że model neuronu nie ogranicza się
jedynie do ludzkiego mózgu, ale w rzeczywistości opisuje szerszy zakres,
jakim jest każdy automat skończony, bez ograniczeń na to, czy jest to
zwierzę, człowiek czy maszyna. Tym samym nawiązał do wykładu o
automatach, jaki von Neumann dał w trakcie Sympozjum Hixona. {[}9{]} To
właśnie John von Neumann był jednym z pierwszych, którzy próbowali
przedstawić ogólną teorię automatów wykorzystując narzędzia logiki.
Artykuł Kleene'a był pierwszym krokiem na drodze ku temu, by można było
wyrażenia regularne zaszczepić w programach komputerowych.

\section[c-jak-by-tu-tego-użyć-w-informatyce]{c) Jak by tu tego
użyć w informatyce}

Prawdopodobnie efekt pracy, którą dokonali wymienieni wcześniej
naukowcy, nigdy by nie trafił do komputerów, gdyby nie Janusz
Brzozowski. {[}10{]} Brzozowski, który uzyskał tytuł doktora nauk
elektrycznych na Princeton University, przez pewien czas wykładał w
Berkeley. Wtedy też zaprezentował sposób na wykorzystanie wyrażeń
regularnych, by stworzyć diagram stanów, które były szeroko
wykorzystywane w latach 50-tych do projektowania i implementowania
programów w ówczesnych komputerach. Właśnie w Berkeley z Brzozowskim
zetknął się Ken Thompson {[}12{]}. Bazując na tym, co usłyszał,
postanowił zaadaptować to podejście i zaimplemenować je w
oprogramowaniu. Pracując przy tworzeniu UNIX-a, napisał coś, co
ostatecznie miało stać się narzędziem \quotation{grep}, a
zaimplementowane było w edytorze \quotation{QED}. \quotation{grep} to w
rzeczywistości skrót od \quotation{Globally search for regular
expression re and Print it}, czyli w wolnym tłumaczeniu
\quotation{globalne wyszukiwanie wyrażeń regularnych oraz wyświetlanie
ich}. Swój algorytm opisał w artykule \quotation{Regular Expression
Search Algorithm} {[}14{]}, a całość zaimplementował w asemblerze dla
IBM 7090.

\section[d-gdzie-tego-szukać]{d) Gdzie tego szukać?}

Od momentu, gdy Thompson zaimplementował pierwszą wersję wyrażeń
regularnych w QED, minęło sporo czasu. Wraz z rozwojem oprogramowania,
regexpy można znaleźć w \quotation{vi}, \quotation{lex} \quotation{sed},
\quotation{AWK}, czy w \quotation{Emacsie}. Oprócz tego, mnóstwo IDE
korzysta z odkryć, których korzenie sięgają lat 40-tych. Jednym z
pierwszych zastosowań w językach programowania może się poszczycić Perl.
Bazując na tym, większość współczesnych języków posiada zbliżoną
składnię: do tego grona można zaliczyć Javę, JavaScript, Ruby oraz
oczywiście Pythona.

\section[e-podsumowanie]{e) Podsumowanie}

Zakres transformacji, które wyrażenia regularne przeszły, jest
zadziwiający. Zaczynając od pracy Rudolfa Carnapa {[}11{]} w zakresie
logicznej składni języka, poprzez wspomniane neurony McCullocha-Pittsa.
Następnie podchwycone przez von Neumannowy opis komputera EDVAC, czy
jego późniejszą teorię automatów. By ostatecznie otrzymać nazwę od
matematyka Stephena Kleene. Dzięki Januszowi Brzozowskiemu, który
wykorzystał je, by zaprojektować diagramy stanów oraz dzięki pierwszej
implementacji wykonanej przez Kena Thompsona, ostatecznie trafiły do
wszelkiej maści programów, by służyć ku uciesze programistów.

\subsection[bibliografia]{2. Bibliografia}

Powyższy opis jest mocno skrótowy i nie wyczerpuje tematu. Jest bardziej
zarysem tego, jak interesujące są początki informatyki, którą znamy w
obecnej formie. W celu zgłębienia tematu polecam pozycje {[}2{]} oraz
{[}14{]}. Jak widać na powyższym przykładzie, z wielu, wydawałoby się,
niezależnych przemyśleń powstało coś uniwersalnego. Dlatego warto znać
historię i dokonania poprzednich pokoleń, by bazując na nich tworzyć
lepszą jakość.

\startitemize[n][stopper=.,width=2.0em]
\item
  https://pl.wikipedia.org/wiki/Wyrażenie\_regularne
\item
  Speech and Language Processing: An introduction to natural language
  processing, computational linguistics, and speech recognition. Daniel
  Jurafsky \& James H. Martin
\item
  https://pl.wikipedia.org/wiki/Maszyna\_Turinga
\item
  https://pl.wikipedia.org/wiki/Neuron\_McCullocha-Pittsa
\item
  http://blog.staffannoteberg.com/2013/01/30/\crlf
  regular-expressions-a-brief-history/
\item
  http://books.google.com/books?id=oL57iECEeEwC
\item
  https://pl.wikipedia.org/wiki/Język\_regularny
\item
  http://kelty.org/or/papers/Kelty\_Franchi\_LogicalInstruments\_2009.pdf
\item
  https://www.cs.ucf.edu/\lettertilde{}dcm/Teaching/COP5611Spring2010/\crlf
  vonNeumannSelfReproducingAutomata.pdf
\item
  https://en.wikipedia.org/wiki/Janusz\_Brzozowski\_(computer\_scientist)
\item
  https://en.wikipedia.org/wiki/Rudolf\_Carnap\#Logical\_syntax
\item
  https://pl.wikipedia.org/wiki/Ken\_Thompson
\item
  https://pl.wikipedia.org/wiki/Unix
\item
  http://www.fing.edu.uy/inco/cursos/intropln/material/p419-thompson.pdf
\stopitemize


\stoptext
