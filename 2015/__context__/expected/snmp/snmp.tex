\usemodule[pycon-2015]
\starttext

\Title{Writing SNMP Apps in Python}
\Author{Ilya Etingof}
\MakeTitlePage

\subsection[introduction]{Introduction}

Network management is important for keeping your ever growing network
infrastructure healthy and secure. SNMP is a well established technology
designed to identify and access key operational metrics of networked
hosts, routers and all sorts of industrial equipment.

As we all know, many organizations around the world use Python for their
IT automation needs, including network management. Over the course of
last decade, many SNMP implementations appeared. Some are Python
bindings to C-based {[}Net-SNMP library{]}, which is considered by many
a reference implementation for SNMP technology. Others are pure-Python
modules addressing specific SNMP features.

Among many SNMP libraries existing in the Python landscape, right from
the start, PySNMP project aims at complete and universal SNMP
implementation, offering its users full power of SNMP technology across
all computing platforms. Having taken this project seriously,
{[}PySNMP{]} developers also designed a couple of foundation libraries:
{[}PyASN1{]} and {[}PySMI{]} as a byproduct of their PySNMP work.

\subsection[hello-snmp-world]{Hello, SNMP world!}

Most frequent and well understood SNMP operation is about fetching a
value for a SNMP variable. In UNIX environment it is traditionally done
with snmpget tool:

\starttyping
$ snmpget -v2c -c public demo.snmplabs.com sysDescr.0
SNMPv2-MIB::sysDescr.0 = STRING: SunOS zeus.snmplabs.com 4.1.3_U1 1 sun4m
\stoptyping

Here we queried publicly available SNMP Manager at
{\em demo.snmplabs.com} for a value of MIB variable named
{\em sysDescr.0}. The same operation can be performed right from your
Python prompt:

\starttyping
from pysnmp.hlapi import *

errorIndication, errorStatus, errorIndex, varBinds = next(
    getCmd(SnmpEngine(),
           CommunityData('public'),
           UdpTransportTarget(('demo.snmplabs.com', 161)),
           ContextData(),
           ObjectType(ObjectIdentity('SNMPv2-MIB', 'sysDescr', 0)))
)

if errorIndication:
    print(errorIndication)
elif errorStatus:
    print('%s at %s' % (errorStatus.prettyPrint(),
                        varBinds[int(errorIndex)-1][0]
                        if errorIndex else '?'))
else:
    for varBind in varBinds:
        print(' = '.join([x.prettyPrint() for x in varBind ]))
\stoptyping

That code is somewhat verbose for a reason: PySNMP API exposes many SNMP
details, giving programmer great power and flexibility (attentive
readers may have spotted a Python generator in the code). But before we
dive into the details let me remind our readers basic SNMP design and
how PySNMP architecture maps into it.

\subsection[a-bit-of-background]{A bit of background}

In the early days of computer networking, as local networks grew in size
and complexity, keeping an eye on expanding farm of computers,
applications and other network equipment became a hassle to system
administrators. Besides simple ping-like methods for testing hosts and
services availability, it became a necessity to gather more detailed
information on systems health. Manual configuration of increasing number
of networked boxes did not scale well.

At that time there was no single predominant packet switching technology
like we see today. Large organizations were busy developing their own
network stacks each equipped with some form of network management
facilities. Ultimately, in early 90's, Internet Protocol Suite wins this
competition practically eliminating all other rivals. For TCP/IP network
management they offered SNMP, which remains principal technology even
today.

As a technology, SNMP defines application layer protocol, data model and
data objects. The protocol, which evolved greatly since its initial
introduction in early 80's, serves as a more or less secure, lightweight
and fault-tolerant communication channel between parties. SNMP data
model maps all interesting nuances of host or application internals to
named variables organized into hierarchy. Concrete collection of
variables is domain-specific, it is formally defined in MIBs -- files
written in a special domain-specific language, subset of ASN.1.

Although SNMP designers were trying to kill two birds with one stone,
offering both information collection and versatile remote configuration
features, the latter never really enjoyed much popularity among
implementers. Thus most frequently, SNMP is used for gathering some or
all variables from hosts or applications being managed.

With SNMP architecture, managed host or application should have a
component called SNMP Agent. It acts as an intermediate having access to
host/application internals and being able to funnel that information
over SNMP to interested parties in form of SNMP variables.

The other part of the system is called SNMP Manager, this component is
always looking for SNMP variables either by querying SNMP Agents or
listening for notifications they may produce whenever something happens.

\subsection[library-orientation]{Library orientation}

The PySNMP library is internally structured along the lines of
{[}RFC3411{]}. Components that are not unique to SNMP are put into
stand-alone Python packages to promote reusability.

SNMP protocol is defined in terms of ASN.1 data structures, SNMP
messages travelling the wire are encoded in {[}BER{]}. For those
purposes PySNMP relies on generic implementation of ASN.1 types and
codecs distributed as a dedicated Python package under the name of
{[}PyASN1{]}.

SNMP-level data processing is performed by a collection of SNMP Message
Processing {[}RFC3412{]} and Security {[}RFC3414{]} modules living in
{\em pysnmp.proto} sub-package. All crypto operations are offloaded to
the third-party {[}PyCrypto{]} package.

Base classes representing SMI types {[}RFC2587{]} are defined in
{\em pysnmp.smi}. They administer both of MIB purposes: for SNMP Manager
apps, it's a hierarchical database of MIB variables served by remote
SNMP Agent. Agents can use PySNMP SMI objects for interfacing with
backend host or application being managed.

PySNMP is designed to run in asynchronous I/O environment. Its I/O
subsystem is built around a set of abstract classes
({\em pysnmp.carrier}) whose purpose is to facilitate basing SNMP engine
on top of a third-party I/O framework. The library is shipped with a
handful of ready-to-use bindings to popular asynchronous cores including
{\em asyncore}, {[}asyncio{]} and {[}Twisted{]}.

All SNMP services are delivered through and components are orchestrated
by the SNMP Engine entity ({\em pysnmp.entity.engine}).

So called Standard SNMP Applications {[}RFC3413{]} are shipped in\crlf
{\em pysnmp.entity.rfc3413}. They all employ SNMP Engine instance for
their operations.

A more concise and higher-level programming interface to most frequently
used Standard SNMP Applications is offered via {\em pysnmp.hlapi
sub-package}. This slighly revised and improved API first appeared in
PySNMP 4.3. We will use this API throughout the article.

At this point we end up with a fully-functional SNMP entity that can run
in standard Manager, Agent and Proxy roles. However one SNMP feature is
still missing and that is MIB support. More often than not, MIBs are
used in the context of SNMP operations. However, other uses are not
impossible. For example one may want to transform MIB structures into
XML/HTML form or generate code in some programming language implementing
MIB features. That was the rationale behind PySNMP developers' decision
to isolate MIB parsing from SNMP Engine implementation and put it into a
dedicated Python package called {[}PySMI{]}.

Being optional, PySMI will be discovered and automatically used by
Manager applications, running on top of high-level PySNMP API, for MIB
variable names and types resolution.

\subsection[common-operations]{Common operations}

Besides reading known scalar variables we mentioned earlier, SNMP is
able to fetch a range of variable including those not known in advance.
The following code fetches all variables related to host's interface
table:

\starttyping
from pysnmp.hlapi import *

for errorIndication, errorStatus, errorIndex, varBinds in \
        nextCmd(SnmpEngine(),
                        CommunityData('public'),
                        UdpTransportTarget(('demo.snmplabs.com', 161)),
                        ContextData(),
                        ObjectType(ObjectIdentity('IF-MIB', 'ifDescr')),
                        ObjectType(ObjectIdentity('IF-MIB', 'ifType'))):

    if errorIndication:
        print(errorIndication)
        break
    elif errorStatus:
        print('%s at %s' % (errorStatus.prettyPrint(),
                            varBinds[int(errorIndex)-1][0]
                            if errorIndex else '?'
            )
        )
        break
    else:
        for varBind in varBinds:
            print(' = '.join([ x.prettyPrint() for x in varBind ]))
\stoptyping

In this example we iterate remote SNMP Agent over two MIB variables
({\em IF-MIB::ifDescr} and {\em IF-MIB::ifType}) which are in fact two
columns of SNMP table.

Any operation carried out through high-level API involves Python
generator. Each invocation of such generator translates into SNMP
message being sent and response processed. Generator functions are
specific to SNMP message type and are uniformly initialized with:

\startitemize
\item
  SNMP Engine object: this is the umbrella object coordinating all SNMP
  operations. It's used by SNMP applications like Command Generator
  application featured in example.
\item
  SNMP authentication method: that can be SNMPv1/v2c Community Name or
  {[}RFC3414{]} {\em UsmUser} object conveying USM username, encryption
  and ciphering keys.
\item
  Type of I/O to use for communication, endpoints addresses and other
  transport-specific options
\item
  SNMP Context Engine ID and Context Name: these are only applicable to
  SNMPv3 operations and can be used to identify a non-default remote
  SNMP Engine instance or specific instance of MIB variables collection
  behind remote SNMP Engine.
\item
  Sequence of MIB variables to query. Sometimes MIB variable name should
  be accompanied with a value to transfer to remote SNMP entity. Such
  value could be passes through {\em ObjectType()} initializer.
\stopitemize

As for return values, on each iteration generator give us a tuple of
result items:

\startitemize
\item
  errorIndication: if evaluates to True, it indicates some fatal problem
  occurred to local or remote SNMP engine. Most likely a timeout or
  authentication problem.
\item
  errorStatus and errorIndex: if errorIndex is not zero, that indicates
  a problem with particular MIB variable put into request. The
  errorStatus object provides more information on the nature of the
  problem.
\item
  varBinds: is a list of two-element tuples, each correspond to MIB
  variable and its value.
\stopitemize

As we can send data back into running generator, our script could be
modified to cherry-pick smaller sequences of adjacent MIB variables or
even individual scalars:

\starttyping
from pysnmp.hlapi import *

queue = [ [ ObjectType(ObjectIdentity('IF-MIB', 'ifInOctets')) ],
          [ ObjectType(ObjectIdentity('IF-MIB', 'ifOutOctets')) ] ]

iter = nextCmd(SnmpEngine(),
               UsmUserData('usr-md5-none', 'authkey1'),
               UdpTransportTarget(('demo.snmplabs.com', 161)),
               ContextData())

next(iter)

while queue:
    errorIndication, errorStatus, errorIndex, varBinds = iter.send(
        queue.pop())
    if errorIndication:
        print(errorIndication)
    elif errorStatus:
        print('%s at %s' % (
                errorStatus.prettyPrint(),
                varBinds[int(errorIndex)-1][0] if errorIndex else '?'
            )
        )
    else:
        for varBind in varBinds:
            print(' = '.join([ x.prettyPrint() for x in varBind ]))
\stoptyping

At any moment SNMP Agent may consider reporting specific events to SNMP
Manager. Such SNMP Notification message might include relevant variables
that help both Manager software and human reader learning the details of
the event being reported.

\starttyping
from pysnmp.hlapi import *

errorIndication, errorStatus, errorIndex, varBinds = next(
    sendNotification(SnmpEngine(),
                     UsmUserData('usr-md5-des', 'authkey1', 'privkey1'),
                     UdpTransportTarget(('demo.snmplabs.com', 162)),
                     ContextData(),
                     'trap',
                     NotificationType(ObjectIdentity('IF-MIB',
                         'linkDown')))
)

if errorIndication:
    print(errorIndication)
\stoptyping

Like MIB variables, SNMP Notifications are identified by Object
Identifier (OID). Notifications are specified in MIBs along with MIB
variables that should be reported in notification message. This is MIB
specification of the above notification (from IF-MIB.txt):

\starttyping
linkDown NOTIFICATION-TYPE
    OBJECTS { ifIndex, ifAdminStatus, ifOperStatus }
    STATUS  current
    DESCRIPTION
            "A linkDown trap signifies that the SNMP entity, acting in
            an agent role, has detected that the ifOperStatus object for
            one of its communication links is about to enter the down
            state from some other state (but not from the notPresent
            state).  This other state is indicated by the included value
            of ifOperStatus."
    ::= { snmpTraps 3 }
\stoptyping

Note the OBJECTS clause: it makes SNMP Notification Originator
application gathering {\em ifIndex}, {\em ifAdminStatus},
{\em ifOperStatus} objects from its MIB variables store and reporting
their values in the notification.

\subsection[referring-to-mibs]{Referring to MIBs}

Surprisingly, SNMP can work without MIBs! Protocol itself does not
depend on them. MIBs are designed as complimentary database to make SNMP
data more human readable and convenient to deal with.

PySNMP library is designed to work without MIBs. You can always refer to
a MIB variable via Object Identifier:

\starttyping
ObjectType(ObjectIdentity('1.3.6.1.2.1.1.1.0'), OctetString('my host'))
\stoptyping

Whenever data type has to be specified, that can be done by passing
corresponding PyASN1 type. Otherwise both variable identifier and data
type associated with a variable would be looked up at MIB.

\starttyping
ObjectType(ObjectIdentity('SNMPv2-MIB', 'sysDescr', 0), 'my host')
\stoptyping

For MIB resolution to work, PySNMP application must load up MIB module
being looked up. Internally, PySNMP represents MIBs as Python modules
containing objects each representing MIB variable:

\starttyping
# pysnmp/smi/mibs/SNMPv2-MIB.py:
# ...
sysDescr = MibScalar((1, 3, 6, 1, 2, 1, 1, 1),
    DisplayString().subtype(subtypeSpec=
    ValueSizeConstraint(0, 255))).setMaxAccess("readonly")
\stoptyping

Conversion of original, ASN.1 MIB text files into PySNMP-compliant
Python modules is done behind the scenes by PySMI. Normally, PySNMP
users do not need to bother explicitly invoking PySMI services. PySNMP
would call PySMI automatically provided PySMI package is installed and
requested MIB module is available.

Users can point their PySNMP applications to either local MIB
repositories or remote ones that are accessible through HTTP or FTP:

\starttyping
from pysnmp.hlapi import *

errorIndication, errorStatus, errorIndex, varBinds = next(
    getCmd(SnmpEngine(),
           CommunityData('public'),
           UdpTransportTarget(('demo.snmplabs.com', 161)),
           ContextData(),
           ObjectType(ObjectIdentity('IF-MIB', 'ifInOctets', 1
               ).addAsn1MibSource('file:///usr/share/snmp',
               'http://mibs.snmplabs.com/asn1/@mib@'))
   )
)
\stoptyping

To figure out what MIBs are implemented by particular SNMP Agent,
Manager could query {\em SNMPv2-MIB::sysORTable} MIB variables:

\starttyping
$ snmpwalk -v2c -c public demo.snmplabs.com sysORTable
SNMP-MIB::sysORID.1 = OID: SNMPv2-MIB::snmpMIB
SNMP-MIB::sysORID.2 = OID: SNMP-VIEW-BASED-ACM-MIB::vacmBasicGroup
SNMP-MIB::sysORID.3 = OID: SNMP-MPD-MIB::snmpMPDCompliance
SNMP-MIB::sysORID.4 = OID: SNMP-USER-BASED-SM-MIB::usmMIBCompliance
SNMP-MIB::sysORID.5 = OID: SNMP-FRAMEWORK-MIB::snmpFrameworkMIBCompliance
...
\stoptyping

An efficient Manager could then optimize its behavior by fetching
required MIBs and querying MIB variables from those MIBs.

\subsection[snmp-agents]{SNMP Agents}

PySNMP could act as a full-blown SNMP Agent supporting most of standard
SNMP features pretty much out of the box. When building SNMP Agent, most
development efforts normally go into establishing mapping between
something to be monitored and MIB variable to be reported back to SNMP
Manager.

Multiple solutions exist at different levels of PySNMP library. Probably
the simplest approach is to subclass {\em MibScalarInstance()},
representing specific MIB variable, so that relevant information is
returned when queried or some action is performed when MIB variable is
modified:

\starttyping
# __MY_MIB.py:
# ...
class FileStorage(MibScalarInstance):
    file = '/var/run/agent/store'

    def getValue(self, name, idx):
        with f in open(self.file):
            return self.getSyntax().clone(f.read())

    def setValue(self, value, name, idx):
        with f in open(self.file, 'w'):
            f.write(value)
        return self.getValue(name, idx)

# Register our MIB variable implementation
mibBuilder.exportSymbols(
    '__MY_MIB', FileStorage((1,3,6,5,1,1,2,3,1), (0,), OctetString())
)
\stoptyping

Sometimes it is easier to bypass most of PySNMP SMI implementation by
provisioning your own instance of MIB Controller object to SNMP Agent:

\starttyping
class EchoMibInstrumController(AbstractMibInstrumController):
    def readVars(self, varBinds, acInfo):
            return [ (ov[0], OctetString(
                'Would read OID %s' % varBind[0]))
                for varBind in varBinds]

    def writeVars(self, varBinds, acInfo):
            return [ (varBind[0], OctetString(
                'Would set OID %s to %s' % varBind))
                for varBind in varBinds]

snmpContext.registerContextName(
    OctetString('my-context'), EchoMibInstrumController()
)

GetCommandResponder(snmpEngine, snmpContext)
SetCommandResponder(snmpEngine, snmpContext)
\stoptyping

\subsection[real-world-applications]{Real-world applications}

Over the years PySNMP developers created a couple of software products
based on PySNMP library. That kind of reality check experience
influenced PySNMP design a great deal.

{[}SNMP Simulator{]} is probably the most sophisticated free and open
source {[}SNMP Agent emulation software{]} at the time being. This tool
can create an illusion that large SNMP-managed network of various
devices exists in your virtual laboratory. Emulated devices try to look
live by reporting changing data. MIB variables can be configured to
change according to some formula. Modified data can be made persistent
by keeping MIB variables in SQL or noSQL database.

SNMP simulation toolkit also offers utilities that can capture SNMP
traffic from real SNMP Agents, store captured MIB variables so that they
could then be replayed by SNMP simulator. Another way to gather a
collection of MIB variables to simulate is to populate MIB it from ASN.1
MIB file.

You can experience SNMP Simulator yourself by talking to it on the
Internet. Configuration details could be found on {[}SimConfig{]}.

Another large project is {[}SNMP Proxy Forwarder{]}. That tool is
designed to work as cross-platform, transparent, application-level
firewall passing SNMP traffic between open and protected network
segments. The system is split into client and server parts connected
with each other over encrypted channel. Typically, server part resides
on public network responding to SNMP queries, while client part sits in
DMZ talking to SNMP devices on protected network. SNMP protocol version
translation could be performed on the fly, sophisticated SNMP data
filtering and modification could be performed, response caching and
request rate limiting could be set up to reduce SNMP load on end
devices.

Finally, we ship a pure-Python version of SNMP {[}command-line tools{]}
that try to mimic their Net-SNMP prototypes. Just do {\em pip install
pysnmp-apps} and off you go!

\subsection[references]{References}

\startitemize
\item
  {[}asyncio{]} https://docs.python.org/3/library/asyncio.html
\item
  {[}BER{]}
  https://en.wikipedia.org/wiki/X.690\#BER\_encoding
\item
  {[}command-line tools{]} https://pypi.python.org/pypi/pysnmp-apps
\item
  Dave Zeltserman: Practical Guide to Snmpv3 and Network Management
\item
  David T. Perkins, Evan McGinnis: Understanding SNMP MIBs
\item
  {[}Net-SNMP library{]} http://www.net-snmp.org
\item
  {[}PyASN1{]} http://pyasn1.sf.net
\item
  {[}PyCrypto{]} https://www.dlitz.net/software/pycrypto/
\item
  {[}PySMI{]} http://pysmi.sf.net
\item
  {[}PySNMP{]} http://pysnmp.sf.net
\item
  SNMP RFCs
  http://www.snmp.com/protocol/snmp\_rfcs.shtml
\item
  {[}RFC2587{]} http://www.ietf.org/rfc/rfc2578.txt
\item
  {[}RFC3411{]} http://www.ietf.org/rfc/rfc3411.txt
\item
  {[}RFC3412{]} http://www.ietf.org/rfc/rfc3412.txt
\item
  {[}RFC3413{]} http://www.ietf.org/rfc/rfc3413.txt
\item
  {[}RFC3414{]} http://www.ietf.org/rfc/rfc3414.txt
\item
  {[}SimConfig{]}
  http://snmpsim.sourceforge.net/public-snmp-simulator.html
\item
  {[}SNMP Agent emulation software{]}
  https://en.wikipedia.org/wiki/SNMP\_simulator
\item
  {[}SNMP Proxy Forwarder{]} http://snmpfwd.sf.net
\item
  {[}SNMP Simulator{]} http://snmpsim.sf.net
\item
  {[}Twisted{]} http://www.twistedmatrix.com
\stopitemize


\stoptext
