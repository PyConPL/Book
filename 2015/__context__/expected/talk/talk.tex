\usemodule[pycon-2015]
\starttext

\Title{Porozmawiaj ze swoją aplikacją}
\Author{Kamil Kujawiński, Marcin Najtkowski}
\MakeTitlePage

\subsection[systemy-ivr]{Systemy IVR}

IVR (ang. Interactive Voice Response) to nazwa systemu w
telekomunikacji, umożliwiającego interaktywną obsługę osoby dzwoniącej.
IVR ma więc funkcjonalność automatycznego Call center (lub jego części).
\footnote{https://pl.wikipedia.org/wiki/Interactive\letterunderscore{}Voice\letterunderscore{}Response}
System IVR może być zainstalowany bezpośrednio u klienta, w sieci PSTN,
a także jako aplikacja korzystająca z usług IaaS. Ostatniemu rozwiązaniu
poświęcimy ten artykuł oraz naszą prezentację.

Najpopularniejszą funkcją systemów IVR jest wybór języka, autoryzacja
klienta (np. przy pomocy numeru PIN), częściowa lub w pełni automatyczna
obsługa przez telefon. Wydawać by się mogło, że używanie telefonów do
komunikacji z systemami to przeszłość. Jednak różni ludzie mają różne
potrzeby, a szeroki wachlarz kanałów dostępu pozwala zwiększyć zasięg
naszych usług. Osoby często i długo podróżujące samochodem mogą
zaoszczędzić sporo czasu telefonicznie robiąc zakupy, płacić rachunki,
zamawiać ubezpieczenia, rezerować bilety do kina czy teatru. Starsze
osoby mające problemy z obsługą komputerów, mogą preferować korzystanie
z telefonów, które już dobrze znają.

\subsection[produkty-dostępne-na-rynku]{Produkty dostępne na rynku}

Rynek dostawców usług telekomunikacyjnych działających w chmurze jest
bardzo duży. Jest wiele firm, które różnią się możliwościami,
infrastrukturą i przede wszystkim cenami. Najważniejszym dla nich
rynkiem są Stany Zjednoczone i to tamtejsi klienci mogą wykorzystać
100\% dostępnych możliwości. Nawet Brytyjczycy nie mogą się cieszyć
pełnym wsparciem dla ich języka, ponieważ wiele opcji jest ograniczona
do amerykańskiej wersji języka angielskiego.

W tabeli poniżej przedstawiliśmy elementy, które mogą wpłynąć na wybór
dostawcy, ale przede wszystkim pokazać w jakich segmentach operują
dostawcy.

\bTABLE
\bTR
  \bTD \eTD
  \bTD \bf Plivo \eTD
  \bTD \bf Tropo \eTD
  \bTD \bf Twilio \eTD
\eTR
\bTR
  \bTD Ceny (polski numer miesięcznie / minuta rozmowy wychodzącej / minuta rozmowa przychodzącej) \eTD
  \bTD 0,8\$ / 2,35¢-7,3¢ / 0,5¢ \eTD
  \bTD 10\$ / 5-7¢ / 3¢ \eTD
  \bTD 1\$ / 2¢-7,5¢ / 0,75¢ \eTD
\eTR
\bTR
  \bTD Serwery \eTD
  \bTD dedykowane \eTD
  \bTD Voxeo \eTD
  \bTD AWS \eTD
\eTR
\bTR
  \bTD Paczka w PyPi \eTD
  \bTD plivo \eTD
  \bTD tropo-webapi-python \eTD
  \bTD twilio \eTD
\eTR
\bTR
  \bTD Synteza mowy / język polski \eTD
  \bTD 16 języków / tak \eTD
  \bTD 25 języków / tak \eTD
  \bTD 26 języków / tak \eTD
\eTR
\bTR
  \bTD Transkrypcja nagrań \eTD
  \bTD tylko angielski \eTD
  \bTD tylko angielski \eTD
  \bTD tylko angielski \eTD
\eTR
\bTR
  \bTD Rozpoznowanie mowy / język polski \eTD
  \bTD tak / nie \eTD
  \bTD tak / tak \eTD
  \bTD nie / nie \eTD
\eTR
\bTR
  \bTD Standardy \eTD
  \bTD PlivoXML \eTD
  \bTD VoiceXML, SSML, SRGS \eTD
  \bTD TwiML \eTD
\eTR
\bTR
  \bTD SMS / MMS \eTD
  \bTD tak / nie \eTD
  \bTD tak / nie \eTD
  \bTD tak / tylko USA i Kanada \eTD
\eTR
\bTR
  \bTD Numery telefonów / polskie numery \eTD
  \bTD 55 krajów / tak \eTD
  \bTD 28 krajów / tak \eTD
  \bTD 44 krajów / tak \eTD
\eTR
\bTR
  \bTD Instalacja on-premises \eTD
  \bTD tak (open source) \eTD
  \bTD tak (licencja) \eTD
  \bTD nie \eTD
\eTR
\bTR
  \bTD Najwięksi użytkownicy \eTD
  \bTD Mozilla, Netflix \eTD
  \bTD Deutsche Telekom, IBM \eTD
  \bTD Uber, Paypal, Coca-Cola \eTD
\eTR
\eTABLE

Od momentu przejęcia Tropo przez Cisco wiele informacji (serwery,
najwięksi użytkownicy) nie jest ujawnianych.

Zdecydowanie najpopularniejszym rozwiązaniem na rynku jest Twilio.
Jednak z różnych względów jest to rozwiązanie, które nie odpowiada
wszystkim użytkownikom. Twilio nie posiada własnej serwerowni - korzysta
z Amazon Web Services. Krytycy uważają, że wirtualne maszyny są
zdecydowanie mniej efektywne w przetwarzaniu dźwięku. Dla wielu firm
niedopuszczalne także jest udostępnianie swoich danych poza serwery
firmowe. Na takie zapotrzebowanie odpowiadają Tropo oferując możliwość
instalacji On-Premises w pakiecie Enterprise oraz Plivo, które otworzyło
kod źródłowy na licencji Mozilla Public License Version 1.1.

Tak jak w przeglądarkach internetowych uznanym standardem jest HTML, tak
w komunikacji głosowej człowiek - komputer standardem jest VoiceXML
(VXML)\footnote{https://en.wikipedia.org/wiki/VoiceXML}. Przy pomocy
tego formatu definiujemy syntezę i rozpoznawanie mowy, zarządzenie
rozmową oraz odtwarzanie dźwięku. Istnieją specyfikacje rozszerzające,
które pozwalają na definicję gramatyki rozpoznawanej mowy (SRGS), opis
sposobu syntezy mowy (SSML) oraz kilka innych. Specyfikacja VoiceXML
pozwala na stworzenie całej aplikacji i zainstalowanie jej po stronie
klienta, jednak większość dostawców nie wspiera w pełni tych standardów
stawiając na własne formaty oraz większą rolę API.

Większość graczy na rynku zrezygnowała z korzystania ze standardu
VoiceXML i utworzyła własne języki komunikacji z API - tak powstały
PlivoXML i TwiML. Możliwości tych języków są zbieżne z VoiceXML.

Nie zawsze stworzenie własnego Call Center wymaga pracy programistów, w
prostych przypadkach menadżer może sobie wyklikać cały scenariusz
rozmowy z klientem. W tym celu Twilio stworzyło otwarty projekt OpenVBX,
który oczywiście korzyta z usług Twilio. Użytkownik ma możliwość przy
pomocy techniki drag \& drop stworzyć system IVR z menu tonowymi,
nagrywaniem rozmów i przekierowywaniem połączeń do wybranych numerów.
Jak wspomnieliśmy OpenVBX jet ograniczony wyłącznie do korzystania z
Twilio, dlatego też powstała zjailbreakowana wersja, która współpracuje
z Tropo. Zawsze to jakiś wybór.

\subsection[twilio]{Twilio}

Jedną z bardziej popularnych, jeśli nie najpopularniejszą platformą
umożliwiającą implementację \quotation{interfejsu telefonicznego} we
własnych projektach jest - jak już wspomnieliśmy - Twilio. Rzut okiem na
listę firm i organizacji korzystających z ich rozwiązań rozwiewa
ewentualne wątpliwości - figurują tam m.in. ebay, Uber, airbnb, a nawet
filadelfijska policja. Z pewnością istotną rolę w zdobywaniu
popularności odgrywa dostępność i przystępność produktów spod szyldu
Twilio. Oferowane w modelu IaaS, pozwalają wzbogacić system o
funkcjonalność wykonywania i odbierania połączeń (do wykorzystania
out-of-the-box syntezator mowy w wielu językach) oraz wysyłania i
odbierania wiadomości tekstowych w sposób dziecinnie prosty. A to daje
szerokie pole do popisu - powiadomienia SMS, telekonferencje, alarmy,
automatyczna infolinia, autoryzacja i wiele, wiele innych, gotowe do
użycia w przeciągu kilku godzin. Najważniejsze jest to, że nie trzeba
się martwić o skomplikowaną infrastrukturę. Oczywiście usługi nie są
darmowe, jednak ich ceny wydają się rozsądne nawet dla prywatnych
projektów czy małych firm, nie wspominając o porównaniu do kosztów
zbudowania podobnych rozwiązań od podstaw. Warto przy okazji zauważyć,
że Twilio dość aktywnie wspiera developerów i uczestniczy m.in. w
organizacji hackathonów - na takiej formie promocji korzystają obie
strony.

Połączenia i wiadomości wysłane za pośrednictwem Twilio dotrą do niemal
200 krajów na całym świecie, natomiast odbiór poprzez lokalne numery
możliwy jest w ponad 40 krajach. Możemy więc założyć kilka numerów i
udostępnić własną aplikację na lokalnych rynkach w Polsce, Niemczech,
Meksyku, Portoryko i Hong Kongu.

Komunikacja z Twilio odbywa się dwukierunkowo. Akcje wykonywane przez
aplikację wywołują metody RESTowego API (\type{api.twilio.com}). Za
reakcję na zdarzenia odpowiedzialne jest już nasze własne API, do
którego zapytania kieruje Twilio, oczekując odpowiednich komunikatów
XML. Korzystanie z obu wymienionych ułatwia udostępniona przez Twilio
biblioteka (\type{pip install twilio}). Moduł \type{twilio.rest}
wykorzystujemy np. w celu zainicjowania połączenia czy wysłania
wiadomości. Zaś do generowania XML-owych\ldots{} tzn. TwiML-owych
odpowiedzi obsługujących odbiór połączeń i wiadomości, używamy
\type{twilio.twiml}.

Aby rozpocząć, należy w pierwszej kolejności - tu bez niespodzianek -
założyć konto. Następnie wybieramy kraj i numer telefonu (w zależności
od lokalizacji, niektóre numery umożliwiają tylko obsługę wiadomości
tekstowych lub połączeń). Pozostaje jeszcze uzupełnić adres naszego API,
zapisać SID oraz token.

Krótki przykład najlepiej zobrazuje tę prostotę, którą tak się
zachwycamy.

\starttyping
>>> from twilio.rest import TwilioRestClient
>>> client = TwilioRestClient(MY_SID, MY_TOKEN)
>>> client.messages.create(to='+48123456789', from_=MY_NUMBER,
        body='Hello world!')
\stoptyping

Tym sposobem zostało wykonane zapytanie do RESTowego API, co z kolei
poskutkowało wysłaniem SMSa.

IVR wymaga już nieco więcej wkładu, niemniej nadal jest stosunkowo łatwy
do zaimplementowania. Poniżej fragment kodu - widok Django - stanowiący
bazę pod IVR. Dzwoniący wybiera dział, z którym chce się połączyć. Po
dokonaniu wyboru rozmowa zostaje przekierowana na odpowiedni numer.

\starttyping
from django.core.urlresolvers import reverse
from django.http import HttpResponse
from django.views.generic import View
from twilio import twiml

class IVR(View):
    """Główny widok IVR"""

    # Adresy kolejnych widoków, w zależności od wybranej opcji
    choices = {
        '1': reverse('incoming_complaints'),
        '2': reverse('incoming_sales'),
        '3': reverse('incoming_consultant'),
        '0': reverse('incoming_tech'),
    }

    def post(self, request):
        response = twiml.Response()
        choice = request.POST.get('Digits')
        if choice:
            if self.choices.get(choice):
                response.redirect(self.choices[choice])
                return HttpResponse(response)
            else:
                response.say(
                    'Wybrano niewłaściwą opcję.',
                    voice='alice',
                    language='pl-PL',
                )

        with response.gather(
            numDigits=4,
            action=reverse('ivr'),
            method='POST',
            timeout=10,
        ) as response:
            response.say(
                'Aby połączyć się z działem reklamacji, wybierz jeden. '
                'Aby połączyć się z działem handlowym, wybierz dwa. '
                'Aby połączyć się z działem technicznym, wybierz trzy. '
                'W celu połączenia z konsultantem, wybierz zero.',
                voice='alice',
                language='pl-PL',
                loop=3,
            )
        return HttpResponse(response)
\stoptyping

Dzięki prostej obsłudze przekierowań aplikację można dowolnie
rozbudowywać, dokładając kolejne widoki zwracające komunikaty TwiML. A
kiedy głos Alice stanie się męczący, nic nie stoi na przeszkodzie, by
nagrać własne komunikaty.

Co dalej? Twilio przygotowało dość obszerną dokumentację. I choć
nawigacja wydaje się umiarkowanie intuicyjna, dokumentacja pokrywa
zdecydowaną większość dostępnych funkcjonalności oraz ich kombinacji i
stanowi dobry materiał do dalszej eksploracji tematu. Do czego gorąco
zachęcamy.

\subsection[bibliografia]{Bibliografia}

\startitemize
\item
  https://www.twilio.com/
\item
  https://www.plivo.com/
\item
  https://www.tropo.com/
\stopitemize


\stoptext
