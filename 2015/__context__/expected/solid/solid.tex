\usemodule[pycon-2015]
\starttext

\Title{SOLID Python}
\Author{Ignacy Sokołowski}
\MakeTitlePage

SOLID is an acronym for five basic principles of Object Oriented
Programming. The principles intend to make software entities easy to
understand, maintain, extend, unit-test and reuse in different contexts.

They are mostly used in strongly typed languages such as Java or C++.
Probably that's why Python developers don't follow them too often or
they do but didn't ever realize it.

In this article, I will briefly explain three of the SOLID principles
and show you how to follow them in Python code.

{\bf SOLID} is actually an acronym of acronyms:

\startitemize
\item
  {\bf SRP} - Single Responsibility Principle
\item
  {\bf OCP} - Open/Closed Principle
\item
  {\bf LSP} - Liskov Substitution Principle
\item
  {\bf ISP} - Interface Segregation Principle
\item
  {\bf DIP} - Dependency Inversion Principle
\stopitemize

The principles were collected by Robert C. Martin, also known as
\quotation{Uncle Bob}, in early 2000s.

\subsection[single-responsibility-principle]{Single Responsibility
Principle}

The Single Responsibility Principle states that every class should be
responsible for only one thing. We could also extend it to methods and
in languages like Python, to functions.

What does it mean to be responsible for only one thing?

\section[not-just-to-do-only-one-thing]{Not just to \quotation{do
only one thing}}

Say you have a function for sending e-mail reports. It gets data for the
report, then formats it and finally sends it via e-mail.

So, it does three different things.

But if each of these three things is implemented in a separate function
and the main function just calls them, then it will not be responsible
for how they are implemented.

\starttyping
def send_email_report(email, time_period):
    report = prepare_report(time_period)
    report_body = format_report(report)
    send_email(email, report_body)
\stoptyping

The function above does not have to know how to get the data nor how to
format the report. It just tells these functions to do their job. It is
also easier to see the flow of this function, when you are not
distracted by details. If you want to see how exactly the report is
formatted, you can look at the \type{format_report()} function.

\section[one-level-of-abstraction]{One level of abstraction}

The function for sending e-mail reports is responsible for only one
level of abstraction.

In order to send a report, we need to perform three tasks.

\startitemize
\item
  Get the data
\item
  Format the report
\item
  Send it via e-mail
\stopitemize

So, here just {\em sending report} is the highest level of abstraction
and it performs three tasks on a lower level of abstraction. Each of
these three tasks need to perform several different tasks on an even
lower level.

To get the report data, we need to:

\startitemize
\item
  Build an SQL query
\item
  Send the query to a database
\item
  Parse the results on data objects
\stopitemize

Each of these tasks will be implemented in a separate function and they
all will be invoked by \type{prepare_report()}, the same way as
\type{send_email_report()} does.

\section[one-reason-to-change]{One reason to change}

Uncle Bob defines responsibility as a reason to change. So, a class or
function should have no more than one reason to change.

If you'd like to change the way reports are formatted, you should not
need to change the class responsible for preparing report data. If you'd
like to change the database for articles, you should not need to change
controller for displaying them.

In our example, if we'd like to change the way reports are formatted, we
change only the \type{format_report()} function.
\type{send_email_report()}, \type{prepare_report()} and
\type{send_email()} stay untouched.

\subsection[openclosed-principle]{Open/Closed Principle}

As I mentioned at the beginning of the article, one of the goals of
SOLID is extensibility. If we'd like to introduce a new class which
differs from an existing one by only a small detail, we should not have
to copy all the code from one class to another.

The Open/Closed Principle states that \quotation{software entities
(classes, modules, functions, etc.) should be open for extension, but
closed for modification}.

In our example, we have a function for formatting reports.

The \type{send_email_report()} function explicitly calls
\type{format_report()}. If we wanted to change the way reports are
formatted, we would change the \type{format_report()} function. That's
OK if it's in our own codebase and if we always want to format reports
in the same way.

But what if all these functions are in a third party library and we can
only import and call them? Then, the only way to change the way of
formatting reports would be to implement our own version of
\type{format_report()} and then, to rewrite \type{send_email_report()}
to invoke it.

If the code is in our own codebase, there could be another problem with
extending it. What if we had two different report formats? Sometimes we
want to use the existing \type{format_report()} function, so we don't
want to change it. But sometimes we'd like to use another one. There's
no way to tell \type{send_email_report()} to format reports differently.

It means that our function is closed for extension. It is also
automatically open for modification because the only way to use a
different report formatter is to modify our function.

One of the advantages of classes over functions is that they can be
easily extended by inheritance. We could move our functions into methods
of a class:

\starttyping
class EmailReporter:

    def report(self, email, time_period):
        report = self._prepare_report(time_period)
        report_body = self._format_report(report)
        self._send_email(email, report_body)

    def _prepare_report(self, time_period):
        # ...

    def _format_report(self, report):
        # ...

    def _send_email(self, email, body):
        # ...
\stoptyping

Now, if we'd like to modify the way of formatting reports, we can create
a subclass and override or extend only one method:

\starttyping
class EmailReporter2(EmailFormatter):

    def _format_report(self, report):
        # The new implementation.
\stoptyping

But that's not the best way to go. We're back to the problem of a class
with more than one responsibility. There could be more than one place
where we send messages to our users. We wouldn't like to use
\type{EmailReporter} to send newsletter or registration e-mails.

\subsection[dependency-inversion-principle]{Dependency Inversion
Principle}

Our functions are dependent on each other. The problem is the way the
dependencies are defined. In fact, they are now hard-coded into the
\type{send_email_report()} function.

The Dependency Inversion Principle proposes to invert the way we define
dependencies. Instead of defining them inside function or method
definitions, we should be able to set them from outside. How can we do
that?

One way would be to use dependency injection. We could add new arguments
to \type{send_email_report()} which would specify which functions to
use:

\starttyping
def send_email_report(
        email, time_period,
        prepare_report, format_report, send_email):
    report = prepare_report(time_period)
    report_body = format_report(report)
    send_email(email, report_body)
\stoptyping

I personally don't like injecting dependencies into functions or methods
as it pollutes function's interface and requires me to pass all the
dependencies every time I want to use the function.

A better way would be to use a class instead of a function:

\starttyping
class EmailReporter:

    def __init__(self, report_preparator, report_formatter, email_sender):
        self._report_preparator = report_preparator
        self._report_formatter = report_formatter
        self._email_sender = email_sender

    def report(self, email, time_period):
        report = self._report_preparator.prepare(time_period)
        report_body = self._report_formatter.format(report)
        self._email_sender.send(email, report_body)
\stoptyping

Now, the public interface for \type{EmailReporter.report()} does not
include dependencies. No matter what report formatter we want to use, we
invoke it the same way. We don't have to pass the dependencies in every
call.

As long as our class is stateless, we can create one instance of it for
each variation of dependencies:

\starttyping
# Common dependencies
report_preparator = ReportPreparator()
email_sender = EmailSender()

# Two kinds of reporters
email_reporter_html = EmailReporter(
    report_preparator=report_preparator,
    report_formatter=HTMLReportFormatter(),
    email_sender=email_sender,
)
email_reporter_pdf = EmailReporter(
    report_preparator=report_preparator,
    report_formatter=PDFReportFormatter(),
    email_sender=email_sender,
)
\stoptyping

And we can call \type{email_reporter_html.report(email, time_period)}
for any e-mail address or time period.

\section[stateless-classes]{Stateless classes}

For many Python developers, this kind of class can look a bit weird. We
usually use classes only when we need to manage the state of an object.
Otherwise, we just use pure functions.

It is important to understand the difference between two types of
objects: those that store and represent data, and those that perform
some actions.

The former ones store data in their attributes. They don't even have to
expose any public methods, the most important job they do is to
aggregate data into one object.

\starttyping
class Person:

    def __init__(self, first_name, last_name, email, gender, spouse=None):
        self.first_name = first_name
        self.last_name = last_name
        self.email = email
        self.gender = gender
        self.spouse = spouse
\stoptyping

If they have any public methods, they should not interact with the
outside world. They should either change the object's state or return
some data computed from the object's attribute values:

\starttyping
class Person:

    def __init__(self, first_name, last_name, email, gender, spouse=None):
        self.first_name = first_name
        self.last_name = last_name
        self.email = email
        self.gender = gender
        self.spouse = spouse

    @property
    def is_male(self):
        return self.gender == 'male'

    @property
    def is_female(self):
        return self.gender == 'female'

    @property
    def full_name(self):
        return '{} {}'.format(self.first_name, self.last_name)

    def marry(self, fiancee):
        self.spouse = fiancee
        if fiancee.is_male:
            self.last_name = fiancee.last_name
\stoptyping

If we needed to be able to send e-mails to a \type{Person}, we should
not implement \type{send_email()} method in the \type{Person} class but
rather create a new \type{EmailSender} class with \type{send()} method,
to which we would pass a \type{person} instance or just its e-mail
address.

Why not to pass the person's e-mail to \type{EmailSender.__init__}?
Because then we would need a separate sender for every e-mail recipient.
Otherwise, we can have only one instance to send e-mails to any person.
We don't ever have to instantiate \type{EmailSender} in a method where
we want to send an e-mail.

\type{EmailSender} is the second type of class: the one that can perform
an action. Such classes act like functions. Since we want to follow the
SRP, they will usually have only one public method. The only thing that
varies them from functions is the possibility to configure them during
instantiation.

Note that in Python, if a class implements \type{__call__()} method, its
instance does not differ at all from a regular function. We could do
that with our \type{EmailReporter} class by renaming its \type{report()}
method to \type{__call__()}. You could treat such classes as factories
of functions:

\starttyping
send_email_report = EmailReporter(...)
send_email_report(email, time_period)
\stoptyping

\section[interfaces]{Interfaces}

When one class needs to use another class to perform a task, it accepts
it as an argument to the \type{__init__} method. It does not know of
what type the dependency will be. The only thing it has to know about
the dependency is the interface of its public methods.

Many times I have seen injecting dependencies as classes rather than as
instances, for example:

\starttyping
class EmailReporter:

    def __init__(self, report_formatter_class, ...):
        self._report_formatter = report_formatter_class()
\stoptyping

Here we depend on signature of the class's \type{__init__}. It is
important to know that the initializer is not a part of the public
interface. We could have many different formatters which will require
different arguments to be constructed but as long as they share the same
signature for public methods, they implement the same interface.

The \type{EmailReporter} requires a report formatter. It does not care
what type of reporter will be provided so it does not know its concrete
type nor the arguments to its \type{__init__} method.

\starttyping
class EmailReporter:

    def __init__(self, report_formatter, ...):
        self._report_formatter = report_formatter
        # [...]

    def report(self, email, time_period):
        # [...]
        self._report_formatter.format(data)
        # [...]
\stoptyping

It only knows that it will have a \type{format()} method which accepts
one argument. Thanks to this, it is easy to implement any kind of
formatter that will satisfy the reporter.

\starttyping
class MarkdownReportFormatter:

    def format(self, report):
        offers = '\n\n'.join([
            '## {offer.title}\n\n{offer.text}'.format(offer=offer) for
            offer in report.offers
        ])
        return '# {title}\n\n{offers}'.format(
            title=report.title,
            offers=offers,
        )
\stoptyping

\starttyping
class JinjaTemplateReportFormatter:

    def __init__(self, environment, template_name):
        self._environment = environment
        self._template_name = template_name

    def format(self, report):
        template = self._environment.get_template(self._template_name)
        return template.render(report=report)
\stoptyping

We have two different formatters with two different initializers. As
they share the same signature for the \type{format()} method, we can use
both of them as a dependency for the \type{EmailReporter}:

\starttyping
EmailReporter(
    report_preparator=ReportPreparator(),
    report_formatter=MarkdownReportFormatter(),
    email_sender=EmailSender(),
)
EmailReporter(
    report_preparator=ReportPreparator(),
    report_formatter=JinjaTemplateReportFormatter(
        environment=jinja2.Environment(
            loader=jinja2.FileSystemLoader('path/to/templates'),
        ),
        template_name='report.html',
    ),
    email_sender=EmailSender(),
)
\stoptyping

\section[loose-coupling]{Loose coupling}

The main goal of the Dependency Inversion Principle, is to help writing
loosely coupled software. If our modules and classes are less coupled,
it is easier to change a part of code without a need to touch the other
parts. Modules also become portable and easier to test.

You have already noticed that our classes for sending e-mail reports do
not depend on each other. The \type{EmailReporter} does not know about
the existence of the \type{HTMLReportFormatter} nor about the
\type{EmailSender}. If we implemented each of these classes in a
separate module, the modules would not import from each other.

Our classes automatically become more reusable. As the
\type{EmailSender} does not know that it will send reports, it can be
used to send any kinds of e-mail messages. Thanks to low coupling of
modules, classes can be easily ported between different projects. If we
have multiple applications where we send e-mails, we can release our
\type{EmailSender} in an external library.

\section[testing]{Testing}

When we want to unit-test classes in isolation, we often use mocking. We
use \type{mock.patch()} to patch the behavior of a method which should
not be executed in a particular test. When our classes have many
dependencies which are tightly-coupled, we need to patch a lot.

If we wanted to test our \type{send_email_report()} function, we would
need to patch \type{prepare_report()}, \type{format_report()} and
\type{send_email()} in every test. We need patching because the code is
closed for extension.

On the other hand, if we wanted to unit-test the \type{EmailReporter},
instead of mocking, we could simply pass different instances of our
dependencies than those used in production code.

\starttyping
reporter = EmailReporter(
    report_preparator=FakeReportPreparator(),
    report_formatter=FakeReportFormatter(),
    email_sender=FakeEmailSender(),
)
\stoptyping

If we don't want to create fake classes just for the sake of testing, we
can still use mocks but we don't have to patch.

\starttyping
report_preparator_mock = mock.create_autospec(ReportPreparator)
report_preparator_mock.prepare.return_value = Report(...)
report_formatter_mock = mock.create_autospec(ReportFormatter)
report_formatter_mock.format.return_value = '<formatted report>'
reporter = EmailReporter(
    report_preparator=report_preparator_mock,
    report_formatter=report_formatter_mock,
    email_sender=mock.create_autospec(EmailSender),
)
\stoptyping

Our unit-tests can now focus on behavior of the class under test and are
independent of the other classes. But are such tests enough to be sure
that our software works as expected? Of course not. We also need to test
if the classes which we tie together for production code, integrate with
each other as expected. That's what we do in integration tests.

Loose coupling helps us to separate unit-tests from integration tests.
Unit-tests assert that a single class's logic is correct. Such tests can
be ported to any other project along with the class under test as they
are independent of the whole application and the context in which it is
used.

Integration tests are written against production-ready instances, with
all the dependencies set. In our case, we would write integration tests
for the \type{email_reporter_html} instance, where we would patch only
the e-mail server. Such tests also make sure that the functionality of
e-mail reporting works even if we change the report formatter or the
data preparator.

\section[dependency-injection-container]{Dependency Injection
Container}

OK, now when we have our loosely coupled classes, we have to somehow
wire them together. Where should we do that?

Dependency gluing should be done at the highest level, nearest the main
entry point of your application. If it's a simple application, you can
do that manually, for example in your \type{main()} function:

\starttyping
def main(args=sys.argv):
    config_path, email, time_period = args[1:]

    config = parse_config_file(config_path)
    db = Database(url=config['db_url'])
    offers = OfferRepository(db)
    email_reporter = EmailReporter(
        report_preparator=ReportPreparator(offers=offers),
        report_formatter=JinjaTemplateReportFormatter(
            environment=jinja2.Environment(
                loader=jinja2.FileSystemLoader(config['templates_dir']),
            ),
            template_name='report.html',
        ),
        email_sender=EmailSender(
            smtp_uri=config['smtp_uri'],
            from_address=config['from_email'],
        ),
    )

    email_reporter.report(email, time_period)

if __name__ == '__main__':
    main()
\stoptyping

But as the number of your dependencies and connections between them
grow, it will no longer be so easy to maintain them in this form. That's
the moment when you need a Dependency Injection Container.

DI Container is an object which stores all the dependencies of your
application in one place. There are multiple implementations of
containers in several programming languages like Java, .NET, PHP, and
even in Python.

Here's an example of a Knot container:

\starttyping
def make_container(config):
    c = knot.Container(config)

    @knot.service(c)
    def db(c):
        return Database(url=c['db_url'])

    @knot.service(c)
    def offers(c):
        return OfferRepository(db=c('db'))

    @knot.service(c)
    def report_preparator(c):
        return ReportPreparator(offers=c('offers'))

    @knot.service(c)
    def jinja_env(c):
        return jinja2.Environment(
            loader=jinja2.FileSystemLoader(c['templates_dir']),
        )

    @knot.service(c)
    def report_formatter(c):
        return JinjaTemplateReportFormatter(
            environment=c('jinja_env'),
            template_name='report.html',
        )

    @knot.service(c)
    def email_sender(c):
        return EmailSender(
            smtp_uri=c['smtp_uri'],
            from_address=c['from_email'],
        )

    @knot.service(c)
    def email_reporter(c):
        return EmailReporter(
            report_preparator=c('report_preparator'),
            report_formatter=c('report_formatter'),
            email_sender=c('email_sender'),
        )

    return c
\stoptyping

And here's how to use it:

\starttyping
def main(args=sys.argv):
    config_path, email, time_period = args[1:]

    config = parse_config_file(config_path)
    dependencies = make_container(config)

    dependencies('email_reporter').report(email, time_period)

if __name__ == '__main__':
    main()
\stoptyping

One of the advantages of such container is that you can give it
different configuration for tests and write integration tests against
particular dependencies.

\starttyping
def test_email_reporter():
    container = make_container({'db_url': 'sqlite:///'})
    email_reporter = container('email_reporter')
    email_reporter.report(...)
\stoptyping

\section[summary]{Summary}

For some of you, this way of thinking about object-oriented design may
be completely new and quite shocking. You may find it \quotation{not
pythonic} or think that it's over engineering.

I was shocked too, when I first saw it. It was not easy to get rid of my
old habits. But I decided to try it out and now I see how much easier
programming and testing can be when applying these principles,
especially when I'm working on a rapidly growing project for which
requirements and business decisions change quite often.

What I showed you in this article, is just the tip of the iceberg. I
encourage you to read more about SOLID principles and design patterns. I
promise it will pay off!

\section[further-reading]{Further reading}

\startitemize
\item
  Robert C. Martin - \quotation{Design Principles and Design
  Patterns}\crlf
http://www.objectmentor.com/resources/articles/Principles\_and\_Patterns.pdf
\item
  Robert C. Martin - \quotation{Agile Software Development: Principles,
  Patterns, and Practices}
\item
  Robert C. Martin - \quotation{Clean Code. A Handbook of Agile Software
  Craftsmanship}
\stopitemize


\stoptext
