\usemodule[pycon-2015]
\starttext

\Title{with modern\_peripherals: Python and Flask}
\Author{Piotr Dyba}
\MakeTitlePage

\subsection[introduction]{Introduction}

Auto-scrolling sites, glance-following ads, and gesture friendly web
pages are coming! Over the last few years three products emerged that
enable interaction with computer in a new way: Myo Armband, Leap Motion
Controller and EyeTribe. The integration between the aforementioned
products, other devices and web applications is based on the web sockets
protocol. Let's start with a really short introduction to the Web
Sockets, after that I'll get through implementation, wrappers and an
imaginary application for each of the new devices.

\subsection[web-sockets-and-python-socket-solution]{Web Sockets and
Python socket solution}

In short, web sockets is a protocol providing full-duplex communication
channels over a single TCP connection available in web browsers.

\placefigure[here,nonumber]{01 Web Sockets
Diagram}{\externalfigure[./001_web_sockets.png]}

It starts with a handshake which upgrades the HTTP connection to use the
Web Sockets protocol. This enables message exchange to and from a
server. Such exchange can be triggered by a user or by some programmer
defined event. It's what interests us most. Communication channel is
terminated after one of the sides sends a kill message. Builtin Python
library is nice but writing JS from scratch on front end would be time
consuming so I did not use it. Because of that, I decided to use the
Socket.IO which enables real-time bidirectional event-based
communication. It works on every platform, browser or device. It's a
powerful solution but it requires additional socket server. Nevertheless
\useURL[url1][https://github.com/miguelgrinberg/Flask-SocketIO][][Flask-SocketIO]\from[url1]
was my choice for implementing sockets and serving the application,
all-in-one solution, and I really like using the simplicity of Flask.
The beauty of the @app.route decorators was applied to that extension,
so it is possible to use it in the similar way @socketio.on().

Flask decorator:

\starttyping
@app.route('/')
def index():
    return render_template('leap_motion.html')
\stoptyping

Event in backend activated from frontend and SocketIO decorator:

\starttyping
@socketio.on(‘connect’, namespace='/test')
def disconnect():
    print('Client connected’)
\stoptyping

Event on frontend activated from backend:

\starttyping
def send_msg(msg):
    emit('msg_event', msg)
\stoptyping

The front end is an out-of-the-box solution and usage is as simple as
attaching single \type{<script>} tag library. When using with
Flask-SocketIO, we have to use version 0.9.16 of JS SocketIO library,
because of incompatibility with version 1.0.

Frontend code:

\starttyping
$(document).ready(function() {
    var namespace = '/test';
    var socket = io.connect('http://' + document.domain + ':' +
        location.port + namespace);
    // connecting event
    socket.on('connect', function () {
        socket.emit('connect');
        $('#status').append('connected');
    });
    // event on frontend activated from backend
    socket.on('even_name', function (){
        console.log("hi");
        return 0;
    });
    // event in backend activated from frontend
    $('#some_button').click(function() {
        socket.emit('from_end_event');
        return 0;
    });
});
\stoptyping

\subsection[eye-tribe]{Eye Tribe}

EyeTribe is an eye tracker hardware that is advertised as a cheap
solution, despite the self-describing name, what is eye tracking? Eye
tracking is the process of using sensors to estimate where someone is
looking - a point of gaze. Eye tribe uses infrared sensors for that
purpose. Eye tracking can be used in a wide variety of applications
typically categorized as active or passive. Active applications involve
device control, for example aiming in games, eye activated login or
hands-free typing. Even in '80s and '90s there were a few photo cameras
that used it, like Canon 1V for choosing the camera's focus point.
Passive applications include performance analysis of design, layout and
advertising. The main usage of the eye tracking for web developers is
analysis of the web application that will improve ads, layout and
design. Controlling the site with eyes will be the next step. It will be
especially useful for handicapped people with severe body or mind
damage. Think about Steven Hawking, and how well he is doing despite his
condition, but using eye tracking in such a manner needs some level of
focus. I think that when eye trackers will get more popular, or using
the built-in cameras for this purpose will be more efficient and common.
When we will get used to the technology, its advantages, some people
will start using it on daily basis, the eye following ads will emerge.
Closed only by blinking three times or roll eyes gestures.

\subsection[wrapper]{Wrapper}

The Python wrapper is straightforward and one file only.
\useURL[url2][https://github.com/baekgaard/peyetribe][][Eye Tribe
wrapper]\from[url2] connects directly to the Eye Tribe daemon. The Eye
Tribe was the easiest device to use with Python. The wrapper provides
only information about point of gaze and eyes location for each eye
separately, but that is all what we need.

\subsection[js-vs-python]{JS vs Python}

I've compared efficiency of Python solution against pure JavaScript one,
during two tests, the first one was displaying live data for each
approach, the second one was only collecting data. The main advantage of
pure JS solution was ability to use it live without major glitches. The
Python solution still relied on JS to display the current location of
gaze so it was disadvantage from the start. The situation was quite
opposite when I used Python only to gather the data and JS to display
the outcome overlaid on the survived website. As expected, the best
solution depends on the type of application and its requirements.

\subsection[hypothetical-application]{Hypothetical application}

Imagine an application that would measure and rate user's art knowledge
by the way he looked at a picture or painting. What did he focus on, and
how long was he really looking at it. Wouldn't it be nice to be able to
grade the ones photographic skill not only by how he takes photos but
also how he looks at them.

\subsection[myo-armband]{Myo Armband}

The Myo Armband was introduced as a new way to interact with a computer,
but now it is advertised more like presentation tool which it really
excels in, I often use it when presenting. If you desire stealth it is
possible to go thorough the whole presentation without audience knowing
how the slides are being changed. The armband reads your muscle activity
using electromyography a sensor to read electric impulses going through
the muscles, it also have accelerometer so you can control your software
with gestures and motion. The first possible use that comes to my mind
is to simply hand over the control over the application to Myo, it would
be handy when the application does not need text input. It would be
really useful solution in hostile environment where using keyboard would
be problematic like extreme cold or volcanic heat. The second use case
would be when one would need gesture control but without touching the
screen for example surgeon or cook would benefit from that. For
navigation functions imagine to see the login screen on metal band fun
site you would have to make a 'bunny gesture". Last but not least it
could be an addition to an application that would require user to do few
exercises when he works too much by making him or her to do gestures to
relax muscles. By the way it is possible to use two armbands at once.

\subsection[wrapper-1]{Wrapper}

The wrapper is an active project with a very nice and helpful initiator
and it is bigger than the Eye Tribe's one.
\useURL[url3][https://github.com/NiklasRosenstein/myo-python][][Myo
Armband wrapper]\from[url3] It provides information about how we move
our hand, its acceleration and if one of the predefined gestures was
made.

\subsection[hypothetical-application-1]{Hypothetical application}

A \quotation{real tinder} (Tinder is and dating application where you
can like or dislike someone by swiping the screen) where you could
dislike with a slap in the face gesture or like her/him with grab or
heart gesture. How cool is that?

\subsection[leap-motion]{Leap Motion}

Leap motion brings gestures known from tablets to notebooks and
desktops. The Leap Motion Controller tracks both hands in front of the
screen. From a developer's perspective, as Myo allows us to use
gestures, previously restricted to touch devices, on desktops. Leap
works using 2 infrared cameras with infrared lights. The magic of
converting to the 3d image/data is hidden within the algorithms, but due
to limits on usb 3.0 it is available in 60cm range only. The only use
for this device that I came with, was to control a web page or extend it
by adding additional features

\subsection[wrapper-2]{Wrapper}

The Leap manufacturer provides the Python wrapper for the device which
is really nice. The wrapper is really pretty and also provides a really
comprehensive example of use. This wrapper needs LeapPython.so and
dynamic library libLeap.dylib to work. It is really well made, it is by
far the best one out of this three. It gives all possible information
about what your hands are doing from bones joints up to defined
gestures.

\subsection[hypothetical-application-2]{Hypothetical application}

Imagine an application lets call it \quotation{Touch your pizza} that
allows you to put the ingredients you want on the pizza virtually on
your desktop, because Leap recognizes not only fingers but also tools
like rice sticks or forks.

\subsection[short-summary]{Short summary}

To sum up as all three devices can be implemented in similar way using
web sockets their usage will differ among the because of their different
purposes. I definitely had a lot of fun playing with them and figuring
the possible usages. I hope to see their usage becoming more common.

\subsection[sources]{Sources}

\startitemize
\item
  Myo Armband \hyphenatedurl{https://www.thalmic.com/myo/}
\item
  Leap Motion \hyphenatedurl{https://www.leapmotion.com}
\item
  EyeTribe \hyphenatedurl{https://theeyetribe.com}
\item
  flask-SocketIO documentation
  \hyphenatedurl{https://flask-socketio.readthedocs.org/en/latest/}
\item
  Flask docs \hyphenatedurl{http://flask.pocoo.org/docs/0.10/}
\item
  Socket.IO \hyphenatedurl{http://socket.io}
\stopitemize


\stoptext
