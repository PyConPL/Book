\usemodule[pycon-2016]
\starttext

\Title{Object calisthenics}
\Author{Paweł Lewtak}
\MakeTitlePage

\subsection[czym-jest-kalistenika]{Czym jest kalistenika?}

Według definicji podanej przez Wikipedię{[}1{]} kalistenika to
\quotation{aktywność fizyczna polegająca na treningu siłowym opartym o
ćwiczenia z wykorzystaniem własnej masy ciała}. Kiedy mówimy o
kalistenice, najprościej jest zwizualizować to pojęcie przywołując
gladiatorów. Aż do XX wieku nie istniały siłownie - żadnych bieżni
mechanicznych, rowerów stacjonarnych, hantli. Był tylko człowiek i masa
jego ciała, którą mógł wykorzystać do ćwiczeń. Efektem takich ćwiczeń,
bez wykorzystania dodatkowego sprzętu, była czysta siła, a masa
mięśniowa stanowiła dodatkowy efekt uboczny - nie była celem samym w
sobie.

Podobnie sprawa ma się z kalisteniką obiektową. Pod tym pojęciem kryją
się proste ćwiczenia. Zupełnie podstawowe reguły, które nie wymagają
nadzwyczajnych umiejętności, ani dodatkowych narzędzi. Celem jest
poprawa jakości kodu, pisanie kodu prostszego w utrzymaniu, prostszego w
zrozumieniu, możliwego do ponownego użycia oraz takiego, który można
łatwo testować.

Kalistenika obiektowa została pierwotnie przedstawiona w eseju Jeffa
Bay'a i opublikowana w książce pt. {\em ThoughtWorks Anthology}{[}2{]}.
Początkowo została pomyślana z przeznaczeniem dla zastosowania w Javie,
natomiast później zaadaptowano ją na potrzeby innych języków
programowania.

Dosyć popularnym powiedzeniem jest, że kod jest dużo częściej czytany
niż pisany. Dlatego tak ważne jest, by kod ten był łatwy do
przeczytania, zrozumienia, testowania i ponownego zastosowania w innym
miejscu. O podobnym celu jest mowa w {\em Czystym kodzie}{[}3{]} Roberta
C. Martina.

Bez zbędnego przedłużania, zobaczmy konkretnie jakie reguły składają się
na kalistenikę obiektową.

\subsection[nie-używaj-więcej-niż-jednego-poziomu-wcięcia-na-metodę.]{1.
Nie używaj więcej niż jednego poziomu wcięcia na metodę.}

Zbyt wiele zagnieżdżeń w ramach jednej funkcji lub metody źle wpływa na
czytelność i prostotę utrzymania. Wiele poziomów pętli i warunków
sprawia, że działamy na różnych poziomach abstrakcji, a to może
oznaczać, że metoda robi więcej niż jedną rzecz. To może nam również
pokazać, że nazwa metody nie odzwierciedla wprost celu metody. Przez to
trudniej jest jej użyć ponownie w innym miejscu, jak również
przetestować.

Zupełnym przeciwieństwem jest kilku, bądź kilkunastolinijkowa metoda,
która ma wyłącznie jedną odpowiedzialność - robi jedną rzecz, robi ją
dobrze, a dodatkowo jest adekwatnie nazwana. Dobra nazwa metody może
znacznie ułatwić zrozumienie jej celu w trakcie czytania kodu. Jeśli
wiemy, co dokładnie jest celem metody, to chętniej jej użyjemy w innym
miejscu. Ponadto krótki kod, który zmienia stan wyłącznie jednego
obiektu i nie posiada efektów ubocznych, jest łatwo testowalny.

\subsection[nie-używaj-słowa-kluczowego-else]{2. Nie używaj słowa
kluczowego \type{else}}

Konstrukcja \type{if/else} jest obecna w prawie każdym języku
programowania i łatwo ją zrozumie nawet laik. Wielu z nas zna przypadek
wielokrotnie powtarzanych ciągów \type{if … elif … elif … else}. Trudno
się je czyta, trudno zrozumieć, częściej trudno zmodyfikować - a pokusa
dopisania kolejnego \type{else} jest ogromna. Tego typu sytuacje często
prowadzą do duplikowania kodu oraz wielu błędów.

Kilkoma rozwiązaniami pozwalającymi unikać takich konstrukcji są
polimorfizm, wzorzec strategii i wzorzec stanu. Kod, który korzysta z
takiego podejścia, znacznie łatwiej się czyta i jest łatwiejszy w
utrzymaniu.

W najprostszym przypadku wystarczy, że sprawdzamy warunek w \type{if} i
wychodzimy jeśli nie został on spełniony. Dzięki temu nie mamy w kodzie
kilku możliwych rozgałęzień. Reguła ta jest prostsza do przestrzegania,
jeśli każda metoda robi tylko jedną rzecz.

\subsection[prymitywne-typy-powinny-być-opakowane-w-klasy.]{3.
Prymitywne typy powinny być opakowane w klasy.}

Powyższa zasada obowiązuje wyłącznie w sytuacji, gdy zmienna przekazuje
jakieś zachowanie. Do czasu Pythona 3.5 nie było możliwości aby wymusić
by metoda na poziomie składni przyjmowała konkretny typ zmiennej. Nawet,
jeśli mamy dostępne tego typu udogodnienie, powyższa zasada ma sens.

Obowiązek poinformowania o intencjach metody spoczywa niejako na nazwie
metody. Jeśli zamiast integer przekażemy do metody obiekt klasy
\type{Hour} lub \type{Year} już sama deklaracja metody jest
czytelniejsza. Nie przekażemy przez pomyłkę obiektu \type{Year}, gdy
metoda spodziewa się od nas \type{Hour} lub odwrotnie. Co jest możliwe
gdy wymusimy wyłącznie typ \type{int}.

Tym samym dajemy programiście proste i czytelne API, które może nie
wymagać nawet wgłębiania się w kod implementacji. Co więcej, jeśli już
mamy tego typu klasy opakowujące typy proste, to jest to naturalne
miejsce by dodać kolejne metody operujące na tych typach.

\subsection[tylko-jedna-kropka-na-linię.]{4. Tylko jedna kropka na
linię.}

(Nie dotyczy getterów i tzw. \quotation{Fluent interface}.) Celem tej
zasady jest powstrzymanie programisty przed sięganiem zbyt głęboko w
implementację klas i tym samym łamaniem enkapsulacji. Jeśli w ramach
jednej linii odwołujemy się do dwóch różnych obiektów, to znaczy że
wiemy zbyt wiele o innych obiektach.

Prawo Demeter{[}4{]} mówi nam, że metoda może odwołać się wyłącznie do
metod należących do: * tego samego obiektu, * dowolnego parametru
przekazanego do niej, * dowolnego obiektu przez nią stworzonego, *
dowolnego składnika klasy do której należy dana metoda.

Pomóc w przestrzeganiu Prawa Demeter może zasada \quotation{tell, don't
ask}. W skrócie polega to na bezpośrednim oddelegowaniu pewnego zadania
do innego obiektu (bez martwienia się o implementację) zamiast
pobierania danych z kilku powiązanych obiektów.

\subsection[nie-skracaj-nazw.]{5. Nie skracaj nazw.}

Chodzi tutaj o wszelkie nazwy klas, metod i zmiennych. Najczęściej
skracamy nazwy dlatego, że pełna wersja jest zbyt długa, a musimy jej
użyć wiele razy. Zbyt długa nazwa może sugerować, że funkcja ma zbyt
wiele odpowiedzialności i stara się zrobić więcej niż jedną rzecz.
Wielokrotne używanie tej samej funkcji może z kolei być znakiem, że
warto pokusić się o usunięcie zduplikowanego kodu.

Warto zadbać o to, by nazwy klas i metod były krótkie i precyzyjne.
Jeśli funkcja ma jeden konkretny cel, to nazwanie jej w zwięzły sposób
nie będzie problemem. Należy również zwracać uwagę na kontekst w jakim
metoda będzie używana. Załóżmy że mamy klasę Order obsługującą
zamówienia. Funkcja służąca do wysyłki nie powinna się nazywać
\type{ship_order()}, gdyż powtarzamy w ten sposób nazwy. Zamiast tego
wystarczy samo ship(), dzięki czemu wywołanie order.ship() jest krótsze
i czytelniejsze.

\subsection[encje-powinny-być-małe.]{6. Encje powinny być małe.}

W oryginalnych założeniach dla Javy była mowa o 15-20 liniach na metodę,
50 liniach w ramach klasy i do 10 klas w module. W przypadku Pythona
możemy te limity pozostawić na niezmienionym poziomie.

Limity te mają na celu wymuszenie trzymania się zasady jednej
odpowiedzialności. Jeśli w ramach metody/klasy chcemy zrobić zbyt wiele,
jest to jasny znak, że należy pomyśleć o osobnej metodzie/klasie do tego
celu. Dzięki temu, klasy tworzą zwartą i zamkniętą całość.

W miarę jak klasy robią się mniejsze i mają mniejszą odpowiedzialność,
okaże się, że wszystkie klasy w ramach modułu są ze sobą logicznie
powiązane i służą osiągnięciu wspólnego celu. Moduły, podobnie jak
klasy, powinny mieć konkretny cel.

\subsection[ograniczona-liczba-atrybutów-w-klasie.]{7. Ograniczona
liczba atrybutów w klasie.}

Jeff Bay wspomina o limicie 2 atrybutów, taką też wartość możemy przyjąć
dla Pythona. Większość klas powinna być odpowiedzialna za stan jednej
zmiennej, ale w pewnych przypadkach potrzeba ich więcej. Dodanie każdej
kolejnej zmiennej (atrybutu) do klasy sprawia, że staje się ona mniej
spójna. Zdaniem Bay'a jeśli chcemy mieć więcej atrybutów niż 2, to
niemal na pewno jest sposób by część z nich wydzielić do osobnej klasy.
Przykład poniżej:

\startitemize[packed]
\item
  Customer:
\item
  Name
  \startitemize[packed]
  \item
    FirstName - string
  \item
    LastName - string
  \stopitemize
\item
  CustomerId - int
\stopitemize

\subsection[używaj-kolekcji-pierwszej-klasy.]{8. Używaj kolekcji
pierwszej klasy.}

Przez kolekcję pierwszej klasy rozumiemy, nie posiadające dodatkowych
atrybutów, klasy operujące na kolekcjach. Jeśli mamy zbiór elementów i
chcemy na nich wykonywać operację, to musimy stworzyć do tego dedykowaną
klasę. Każdy typ kolekcji jest opakowany w swoją klasę. Dzięki temu
zachowanie powiązane z tą kolekcją jest w jednym miejscu.

\subsection[nie-używaj-akcesorów.]{9. Nie używaj akcesorów.}

Nie chodzi tutaj bynajmniej o całkowity zakaz używania setterów i
getterów. Celem tej reguły jest to, by nie pobierać wartości z obiektu i
na tej podstawie podejmować jakichś decyzji. Prawidłowym podejściem jest
pozwolić klasie na wykonanie tej operacji. To wewnątrz klasy powinna się
znajdować logika która opiera się wprost na atrybucie tej klasy. Wraca
tutaj po raz kolejny zasada \quotation{tell, don't ask}. Stosując tę
regułę zachowujemy również zasadę otwarte zamknięte (Open/closed
principle z SOLID).

~

Wszystkie 9 zasad, choć czasem bardzo restrykcyjnych, wymusza na
programiście enkapsulację i myślenie zorientowane na programowanie
obiektowe. Aby lepiej opanować przedstawione zasady sugeruje się proste
ćwiczenie: należy napisać od postaw prostą aplikację do 1000 linii kodu
bez łamania tych zasad. Po zrealizowaniu takiego zadania od decyzji
dewelopera zależeć będzie których zasad się trzymać, a które można nieco
rozluźnić lub całkiem pominąć.

Kalistenika obiektowa to nie jest zbiór najlepszych praktyk i nie należy
przestrzegać ich rygorystycznie. W każdym przypadku można podać
przykład, gdzie stosowanie tych zasad sprawi, że końcowy kod będzie dużo
gorszej jakości. Najważniejsza jest bowiem praktyka i zdrowy rozsądek.

\subsection[bibliografia]{Bibliografia}

\startitemize[n,packed][stopper=.]
\item
  https://pl.wikipedia.org/wiki/Kalistenika
\item
  https://pragprog.com/book/twa/thoughtworks-anthology
\item
  https://www.goodreads.com/book/show/3735293-clean-code
\item
  https://pl.wikipedia.org/wiki/Prawo\type{_}Demeter
\stopitemize


\stoptext
