\usemodule[pycon-2016]
\starttext

\Title{Co mają wspólnego trzęsienia ziemi, cyklon tropikalny, lawiny błotne i nasze repozytoria?}
\HeaderTitle{Co mają wspólnego trzęsienia ziemi, [...] i nasze repozytoria?}
\Author{Paweł Kopka}
\MakeTitlePage

Co łączy repozytoria, trzęsienia ziemi, lawiny błotne oraz cyklony
tropikalne? Pewnie większości z was przychodzi do głowy słowo
\quotation{katastrofa}, ale dzisiaj nie o tym. Odpowiedź na otwierające
pytanie to: rozkłady odwrotnie potęgowe. Nie są tak
\quotation{popularne} jak rozkład normalny, ale też wypełnianiają
znaczącą przestrzeń w naszym świecie.

\subsection[rozkład-odwrotnie-potęgowy]{Rozkład odwrotnie potęgowy}

Rozkłady odwrotnie potęgowe, czasem znane jako rozkłady Pareto lub
Riemann Zeta, można opisać dość prostym wzorem $f(x) = c/x^a$.
Oczywiście charakteryzują się one pewnymi założeniami (takimi jak x nie
może równać się zero), ale nie warto się w nie zagłębiać. W wielkim
skrócie ten rozkład mówi, że małe zdarzenia zdarzają się znacznie
częściej niż wielkie. Świetnym przykładem jest badanie Pareto, który
interesował się zamożnością ludzi. Opisywał rozkład dóbr wśród
społeczeństwa, a, jak można się domyślać, dużo więcej ludzi posiada mały
dobytek, natomiast maleńka część społeczeństwa posiada dużo dłuższe
ciągi zer na kontach. Jako ciekawostkę można dodać, że z tych badań
powstała zasada Pareto, która mówi, że 20\letterpercent{} populacji
posiada 80\letterpercent{} bogactwa. To nam mówi jaka jest skala różnicy
między ilością dóbr na początku rozkładu, a w jego ogonie.

\subsection[przykłady]{Przykłady}

Nasza ulubiona Wikipedia prezentuje wiele przykładów występowania
rozkładów odwrotnie potęgowych z wielu dziedzin. Można zacząć od słowa,
a dokładnie od częstotliwości jego występowania w długich tekstach.
Jeśli policzymy ile razy pojawiają się poszczególne słowa i wyznaczymy
histogram, to zobaczymy, że jego kształt będzie przypominał nasze
rozkłady odwrotnie potęgowe. Przykładami z astronomii oraz geologii są
meteoryty oraz ziarenka piasku. W świecie finansów jest to wielkość
odszkodowań ubezpieczyciela za wypadki komunikacyjne. Kto wie, może z
użyciem tych rozkładów liczone są nasze składki OC i AC. Jako przykład
bliższy tematyce PyCona możemy podać wielkość plików przesyłanych
protokołem TCP. W tytułowym pytaniu wyliczone zostały przykłady z
dziedziny geofizyki, która obfituje w rozkłady odwrotnie potęgowe.
Przykłady rozkładów odwrotnie potęgowych stosowanych w geofizyce
przedstawiono w {[}1{]}. Cyklony tropikalne, a raczej ich energia, jak
wszystkie wcześniej wymienione zjawiska, charakteryzuje się tym, że duże
zdarzają się bardzo rzadko. Myślę, że mieszkańcy Hawajów bardzo cieszą
się z tego powodu. Pewnie to również wpływa na niewielką migrację ludzi
z miejsc zagrożonych. Podobnie jest z lawinami błotnymi. Jednak
najlepszym przykładem rozkładów odwrotnie potęgowych są trzęsienia
ziemi. Przesuwające się płyty tektoniczne podczas zmian w naprężeniach
dość często uwalniają energię. Niemniej jednak są to zjawiska
nieodczuwalne dla człowieka ze względu na ich wielkość. Jednak czasem
głęboko pod ziemią występują zdarzenia o ogromnej energii powodujące
fale sejsmiczne, które wywołują trzęsienia ziemi oraz inne katastrofy,
takie jak tsunami. Pewnie każdy słyszał, jak wielkie zniszczenia niosą
ze sobą takie zjawiska. Może dlatego też wielu naukowców próbuje
zrozumieć ich działanie lub też wyliczyć prawdopodobieństwo ich
wystąpienia. Rozkładem odwrotnie potęgowym możemy opisać stosunek
energii wydarzenia sejsmicznego do częstości jego występowania, który
wiąże się z bardzo dobrze znanym prawem Gutenberga-Richtera, co widać na
wzorze oraz wykresach. W 2002 roku udało się dość dobrze wyznaczyć
wykładnik dla trzęsień ziemi, którego wartość wynosi blisko a=1,63. Co
ciekawe, dla tego zjawiska jest też drugi przykład rozkładu odwrotnie
potęgowego, czyli czasy między trzęsieniami ziemi. Jest to związane z
prawem Omoriego, które mówi, że trzęsienia ziemi potrafią wywoływać
kolejne trzęsienia ziemi, tak zwane \quotation{aftershock}.

\subsection[a-co-z-tymi-repozytoriami]{A co z tymi repozytoriami?}

Od razu warto odnieść się do świetnego mówcy Gary'ego Bernhardta, który
w swoim wystąpieniu {[}2{]} pokazał rozkład potęgowy w dziedzinie
programowania, co jednocześnie przyczyniło się do powstania tego tekstu
i prezentacji na konferencji PyCon PL. Jako że wszystko w Pythonie jest
obiektem, to spójrzmy na klasy oraz ich najczęściej występujące
elementy, takie jak metody i atrybuty. Tym sposobem mamy już dwa
rozkłady potęgowe, oczywiście, jeśli weźmiemy dość duże repozytoria, np.
Django, Sphinx. Pewnie ciężko to sobie wyobrazić na pierwszy rzut oka.
Teraz zajrzyjcie w~głąb swojej duszy\ldots{} albo po prostu spójrzcie na
swoje projekty; jak często tworzycie obiekty z większą liczbą metod niż
10, 20 albo 30? Prawdopodobnie niezbyt często, głównie dlatego, że
jednak lepszą praktyką jest rozbijanie na mniejsze komponenty dla
zachowania czytelności kodu. Dlatego też liczba klas z kilkoma metodami
jest znacznie większa. Jak widać, dobre praktyki pisania kodu spychają
nas w stronę rozkładów odwrotnie potęgowych {[}3{]}. Warto zauważyć, że
tym razem nie jest to żadna katastrofa.

\subsection[może-troszkę-kodu]{Może troszkę kodu}

Najlepsze przykłady to te, które mamy przed oczami. I właśnie tutaj mamy
taki przypadek, możemy policzyć liczbę wystąpień poszczególnych słów, a
później wyznaczyć histogram dla tych zliczeń. Poniżej znajduje się kilka
linii kodu, które robią to z wykorzystaniem uwielbianej przez naukowców
biblioteki matplotlib. Aby było bardziej realistycznie, statystyki nie
są całkiem dokładne.

\starttyping
import matplotlib.pyplot as plt

file_name = 'opis.md'
with open(file_name, 'r') as f:
    data = f.read()
data = filter(lambda c: c.isalpha() or c.isspace(), data)
list_words = data.lower().split(' ')
count_words = {}
for word in list_words:
    if word not in count_words:
        count_words[word] = 0
        count_words[word] += 1
plt.hist(count_words.values(), 100)
plt.show()
\stoptyping

Sporą dawkę wiedzy oraz specjalistów od rozkładów odwrotnie potęgowych
można znaleźć w Zakładzie Geofizyki Teoretycznej (Instytut Geofizyki
Polskiej Akademii Nauk) {[}4{]}. A po konferencji PyCon PL pojawi się
projekt liczący rozkłady dla repozytoriów projektów pisanych w języku
Python {[}5{]}.

\subsection[bibliografia]{Bibliografia}

\startitemize[n,packed][stopper=.]
\item
  Anna Deluca, Álvaro Corral. Fitting and goodness-of-fit test of
  non-truncated and truncated power-law distributions. Acta Geophysica.
  61(6), (2013)
\item
  Gary Bernhardt. The Unix Chainsaw. Cascadia Ruby Conference, Seattle,
  WA, USA, 29-30 czerwca 2011
\item
  Richard Wheeldon, Steve Counsell. Power Law Distributions in Class
  Relationships.
\item
  \hyphenatedurl{http://www.igf.edu.pl/geofizyki-teoretycznej.php}
\item
  \hyphenatedurl{https://github.com/pawelkopka}
\stopitemize


\stoptext
