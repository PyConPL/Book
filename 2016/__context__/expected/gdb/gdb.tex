\usemodule[pycon-2016]
\starttext

\Title{Debugging of CPython processes with gdb}
\Author{Roman Podoliaka}
\MakeTitlePage

pdb{[}1{]} has been, is and probably always will be the bread and butter
of Python programmers, when they need to find the root cause of a
problem in their applications, as it's a built-in and easy to use
debugger. But there are cases, when \type{pdb} can't help you, e.g.~if
your app has got stuck somewhere, and you need to attach to a running
process to find out why, without restarting it. This is where gdb{[}2{]}
shines.

\subsection[why-gdb]{Why gdb?}

\type{gdb} is a general purpose debugger, that is mostly used for
debugging of C and C++ applications (although it actually supports Ada,
Objective-C, Pascal and more).

There are different reasons why a Python programmer would be interested
in \type{gdb} for debugging:

\startitemize
\item
  \type{gdb} allows one to attach to a running process without starting
  an app in debug mode or modifying the app code in some way first
  (e.g.~putting something like \type{import rpdb; rpdb.set_trace()} into
  the code);
\item
  \type{gdb} allows one to take a core dump{[}3{]} of a process and
  analyze it later. This is useful, when you don't want to stop the
  process for the duration of time, while you are introspecting its
  state, as well as when you do post-mortem{[}4{]} debugging of a
  process that has already failed (e.g.~crashed{[}5{]} with a
  segmentation fault);
\item
  most debuggers available for Python (notable exceptions are
  winpdb{[}6{]} and pydevd{[}7{]}) do not support switching between
  threads of the application being debugged. \type{gdb} allows that, as
  well as debugging of threads created by non-Python code (e.g.~in some
  native library used).
\stopitemize

\subsection[debugging-of-interpreted-languages]{Debugging of interpreted
languages}

So what makes Python special when using \type{gdb}?

In contradistinction to programming languages like C or C++, Python code
is not compiled into a native binary for a target platform. Instead
there is an interpreter (e.g. CPython{[}8{]}, the reference
implementation of Python), which executes compiled byte-code{[}9{]}.

This effectively means, that when you attach to a Python process with
\type{gdb}, you'll debug the interpreter instance and introspect the
process state at the interpreter level, not the application level:
i.e.~you will see functions and variables of the interpreter, not of
your app.

To give you an example, let's take a look at a \type{gdb} backtrace of a
CPython (the most popular Python interpreter) process:

\starttyping

#0  0x00007fcce9b2faf3 in __epoll_wait_nocancel () at ../sysdeps/unix/
    syscall-template.S:81
#1  0x0000000000435ef8 in pyepoll_poll (self=0x7fccdf54f240,
    args=<optimized out>, kwds=<optimized out>)
    at ../Modules/selectmodule.c:1034
#2  0x000000000049968d in call_function (oparg=<optimized out>,
    pp_stack=0x7ffc20d7bfb0) at ../Python/ceval.c:4020
#3  PyEval_EvalFrameEx () at ../Python/ceval.c:2666
#4  0x0000000000499ef2 in fast_function () at ../Python/ceval.c:4106
#5  call_function () at ../Python/ceval.c:4041
#6  PyEval_EvalFrameEx () at ../Python/ceval.c:2666
\stoptyping

and one obtained by the means of \type{traceback.extract_stack()}:

\starttyping

/usr/local/lib/python2.7/dist-packages/eventlet/greenpool.py:82
in _spawn_n_impl
    `func(*args, **kwargs)`

/opt/stack/neutron/neutron/agent/l3/agent.py:461
in _process_router_update
    `for rp, update in self._queue.each_update_to_next_router():`

/opt/stack/neutron/neutron/agent/l3/router_processing_queue.py:154
in each_update_to_next_router
    `next_update = self._queue.get()`

/usr/local/lib/python2.7/dist-packages/eventlet/queue.py:313 in get
    `return waiter.wait()`

/usr/local/lib/python2.7/dist-packages/eventlet/queue.py:141 in wait
   `return get_hub().switch()`

/usr/local/lib/python2.7/dist-packages/eventlet/hubs/hub.py:294 in switch
    `return self.greenlet.switch()`
\stoptyping

As is, the former is of little help, when you are trying to find a
problem in your Python code, and all you see is the current state of the
interpreter itself.

However, \type{PyEval_EvalFrameEx}{[}10{]} looks interesting: it's a
function of CPython, which executes bytecode of Python application level
functions and, thus, has access to their state - the very state we are
usually interested in.

\subsection[gdb-and-python]{gdb and Python}

Search results for \type{"gdb debug python"} can be confusing. The thing
is, that starting from \type{gdb} version 7 it's been possible to
extend{[}11{]} the debugger with Python code, e.g.~in order to provide
visualisations for C++ STL{[}12{]} types, which is much easier to
implement in Python rather than in the built-in macro{[}13{]} language.

In order to be able to debug CPython processes and introspect the
application level state, the interpreter developers decided to extend
\type{gdb} and wrote a script{[}14{]} for that in\ldots{} Python, of
course!

So it's two different, but related things:

\startitemize[packed]
\item
  \type{gdb} versions 7+ are extendable with Python modules
\item
  there's a Python \type{gdb} extension for debugging of CPython
  processes
\stopitemize

\subsection[debugging-python-with-gdb-101]{Debugging Python with gdb
101}

First of all, you need to install \type{gdb}:

\starttyping

# apt-get install gdb
\stoptyping

or

\starttyping

# yum install gdb
\stoptyping

depending on the Linux distro you are using.

The next step is to install debugging symbols{[}15{]} for the CPython
build you have:

\starttyping

# apt-get install python-dbg
\stoptyping

or

\starttyping

# yum install python-debuginfo
\stoptyping

Some Linux distros like CentOS or RHEL ship debugging symbols
separately{[}16{]} from all other packages and recommend to install
those like:

\starttyping

# debuginfo-install python
\stoptyping

The installed debugging symbols will be used by the CPython
script{[}14{]} for \type{gdb} in order to analyze the
\type{PyEval_EvalFrameEx} frames (a frame essentially is a function call
and the associated state in a form of local variables and CPU registers,
etc) and map those to application level functions in your code.

Without debugging symbols it's much harder to do - \type{gdb} allows you
to manipulate the process memory in any way you want, but you can't
easily understand what data structures reside in what memory areas.

After all preparatory steps have been completed, you can give \type{gdb}
a try. E.g. in order to attach to a running CPython process, do:

\starttyping

gdb /usr/bin/python -p $PID
\stoptyping

At this point you can get an application level backtrace for the current
thread (note that some frames are \quotation{missing} - this is
expected, as \type{gdb} counts all the interpreter level frames and only
some of those are calls in application level code -
\type{PyEval_EvalFrameEx} ones):

\starttyping

(gdb) py-bt

#4 Frame 0x1b7da60, for file /usr/lib/python2.7/sched.py, line 111,
in run (self=<scheduler(timefunc=<built-in function time>,
delayfunc=<built-in function sleep>,
_queue=[<Event at remote 0x7fe1f8c74a10>]) at remote 0x7fe1fa086758>,
q=[...], delayfunc=<built-in function sleep>,
timefunc=<built-in function time>, pop=<built-in function heappop>,
time=<float at remote 0x1a0a400>, priority=1,
action=<function at remote 0x7fe1fa083aa0>, argument=(171657,),
checked_event=<...>, now=<float at remote 0x1b8ec58>)
    delayfunc(time - now)
#7 Frame 0x1b87e90, for file /usr/bin/dstat, line 2416,
in main (interval=1, user='ubuntu', hostname='rpodolyaka-devstack',
key='unit_hi', linewidth=150, plugin='page', mods=('page', 'page24'),
mod='page', pluginfile='dstat_page', scheduler=<scheduler(timefunc=
<built-in function time>, delayfunc=<built-in function sleep>,
_queue=[<Event at remote 0x7fe1f8c74a10>]) at remote 0x7fe1fa086758>)
    scheduler.run()
#11 Frame 0x7fe1fa0bc5c0, for file /usr/bin/dstat, line 2554,
in <module> ()
    main()
\stoptyping

or find out what exact line of the application code is currently being
executed:

\starttyping

(gdb) py-list

 106            pop = heapq.heappop
 107            while q:
 108                time, priority, action, argument = checked_event =
                        q[0]
 109                now = timefunc()
 110                if now < time:
>111                    delayfunc(time - now)
 112                else:
 113                    event = pop(q)
 114                    # Verify that the event was not removed/altered
 115                    # by another thread after we last looked at q[0].
 116                    if event is checked_event:
\stoptyping

or look at values of local variables:

\starttyping

(gdb) py-locals

self = <scheduler(timefunc=<built-in function time>,
delayfunc=<built-in function sleep>,
_queue=[<Event at remote 0x7fe1f8c74a10>]) at remote 0x7fe1fa086758>
q = [<Event at remote 0x7fe1f8c74a10>]
delayfunc = <built-in function sleep>
timefunc = <built-in function time>
pop = <built-in function heappop>
time = <float at remote 0x1a0a400>
priority = 1
action = <function at remote 0x7fe1fa083aa0>
argument = (171657,)
checked_event = <Event at remote 0x7fe1f8c74a10>
now = <float at remote 0x1b8ec58>
\stoptyping

There are more \type{py-} commands provided by the CPython
script{[}14{]} for \type{gdb}. Check out the debugging guide{[}17{]} for
details.

\subsection[gotchas]{Gotchas}

Although the described technique should work out-of-the-box, there are a
few known gotchas.

\subsection[python-dbg]{python-dbg}

The \type{python-dbg} package in Debian and Ubuntu will not only install
the debugging symbols for \type{python} (which are stripped at the
package build time to save disk space), but also provide an additional
CPython binary \type{python-dbg}.

The latter essentially is a separate build of CPython (with
\type{--with-debug} flag passed to \type{./configure}) with many
run-time checks. Generally, you don't want to use \type{python-dbg} in
production, as it can be (much) slower than \type{python}, e.g.:

\starttyping

$ time python -c "print(sum(range(1, 1000000)))"
499999500000

real    0m0.096s
user    0m0.057s
sys 0m0.030s

$ time python-dbg -c "print(sum(range(1, 1000000)))"
499999500000
[18318 refs]

real    0m0.237s
user    0m0.197s
sys 0m0.016s
\stoptyping

The good thing is, that you don't need to: it's still possible to debug
\type{python} executable by the means of \type{gdb}, as long as the
corresponding debugging symbols are installed. So \type{python-dbg} just
adds a bit more confusion to the CPython/gdb story - you can safely
ignore its existence.

\subsection[build-flags]{Build flags}

Some Linux distros build CPython passing the \type{-g0} or \type{-g1}
option{[}18{]} to \type{gcc}: the former produces a binary without
debugging information at all, and the latter does not allow \type{gdb}
to get information about local variables at runtime.

Both these options break the described workflow of debugging CPython
processes by the means of \type{gdb}. The solution is to rebuild CPython
with \type{-g} or \type{-g2} (\type{2} is the default value when
\type{-g} is passed).

Fortunately, all current versions of the major Linux distros (Ubuntu
Trusty/Xenial, Debian Jessie, CentOS/RHEL 7) ship the
\quotation{correctly} built CPython.

\subsection[optimized-out-frames]{Optimized out frames}

For introspection to work properly, it's crucial, that information
about\crlf
\type{PyEval_EvalFrameEx} arguments is preserved for each call.
Depending on the optimization level{[}19{]} used in \type{gcc} when
building CPython or the concrete compiler version used, it's possible
that this information will be lost at runtime (especially with
aggressive optimizations enabled by \type{-O3}). In this case \type{gdb}
will show you something like:

\starttyping

(gdb) bt

#0  0x00007fdf3ca31be3 in __select_nocancel ()
    at ../sysdeps/unix/syscall-template.S:84
#1  0x00000000005d1da4 in pysleep (secs=<optimized out>)
    at ../Modules/timemodule.c:1408
#2  time_sleep () at ../Modules/timemodule.c:231
#3  0x00000000004f5465 in call_function (oparg=<optimized out>,
    pp_stack=0x7fff62b184c0) at ../Python/ceval.c:4637
#4  PyEval_EvalFrameEx () at ../Python/ceval.c:3185
#5  0x00000000004f5194 in fast_function (nk=<optimized out>,
    na=<optimized out>, n=<optimized out>, pp_stack=0x7fff62b185c0,
    func=<optimized out>) at ../Python/ceval.c:4750
#6  call_function (oparg=<optimized out>, pp_stack=0x7fff62b185c0)
    at ../Python/ceval.c:4677
#7  PyEval_EvalFrameEx () at ../Python/ceval.c:3185
#8  0x00000000004f5194 in fast_function (nk=<optimized out>,
    na=<optimized out>, n=<optimized out>, pp_stack=0x7fff62b186c0,
    func=<optimized out>) at ../Python/ceval.c:4750
#9  call_function (oparg=<optimized out>, pp_stack=0x7fff62b186c0)
    at ../Python/ceval.c:4677
#10 PyEval_EvalFrameEx () at ../Python/ceval.c:3185
#11 0x00000000005c5da8 in _PyEval_EvalCodeWithName.lto_priv.1326 ()
    at ../Python/ceval.c:3965
#12 0x00000000005e9d7f in PyEval_EvalCodeEx () at ../Python/ceval.c:3986
#13 PyEval_EvalCode (co=<optimized out>, globals=<optimized out>,
    locals=<optimized out>) at ../Python/ceval.c:777
#14 0x00000000005fe3d2 in run_mod () at ../Python/pythonrun.c:970
#15 0x000000000060057a in PyRun_FileExFlags ()
    at ../Python/pythonrun.c:923
#16 0x000000000060075c in PyRun_SimpleFileExFlags ()
    at ../Python/pythonrun.c:396
#17 0x000000000062b870 in run_file (p_cf=0x7fff62b18920, filename=
    0x1733260 L"test2.py", fp=0x1790190) at ../Modules/main.c:318
#18 Py_Main () at ../Modules/main.c:768
#19 0x00000000004cb8ef in main () at ../Programs/python.c:69
#20 0x00007fdf3c970610 in __libc_start_main (main=0x4cb810 <main>, argc=2,
    argv=0x7fff62b18b38, init=<optimized out>, fini=<optimized out>,
    rtld_fini=<optimized out>, stack_end=0x7fff62b18b28)
    at libc-start.c:291
#21 0x00000000005c9df9 in _start ()

(gdb) py-bt
Traceback (most recent call first):
  File "test2.py", line 9, in g
    time.sleep(1000)
  File "test2.py", line 5, in f
    g()
  (frame information optimized out)
\stoptyping

i.e.~some application level frames will be available, some will not.
There is little you can do at this point, except for rebuilding CPython
with a lower optimization level, but that often is not an option for
production (not to mention the fact you'll be using a custom CPython
build, not the one provided by your Linux distro).

\subsection[virtual-environments-and-custom-cpython-builds]{Virtual
environments and custom CPython builds}

When a virtual environment is used, it may appear that the extension
does not work:

\starttyping

$ gdb -p 2975

GNU gdb (Debian 7.10-1+b1) 7.10
[...]
Attaching to process 2975
Reading symbols from /home/rpodolyaka/workspace/venvs/default/bin/
python2...(no debugging symbols found)...done.

(gdb) bt

#0  0x00007ff2df3d0be3 in __select_nocancel ()
    at ../sysdeps/unix/syscall-template.S:84
#1  0x0000000000588c4a in ?? ()
#2  0x00000000004bad9a in PyEval_EvalFrameEx ()
#3  0x00000000004bfd1f in PyEval_EvalFrameEx ()
#4  0x00000000004bfd1f in PyEval_EvalFrameEx ()
#5  0x00000000004b8556 in PyEval_EvalCodeEx ()
#6  0x00000000004e91ef in ?? ()
#7  0x00000000004e3d92 in PyRun_FileExFlags ()
#8  0x00000000004e2646 in PyRun_SimpleFileExFlags ()
#9  0x0000000000491c23 in Py_Main ()
#10 0x00007ff2df30f610 in __libc_start_main (main=0x491670 <main>,
    argc=2, argv=0x7ffc36f11cf8, init=<optimized out>,
    fini=<optimized out>, rtld_fini=<optimized out>,
    stack_end=0x7ffc36f11ce8) at libc-start.c:291
#11 0x000000000049159b in _start ()

(gdb) py-bt

Undefined command: "py-bt".  Try "help".
\stoptyping

\type{gdb} can still follow the CPython frames, but information on
\type{PyEval_EvalCodeEx} calls is not available.

If you scroll up the \type{gdb} output a bit, you'll see that \type{gdb}
failed to find the debugging symbols for \type{python} executable:

\starttyping

$ gdb -p 2975

GNU gdb (Debian 7.10-1+b1) 7.10
[...]
Attaching to process 2975
Reading symbols from /home/rpodolyaka/workspace/venvs/default/bin/
python2...(no debugging symbols found)...done.
\stoptyping

How is a virtual environment any different? Why did not \type{gdb} find
the debugging symbols?

First and foremost, the path to \type{python} executable is different.
Note, that I did not specify the executable file, when attaching to the
process. In this case \type{gdb} will take the executable file of the
process (i.e. \type{/proc/$PID/exe} value on Linux).

One of the ways to separate{[}20{]} debugging symbols is to put those
into a well-known directory (default is \type{/usr/lib/debug/}, although
it's configurable via \type{debug-file-directory} option in \type{gdb}).
In our case \type{gdb} tried to load debugging symbols from
\type{/usr/lib/debug/home/rpodolyaka/workspace/venvs/default/bin/}\crlf
\type{python2} and, obviously, did not find anything there.

The solution is simple - specify the executable under debug explicitly
when running \type{gdb}:

\starttyping

$ gdb /usr/bin/python2.7 -p $PID
\stoptyping

Thus, \type{gdb} will look for debugging symbols in the
\quotation{right} place -\crlf
\type{/usr/lib/debug/usr/bin/python2.7}.

It's also worth mentioning, that it's possible that debugging symbols
for a particular executable are identified by a unique \type{build-id}
value stored in ELF{[}21{]} executable headers. E.g. CPython on my
Debian machine:

\starttyping

$ objdump -s -j .note.gnu.build-id /usr/bin/python2.7

/usr/bin/python2.7:     file format elf64-x86-64

Contents of section .note.gnu.build-id:
 400274 04000000 14000000 03000000 474e5500  ............GNU.
 400284 8d04a3ae 38521cb7 c7928e4a 7c8b1ed3  ....8R.....J|...
 400294 85e763e4
\stoptyping

In this case \type{gdb} will look for debugging symbols using the
\type{build-id} value:

\starttyping

$ gdb /usr/bin/python2.7

GNU gdb (Debian 7.10-1+b1) 7.10
[...]
Reading symbols from /usr/bin/python2.7...Reading symbols from /usr/lib/
debug/.build-id/8d/04a3ae38521cb7c7928e4a7c8b1ed385e763e4.debug...done.
done.
\stoptyping

This has a nice implication - it no longer matters how the executable is
called: \type{virtualenv} just creates a copy of the specified
interpreter executable, thus, both executables - the one in
\type{/usr/bin/} and the one in your virtual environment will use the
very same debugging symbols:

\starttyping

$ gdb -p 11150

GNU gdb (Debian 7.10-1+b1) 7.10
[...]
Attaching to process 11150
Reading symbols from /home/rpodolyaka/sandbox/testvenv/bin/
python2.7...Reading symbols from /usr/lib/debug/.build-id/8d/
04a3ae38521cb7c7928e4a7c8b1ed385e763e4.debug...done.

$ ls -la /proc/11150/exe
lrwxrwxrwx 1 rpodolyaka rpodolyaka 0 Apr 10 15:18 /proc/11150/exe ->
    /home/rpodolyaka/sandbox/testvenv/bin/python2.7
\stoptyping

The first problem is solved, \type{bt} output now looks much nicer, but
\type{py-bt} command is still undefined:

\starttyping

(gdb) bt

#0  0x00007f3e95083be3 in __select_nocancel ()
    at ../sysdeps/unix/syscall-template.S:84
#1  0x0000000000594a59 in floatsleep (secs=<optimized out>)
    at ../Modules/timemodule.c:948
#2  time_sleep.lto_priv () at ../Modules/timemodule.c:206
#3  0x00000000004c524a in call_function (oparg=<optimized out>,
    pp_stack=0x7ffefb5045b0) at ../Python/ceval.c:4350
#4  PyEval_EvalFrameEx () at ../Python/ceval.c:2987
#5  0x00000000004ca95f in fast_function (nk=<optimized out>,
    na=<optimized out>, n=<optimized out>, pp_stack=0x7ffefb504700,
    func=0x7f3e95f78c80) at ../Python/ceval.c:4435
#6  call_function (oparg=<optimized out>, pp_stack=0x7ffefb504700)
    at ../Python/ceval.c:4370
#7  PyEval_EvalFrameEx () at ../Python/ceval.c:2987
#8  0x00000000004ca95f in fast_function (nk=<optimized out>,
    na=<optimized out>, n=<optimized out>, pp_stack=0x7ffefb504850,
    func=0x7f3e95f78c08) at ../Python/ceval.c:4435
#9  call_function (oparg=<optimized out>, pp_stack=0x7ffefb504850)
    at ../Python/ceval.c:4370
#10 PyEval_EvalFrameEx () at ../Python/ceval.c:2987
#11 0x00000000004c32e5 in PyEval_EvalCodeEx () at ../Python/ceval.c:3582
#12 0x00000000004c3089 in PyEval_EvalCode (co=<optimized out>,
    globals=<optimized out>, locals=<optimized out>)
    at ../Python/ceval.c:669
#13 0x00000000004f263f in run_mod.lto_priv ()
    at ../Python/pythonrun.c:1376
#14 0x00000000004ecf52 in PyRun_FileExFlags ()
    at ../Python/pythonrun.c:1362
#15 0x00000000004eb6d1 in PyRun_SimpleFileExFlags ()
    at ../Python/pythonrun.c:948
#16 0x000000000049e2d8 in Py_Main () at ../Modules/main.c:640
#17 0x00007f3e94fc2610 in __libc_start_main (main=0x49dc00 <main>, argc=2,
    argv=0x7ffefb504c98, init=<optimized out>, fini=<optimized out>,
    rtld_fini=<optimized out>, stack_end=0x7ffefb504c88)
    at libc-start.c:291
#18 0x000000000049db29 in _start ()

(gdb) py-bt

Undefined command: "py-bt".  Try "help".
\stoptyping

Once again, this is caused by the fact that \type{python} binary in a
virtual environment has a different path. By default, \type{gdb} will
try to auto-load{[}22{]} Python extensions for a particular object file
under debug, if they exist. Specifically, \type{gdb} will look for
\type{objfile-gdb.py} and try to \type{source} it on start:

\starttyping

(gdb) info auto-load

gdb-scripts:  No auto-load scripts.
libthread-db:  No auto-loaded libthread-db.
local-gdbinit:  Local .gdbinit file was not found.
python-scripts:
Loaded  Script
Yes     /usr/share/gdb/auto-load/usr/bin/python2.7-gdb.py
\stoptyping

If, for some reason this has not been done, you can always do it
manually:

\starttyping

(gdb) source /usr/share/gdb/auto-load/usr/bin/python2.7-gdb.py
\stoptyping

e.g.~if you want to test a new version of the \type{gdb} extension
shipped with CPython.

\subsection[pypy-jython-etc]{PyPy, Jython, etc}

The described debugging technique is only feasible for the CPython
interpreter as is, as the \type{gdb} extension is specifically written
to introspect the state of CPython internals (e.g.
\type{PyEval_EvalFrameEx} calls).

For PyPy{[}23{]} there is an open issue{[}24{]} on Bitbucket, where it
was proposed to provide integration with \type{gdb}, but looks like the
attached patches have not been merged yet and the person who wrote those
lost interest in this.

For Jython{[}25{]} you could probably use standard tools for debugging
of \type{JVM} applications, e.g.~VisualVM{[}26{]}.

\subsection[conclusion]{Conclusion}

\type{gdb} is a powerful tool that allows one to debug complex problems
with crashing or hanging CPython processes, as well as Python code that
does calls to native libraries. On modern Linux distros debugging
CPython processes with \type{gdb} must be as simple as installation of
debugging symbols for the concrete interpreter build, although there are
a few known gotchas, especially when virtual environments are used.

\subsection[references]{References}

\startitemize[n,packed][stopper=.,width=2.0em]
\item
  https://docs.python.org/3.5/library/pdb.html
\item
  https://www.gnu.org/software/gdb/
\item
  https://en.wikipedia.org/wiki/Core\type{_}dump
\item
  https://en.wikipedia.org/wiki/Debugging\#Techniques
\item
  https://www.freedesktop.org/software/systemd/man/systemd-coredump.html
\item
  http://winpdb.org/
\item
  https://github.com/fabioz/PyDev.Debugger
\item
  https://en.wikipedia.org/wiki/CPython
\item
  http://security.coverity.com/blog/2014/Nov/\crlf
  understanding-python-bytecode.html
\item
  https://docs.python.org/2/c-api/veryhigh.html\#c.PyEval\type{_}EvalFrameEx
\item
  https://sourceware.org/gdb/current/onlinedocs/gdb/Python.html\#Python
\item
  https://sourceware.org/gdb/wiki/STLSupport
\item
  http://www.ibm.com/developerworks/aix/library/au-gdb.html
\item
  https://github.com/python/cpython/blob/master/Tools/gdb/libpython.py
\item
  http://www.tutorialspoint.com/gnu\type{_}debugger/gdb\type{_}debugging\type{_}symbols.htm
\item
  http://debuginfo.centos.org/
\item
  https://docs.python.org/devguide/gdb.html
\item
  https://gcc.gnu.org/onlinedocs/gcc/Debugging-Options.html
\item
  https://gcc.gnu.org/onlinedocs/gcc/Optimize-Options.html
\item
  https://sourceware.org/gdb/onlinedocs/gdb/Separate-Debug-Files.html
\item
  https://en.wikipedia.org/wiki/Executable\type{_}and\type{_}Linkable\type{_}Format
\item
  https://sourceware.org/gdb/onlinedocs/gdb/\crlf
  Python-Auto\type{_}002dloading.html\#set\letterpercent{}20auto\letterpercent{}2dload\letterpercent{}20python\letterpercent{}2dscripts
\item
  http://pypy.org/
\item
  https://bitbucket.org/pypy/pypy/issues/1204/gdb-hooks-for-debugging-pypy
\item
  http://www.jython.org/
\item
  http://visualvm.java.net/
\stopitemize


\stoptext
