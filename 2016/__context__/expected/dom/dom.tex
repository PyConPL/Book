\usemodule[pycon-2016]
\starttext

\Title{Inteligentny dom z (Micro)Python}
\Author{Krzysztof Czarnota}
\MakeTitlePage

Inteligentne domy stają się coraz bardziej powszechne. Nic dziwnego,
rozwiązania, które ułatwiają nam codzienne życie, a także pozwalają na
oszczędność czasu i~pieniędzy muszą cieszyć się dużą popularnością. Na
rynku pojawia się wiele firm, które oferują swoje usługi w zakresie
automatyzacji budynków, ale niejednokrotnie koszty takiej instalacji
odstraszają potencjalnego użytkownika. Tymczasem stworzenie systemu
inteligentnego domu wcale nie musi być drogie ani trudne.

\subsection[jak-zacząć]{Jak zacząć?}

Instalacja inteligentnego domu składa się z sieci czujników oraz
elementów wykonawczych, która to sieć monitoruje oraz kontroluje obecny
stan budynku. Druga część instalacji to centralny system zarządzania,
który pozwala na personalizację ustawień, a następnie, na podstawie
informacji z czujników, steruje urządzeniami wykonawczymi. W budynkach
(zwłaszcza w nowych inwestycjach) instalowanych jest obecnie wiele
urządzeń pozwalających na zdalne sterowanie. Jednak nie ma jeszcze
jednolitego standardu, który umożliwiałby bezproblemowe połączenie i
sterowanie wszystkimi urządzeniami. W naszej instalacji muszą pojawić
się także elementy pozwalające na sterowanie urządzeniami, które
oryginalnie nie zapewniają takiej funkcjonalności. Najprostszym
rozwiązaniem obu tych problemów jest wykonanie własnego projektu
sprzętowego.

Ale bez obaw, stworzenie urządzenia elektronicznego przestało wiązać się
z niskopoziomowym programowaniem mikrokontrolera i studiowaniem not
aplikacyjnych układów scalonych. Silnie rozwijający się rynek IoT
(Internet of Things) oraz platform prototypowych daje do dyspozycji cały
szereg modułów elektronicznych gotowych do wykorzystania bez wgłębiania
się w ich budowę, a nowoczesne narzędzia deweloperskie pozwalają nam na
napisanie własnego systemu wbudowanego w języku wysokiego poziomu.
Niemniej takie podejście sprawdzi się tylko na początku, dlatego
zachęcam do pogłębiania swojej wiedzy w dziedzinie elektroniki.
Interakcja ze sprzętem daje wiele radości, a uruchomione urządzenie
potrafi przynieść wymierne korzyści.

W celu rozpoczęcia pracy niezbędne jest zaopatrzenie się w kilka
urządzeń oraz modułów, które znajdą zastosowanie w każdej instalacji
inteligentnego budynku:

\startitemize[packed]
\item
  Raspberry Pi - centrum zarządzania budynkiem,
\item
  Moduły ESP-12F - bezprzewodowa komunikacja z czujnikami i urządzeniami
  wykonawczymi,
\item
  Konwerter USB-UART - komunikacja przez port szeregowy,
\item
  Moduł przekaźnika RM0 - umożliwia włączanie/wyłączanie urządzeń 230V,
\item
  Moduł DS18b20 - termometr 1-wire,
\item
  Zasilacz 5V oraz 3,3V.
\stopitemize

Już tak krótka lista pozwoli na sterowanie oświetleniem czy sprawowanie
kontroli nad temperaturą w domu. Na początku, do testowania
funkcjonalności nowych modułów elektronicznych, można wykorzystać jedną
z platform prototypowych np. Arduino.

\subsection[micropython-i-moduł-esp8266]{MicroPython i moduł ESP8266}

MicroPython to implementacja języka Python 3, która zawiera niewielki
podzbiór biblioteki standardowej języka Python i jest zoptymalizowana
pod kątem działania na mikrokontrolerach {[}1{]}. Istnieje kilka
platform sprzętowych pozwalających na uruchomienie środowiska
MicroPython takich jak pyBoard czy WiPy. Układem, na którym oprzemy
naszą instalację inteligentnego domu, będzie ESP8266 {[}2{]}, a
konkretnie moduł o nazwie ESP-12F. Co oferuje nam urządzenie wielkości
znaczka pocztowego (24×16mm) za niecałe 2 dolary? Oto podstawowe cechy
modułu ESP-12F:

\startitemize[packed]
\item
  32 bit RISC CPU - Tensilica Xtensa LX106 @80MHz,
\item
  11 wyprowadzeń GPIO - wejścia/wyjścia cyfrowe,
\item
  1 wejście ADC - 10 bitowy przetwornik analogowo-cyfrowy,
\item
  interfejsy - SPI, I2C, 2×UART,
\item
  WiFi 802.11 b/g/n (2,4GHz),
\item
  4MB flash.
\stopitemize

Komunikacja z modułem ESP-12F odbywa się za pomocą portu szeregowego.
Konwerter USB-UART należy podłączyć do wyprowadzeń oznaczonych jako TXD
oraz RXD. Trzeba zwrócić uwagę, że komponent zasilany jest napięciem
3,3V i pracuje na logice o tym samym potencjale. Do zasilania układu
należy wykorzystać źródło napięcia o wydajności około 300mA. W celu
uruchomienia modułu wymagane jest również podciągnięcie pinu EN
(\type{CH_PD}) do VCC oraz pinu GPIO15 do GND przez rezystory 10kΩ.

Oryginalne oprogramowanie modułu obsługuje komunikację za pomocą komend
AT. W celu uruchomienia środowiska MicroPython należy wgrać nowy
firmware. Najlepszym sposobem uzyskania oprogramowania jest zbudowanie
najnowszej wersji ze źródeł. Szczegółowy opis tego procesu znajduje się
w repozytorium z kodem źródłowym {[}3{]}, można również wykorzystać
znalezione w sieci gotowe pliki binarne firmware. Kolejnym krokiem jest
wyczyszczenie pamięci układu i wgranie nowego firmware. Do wykonania
tych czynności wykorzystamy narzędzie esptool, które służy do
komunikacji z bootloaderem w układzie ESP8266 {[}4{]}. Przed wpisaniem
podanej poniżej komendy należy pamiętać o uruchomieniu modułu ESP w
trybie programowania (flash mode), wykonuje się to przez podłączenie
pinu GPIO0 do masy.

\starttyping
$ esptool.py --port /dev/tty.xxx --baud 460800 write_flash --flash_size=8m 0 firmware-combined.bin
\stoptyping

W przypadku powodzenia naszym oczom powinien ukazać się następujący
rezultat:

\starttyping
Connecting...
Erasing flash...
Writing at 0x0007cc00... (100 %)

Leaving...
\stoptyping

Moduł jest gotowy do uruchomienia środowiska MicroPython. Po odłączeniu
pinu GPIO0 od masy i zrestartowaniu modułu bootloader załaduje nowy
wsad. Możemy już uzyskać dostęp do REPL (Python prompt) poprzez port
szeregowy.

\starttyping
$ screen /dev/tty.xxx 115200
\stoptyping

Czas na pierwszy program. W świecie programowania mikrokontrolerów
odpowiednikiem wypisania tekstu \quotation{Hello, world} jest program
powodujący miganie diody LED (chociaż w naszym przypadku nic nie stoi na
przeszkodzie wypisania tekstu). Poniżej znajduje się kod, który sprawia,
że niebieska dioda LED zamontowana na module będzie migać z
częstotliwością 1Hz.

\starttyping
>>> import machine
>>> import time
>>> pin = machine.Pin(2, machine.Pin.OUT)
>>> while True:
...     pin.high()
...     time.sleep_ms(500)
...     pin.low()
...     time.sleep_ms(500)
\stoptyping

Gotowe! Właśnie zrealizowaliśmy pierwszy projekt sprzętowy. Więcej
informacji pomocnych dla początkujących dostępne jest w tutorialu
środowiska MicroPython dla układu ESP82666 {[}5{]}.

\subsection[raspberry-pi]{Raspberry Pi}

Raspberry Pi to komputer wielkości karty kredytowej (85×56mm), który
znajduje zastosowanie w projektach elektronicznych zastępując komputer
stacjonarny {[}6{]}. Specyfikacja sprzętowa najnowszej wersji Raspberry
Pi~3 model~B jest następująca: * 64 bit CPU - Quad-Core ARM Cortex A53
@1,2GHz, * Pamięć RAM - 1GB LPDDR2 @900MHz, * Interfejsy - 4×USB 2.0,
UART, SPI, I2C, GPIO, * Sieć - Ethernet 10/100Mbps, Wifi 802.11 b/g/n
150Mbps, Bluetooth Low Energy, BLE 4.1.

Istnieją różne warianty sprzętowe (Model A, Model B, Zero), co pozwala
na dobranie urządzenia do potrzeb projektu. Raspberry Pi doskonale
sprawdzi się jako centrum zarządzania inteligentnym domem. Oficjalnym
system operacyjnym jest Raspbian, jest to dystrybucja linuxa oparta na
Debianie co sprawia, że z łatwością uruchomimy dowolną usługę czy skrypt
niezbędny do obsługi czujników oraz urządzeń w domu. Możliwość
podłączenia wyświetlacza dotykowego o rozdzielczości 800×480 pikseli
pozwala na zapewnienie wygodnego interfejsu użytkownika.

\subsection[komunikacja-z-urządzeniami-wykonawczymi]{Komunikacja z
urządzeniami wykonawczymi}

Trudno wyobrazić sobie system inteligentnego domu bez możliwości
kontrolowania urządzeń zainstalowanych w domu takich jak piec, centrala
wentylacyjna czy instalacja ogrzewania podłogowego. Przeważnie każde z
takich urządzeń posiada dedykowany sterownik, który sprawuje kontrolę
nad pracą danego urządzenia. W przypadku prostych urządzeń sterowanie
włącz/wyłącz może odbywać się przez zwieranie odpowiednich styków za
pomocą przekaźnika. Bardziej zaawansowane sterowniki umożliwiają nie
tylko zdalną kontrolę parametrów pracy, ale także diagnostykę
urządzenia. Czasami do nawiązania komunikacji wymagane jest podłączenie
do sterownika dodatkowego modułu komunikacyjnego.

Często wykorzystywanym protokołem komunikacji z urządzeniami jest
Modbus. Modbus jest prostym i niezawodnym protokołem zapewniającym
komunikację między wieloma urządzeniami w architekturze master slave
{[}7{]}. Komunikacja odbywa się przeważnie przez interfejs szeregowy
RS232 {[}8{]} lub RS485 {[}9{]} (istnieje również odmiana protokołu
nazwana Modbus TCP zapewniająca komunikację przez sieć TCP/IP). W celu
podłączenia Raspberry Pi do magistrali RS232 lub RS485 niezbędny będzie
odpowiedni moduł komunikacyjny (UART-RS232 lub UART-RS485). Obsługę
łączności z urządzeniami zewnętrznymi można zrealizować za pomocą
biblioteki pymodbus {[}10{]}, która oferuje pełną implementację
protokołu Modbus w języku Python.

\subsection[podsumowanie]{Podsumowanie}

Technologia odgrywa coraz większą rolę w naszym życiu, a możliwości,
które oferuje nam w zakresie automatyzacji budynków, niewątpliwie staną
się powszechne w najbliższej przyszłości. Okazuje się, że już teraz
możemy uczestniczyć w rozwoju instalacji inteligentnych budynków
wykorzystując dobrze znane nam narzędzia.

\subsection[bibliografia]{Bibliografia}

\startitemize[n,packed][stopper=.,width=2.0em]
\item
  Oficjalna strona MicroPython i PyBoard. http://www.micropython.org
\item
  Wiki ESP8266. https://nurdspace.nl/ESP8266
\item
  Repozytorium zawierające kod źródłowy portu MicroPython dla układu
  ESP8266.
  https://github.com/micropython/micropython/tree/master/esp8266
\item
  Narzędzie do komunikacji z bootloaderem układu ESP8266.
  \hyphenatedurl{https://github.com/themadinventor/esptool}
\item
  Pierwsze kroki z MicroPython i ESP8266.
  \hyphenatedurl{http://docs.micropython.org/en/latest/esp8266/esp8266/tutorial/index.html}
\item
  Oficjalna strona Raspberry Pi. https://www.raspberrypi.org
\item
  Podstawowe informacje o protokole Modbus.
  http://www.modbus.org/faq.php
\item
  Opis interfejsu RS232. https://pl.wikipedia.org/wiki/RS-232
\item
  Opis interfejsu RS485. https://pl.wikipedia.org/wiki/EIA-485
\item
  Implementacja protokołu Modbus dla Python.
  \hyphenatedurl{https://github.com/bashwork/pymodbus}
\stopitemize


\stoptext
