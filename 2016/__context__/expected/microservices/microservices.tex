\usemodule[pycon-2016]
\starttext

\Title{TDD of Python microservices}
\Author{Michał Bultrowicz}
\MakeTitlePage

\subsection[abstract]{Abstract}

To have a successful microservice-based project you might want to start
testing early on, shorten the engineering cycles, and provide a more
sane workplace for the developers. Test-driven development (TDD) allows
you to have that.

Except for the stalwart unit tests, proper TDD also requires functional
tests. This article shows how to implement those tests (using the
Mountepy {[}1{]} library I made, Pytest and Docker), how to enforce TDD
(using multi-process code coverage) and good code style (with automated
Pylint checks) within a team.

Furthermore, contract tests based on Swagger interface definitions are
introduced as a safeguard of the microservices' interoperability.

The focus is on services communicating through HTTP, but some general
principles will also apply to all web or network services.

\subsection[service-tests]{Service tests}

\section[their-place-in-tdd-and-microservice-development]{Their
place in TDD and microservice development}

To have TDD and thus a maintainable microservice project we need tests
that can validate the entirety of a single application. These tests are
essential for implementing the \quotation{real TDD}, that is the
\quotation{double loop} or \quotation{outside-in} TDD (shown on the
diagram below, taken from \quotation{Test-Driven Development with
Python). The term used to describe them can be}service tests\quotation{,
which I prefer,}functional tests" or \quotation{component tests}.

\placefigure{TDD cycle}{\externalfigure[tdd_cycle.png]}

Service tests are necessary to check if units of code work together as
planned, if it won't just crash when starting in a real environment.
They can do that because they run against a process with no special test
flags, no fake database clients, etc. The application under test should
have no idea that it's not in a \quotation{real} environment.

It's also important to remember that as they take longer to run and are
more complex than unit tests, they should only be used to cover a few
critical code paths. Full validation of the application's logic (code
coverage) should be achieved with unit tests.

In his presentation about testing microservices, Martin Fowler places
service tests in the middle of the tests pyramid.

\placefigure{Microservice test
pyramid}{\externalfigure[test_pyramid.png]}

The general idea is that the higher you get in the pyramid, the tests:

\startitemize[packed]
\item
  become more complex and hard to maintain,
\item
  have greater response time (run longer),
\item
  should be fewer.
\stopitemize

All of those kinds of tests are important for a microservice-based
system, but due to limited space I can't get into detail about all of
them.

\section[necessary-tools]{Necessary tools}

To have local service tests that can run in isolation from the
production or staging environment (allowing for parallel development of
multiple microservices), the service's run-time dependencies must
somehow be satisfied. Assuming we're building HTTP microservices, the
dependencies most probably are:

\startitemize[packed]
\item
  databases
\item
  other microservices (or any other HTTP applications)
\stopitemize

\subsection[handling-databases]{Handling databases}

{\em Naively} - just by installing on the machine running the tests.
This is tiresome and unwieldy. Development of every microservice would
either require long manual setup or maintaining installation scripts
(what can also require a lot of work). What's more, if someone works on
a few projects they'll get a lot of junk on their OS.

{\em Verified Fakes} are test doubles of some system (let's say a
database) that are created by the maintainers (or other contributors) of
said system. They are also tested (verified) to ensure that they behave
like the real version during tests. They can make tests faster and
easier to set up, but require effort to develop, and since time is a
precious resource, are rarely seen in the real world.

Using {\em Docker} is my preferred approach. With it, almost every
application can be downloaded as an image and run as a container in a
uniform way. Without messy or convoluted installation processes.

\subsection[handling-other-microservices]{Handling other
microservices}

The problem here is that if we'd want to simply set up the other
services that are required by the service under tests, we would also
need to set up their dependencies as well. This chain reaction could go
on until we had a large chunk (if not all) of the platform on our
development machine. Even if not absolutely cumbersome, this could prove
impossible.

Fortunately, HTTP services can be mocked (or stubbed) out. There are a
few solutions (mock servers) I came across that can be configured to
start imitating HTTP services on selected ports. The imitation is about
responding to a defined HTTP call, e.g.~POST on path /some/path with
body containing the string \quotation{potato}, with a specific response,
e.g status code 201 and a body containing string \quotation{making
potatoes}. Those mock servers are:

\startitemize[packed]
\item
  {\em WireMock} is a Java veteran that can run as a standalone
  application and be configured with HTTP calls.
\item
  {\em Pretenders}, a Python library and a server (like WireMock) that
  can be used from test code. It requires manually starting the server
  before running the tests.
\item
  {\em Mountebank} is similar in principle to the previous two, but has
  more features, including faking TCP services (which can be used to
  simulate broken HTTP messages). It's written in NodeJS, but it can be
  downloaded as a standalone package not requiring a Node installation.
  I chose it as my service mocking solution.
\stopitemize

To use Mountebank in tests comfortably and not be required to start it
manually before tests, I created Mountepy. It's a library that
gracefully handles starting and stopping Mountebank's process. It also
handles configuration of HTTP service imitations, which Mountebank
refers to as \quotation{imposters}. Since Mountepy required
implementation of management logic for an HTTP service process (which
Mountebank is), it also has the feature of controlling the process of
the application under tests, thus providing a complete solution for
framework-agnostic service tests.

\section[test-anatomy]{Test anatomy}

Because of its conciseness and due to its powerful and composable
fixture system, I've picked Pytest as a test framework. An example
service test created with it can look like any other plain and simple
unit test:

\starttyping
# Fixtures are test resource objects that are injected into the test
# by the framework. This test is parameterized with two of them:
# "our_service", a Mountepy handler of the service under test
# and "db", a Redis (but it could be any other database) client.
def test_something(our_service, db):
    db.put(TEST_DB_ENTRY)
    response = requests.get(
        our_service.url + '/something',
        headers={'Authorization': TEST_AUTH_HEADER})
    assert response.status_code == 200
\stoptyping

It's quite clear what's happening in the test: some test data is put in
the database, the service is called through HTTP, and finally, there's
an assertion on the response. Such straightforward testing is possible
thanks to the power of Pytest fixtures. Not only can they parameterize
tests, but also other fixtures. The two top-level ones presented above
are themselves composed of others (as shown in the diagram below) in
order to fine-tune their behavior to our needs.

\placefigure{Fixture composition}{\externalfigure[fixtures.png]}

Let's now trace all the elements that make up our test fixtures. First,
\type{db}:

\starttyping
import docker
import pytest
import redis

# "db" is function scoped, so it will be recreated on each test function.
# It depends on another fixture - "db_session".
@pytest.yield_fixture(scope='function')
def db(db_session):
    # "yield" statement in "yield fixtures" returns the resource object
    # to the test.
    # "db" simply returns the object returned from "db_session".
    yield db_session
    # Code after "yield" is executed when the tests go outside
    # of the fixture's scope,
    # in this case - at the end of a test function.
    # This is the place to write cleanup code for fixtures,
    # no callbacks required.
    # Here the cleanup means deleting all the data in Redis.
    db_session.flushdb()

# "session" scope means that the object will be created only once
# for the entire test suit run.
@pytest.fixture(scope='session')
def db_session(redis_port):
    return redis.Redis(port=redis_port, db=0)

# This fixture simply returns a port number for Redis, but has the side
# effect of creating and later destroying a Docker container (with Redis).
# Thanks to being session-scoped it doesn't need to spawn a new container
# for each test, thus cutting down the test time.
# This practice may be looked down upon by people paranoid
# about test isolation, but if Redis creators did their job well,
# cleaning the database in "db" should be enough
# to start each service test on a clean slate.
@pytest.yield_fixture(scope='session')
def redis_port():
    docker_client = docker.Client(version='auto')
    # Developers don't need to download required images themselves,
    # they only need to run the tests.
    # Pulling an image will, of course, takes some time and
    # freeze the tests, but it's one-time.
    download_image_if_missing(docker_client)
    # Creates the container and waits for Redis
    # to start accepting connections.
    container_id, redis_port = start_redis_container(docker_client)
    yield redis_port
    docker_client.remove_container(container_id, force=True)
\stoptyping

\ldots{}and next, \type{our_service}:

\starttyping
import mountepy

@pytest.fixture(scope='function')
def our_service(our_service_session, ext_service_impostor):
    return our_service

# Creates (and later destroys) the process of the service under test.
# The same as with "db" fixture, it's created once
# for the whole test session to save time.
# The risk of tests influencing each other is present, but any improper
# behavior shows that the application is not stateless (which it should
# be if follows the tenets of 12-factor app), so the tests do their job
# of finding bugs that we are writing and not something that the whole
# community depends on, like Redis.
@pytest.yield_fixture(scope='session')
def our_service_session():
    # Starting a service process with Mountepy requires a shell command
    # in Popen format. The app will run in Waitress (a WSGI container,
    # an alternative to gunicorn, uWSGI, etc.).
    service_command = [
        WAITRESS_BIN_PATH,
        '--port', '{port}',
        '--call', 'data_acquisition.app:get_app']

    # Spawning the service is straightforward.
    service = mountepy.HttpService(
        service_command,
        # Configuration is passed through environment variables.
        env={
            'SOME_CONFIG_VALUE': 'blabla',
            'PORT': '{port}',
            'PYTHONPATH': PROJECT_ROOT_PATH})

    service.start()
    yield service
    service.stop()

@pytest.yield_fixture(scope='function')
def ext_service_impostor(mountebank):
    # Impostor is created in Mountebank, the object able to communicate
    # with it is returned. The configured behavior is simple in this case,
    # it will respond to a POST on the given port and path (e.g.
    # "/some/resource") with an empty response body and status code 204.
    impostor = mountebank.add_imposter_simple(
        port=EXT_SERV_STUB_PORT,
        path=EXT_SERV_PATH,
        method='POST',
        status_code=204)
    yield impostor
    # After each tests the impostor is destroyed and all the requests
    # it received are forgotten.
    impostor.destroy()

# The Mountebank instance is also created once for the test suite.
@pytest.yield_fixture(scope='session')
def mountebank():
    mb = Mountebank()
    mb.start()
    yield mb
    mb.stop()
\stoptyping

Real code demonstrating the solutions presented in this article can be
found in PyDAS {[}2{]}, which was my guinea pig for microservice TDD
experiments.

\section[remarks-about-service-tests]{Remarks about service tests}

Tests that start a few processes and send real HTTP requests (even
through the loopback interface) tend to be orders of magnitude slower
than (proper) unit tests. But in the case of PyDAS, which has 8 service
tests, 3 integrated ones (almost like a unit test, but interact with a
real Redis), and 40 unit tests, they take around 3 seconds (on Python
3.4). That's quite fast. In my opinion, you can run tests that take that
long after every few lines of written code.

Short test runs have the advantage of people actually running them.
Developers can shy away from tests with run-time long enough to break
their flow. And when tests are not run, the entire effort and good
intentions put into them go to waste.

A word of advice - even if the whole suit is quite fast (like 3
seconds), it's good to keep the longer tests (integrated and service
ones) in a directory separate from unit tests, to still be able to
sometimes run only the fastest subset (e.g.~when checking really small
tweaks one after another).

There's also a small caveat about service test failures. When a test
fails in Pytest, all of its output is printed in addition to the test
stack trace. Service tests start a few processes, probably all of which
print quite a few messages, so when they fail you'll be hit with a big
wall of text. The upside is that it this text will most probably state
somewhere what went wrong. Breaking a fixture sometimes happens when
experimenting with and refactor the tests (which I encourage). This can
yield even crazier logs that simply failed service tests.

And the last thing - tests won't save you from all instances of human
incompetence. When I created PyDAS using TDD and wanted to deploy it to
our staging environment, it kept crashing. It turned out that I was
ignoring Redis IP from configuration and had hard-coded localhost, which
was fine with the tests, but didn't at all do in a real environment. So
be confident in your tests, but never a hundred percent.

\subsection[enforcing-tdd]{Enforcing TDD}

Even if your team knows how to do TDD, they sometimes want to cut
corners, which isn't good for anyone in the long run. Luckily, there are
ways to keep them (and you, and me) in line.

\section[measuring-code-coverage]{Measuring code coverage}

Let's take a look at my recommended minimal \type{.coveragerc} file
(configuration for python \quotation{coverage} library)

\starttyping
[run]
source = your_project_source_directory
; This enables (with a few other tricks documented at
; http://coverage.readthedocs.io/en/coverage-4.0.3/subprocess.htm
; and done in PyDAS) test coverage measurement from multiple processes.
; That is, when Mountepy runs the tested service and requests are fired
; against it, the coverage information from the code hit when handling
; the requests will be added to overall coverage data.
; Now there's no need to duplicate scenarios from service tests
; into unit tests for coverage measurement.
parallel = true

[report]
; All code needs to be tested. Especially in a dynamic language
; like Python, which does almost no static validation.
; If even one untested statement is written, the tests will fail.
; With parallel coverage, there's no need to write meaningless tests
; of glue-code that run almost 100% on mocks to ramp up to absolute
; coverage, because the code will be hit by the service test.
fail_under = 100
\stoptyping

\section[mandatory-static-code-analysis]{Mandatory static code
analysis}

To keep code quality high it's good to use static code analysis tool
like Pylint. To keep it even higher, any meaningful complaints can cause
the test suite to fail.

In PyDAS, I've used Tox {[}3{]} to automate test runs. Below is an
abbreviated version of Tox configuration (\type{tox.ini}).

\starttyping
[testenv]
commands =
    ; Pylint is run before tests.
    ; If it returns any output, which happens when it finds errors,
    ; the whole Tox run fails.
    /bin/bash -c "pylint data_acquisition --rcfile=.pylintrc"
    ; Running the tests with coverage measurement.
    coverage run -m py.test tests/
    coverage report -m
\stoptyping

Remember that Pylint's config can be tweaked to lower its standards,
which can be too high at times. Specific code lines can also be
annotated to ignore a Pylint check if you're absolutely sure that what
you're doing is the best way and Pylint is wrong to scold you.

\subsection[contract-tests-with-swagger-and-bravado]{Contract tests with
Swagger and Bravado}

Sometimes slight changes in code, like adding a field to some
unremarkable data container type or adding a small \quotation{if} in
some REST controller can accidentally:

\startitemize[packed]
\item
  change the expectation a service has of its functions' parameters or
\item
  change the shape (schema) of the data returned by the service.
\stopitemize

Those changes breach the contract the service has with the outside
world, and can cause bugs in a microservice system, so precautions are
necessary. They come in the form of contract tests.

\section[swagger]{Swagger}

Swagger {[}4{]} is an interface definition language. It will serve us as
a contract definition language description.

An example Swagger document written in YAML format (JSON also happens)
is below.

\starttyping
swagger: '2.0'
info:
  version: "0.0.1"
  title: Some interface
paths:
  /person/{id}:
    get:
      parameters:
        -
          name: id
          in: path
          required: true
          type: string
          format: uuid
      responses:
        '200':
          description: Successful response
          schema:
            title: Person
            type: object
            properties:
              name:
                type: string
              single:
                type: boolean
\stoptyping

It defines a HTTP endpoint on path /person/\{id\} on the service's
location (e.g.~http://example.com/person/zablarg13). This endpoint will
respond to a GET request. It has one parameter, \type{id}, that is
passed in the path and is a string. The endpoint can respond with a
message with status code 200 containing the representation of a person's
data - and object with their name as a string and the information if
they are single (boolean).

\section[contractservice-tests]{Contract/service tests}

Bravado {[}5{]} is a library that can dynamically create client objects
for a service based on its Swagger contract. It can do contract tests by
automatically validating the types (schemas) of both parameters and the
values returned from services. Can verify HTTP status codes. Status code
and response schema combinations that don't exist in Swagger are treated
as invalid, e.g.~if we can return a Person object with code 200 and an
empty body with code 204, returning a Person with 204 will cause an
error.

It's worth noting that Bravado can be configured to enable or disable
different validation checks. This is helpful in testing the unhappy path
through your service.

\starttyping
# Contract is in a file separate from code
# (definitely not generated from it, which is sometimes done),
# so that when changes break the contract it can be detected.
@pytest.fixture()
def swagger_spec():
    with open('api_spec.yaml') as spec_file:
        return yaml.load(spec_file)

def test_contract_service(swagger_spec, our_service):
    # Bravado client will be used instead of "requests" to call
    # the service. Service tests using Bravado clients double as contract
    # tests with practically no added effort except for maintaining
    # the Swagger spec.
    client = SwaggerClient.from_spec(
        swagger_spec,
        origin_url=our_service.url))

    request_options = {
        'headers': {'authorization': A_VALID_TOKEN},
    }

    # Running and validating the request with a body, a path parameter
    # "worker" and with an authorization header containing
    # a valid security token.
    resp_object = client.v1.submitOperation(
        body={'name': 'make_sandwich', 'repeats': 3},
        worker='Mom',
        _request_options=request_options).result()

    assert resp_object.status == 'whatever'
\stoptyping

\section[contractunit-tests]{Contract/unit tests}

It's unfeasible to cover the entire contract with service/contract
tests, because they take too long. It would be great to give unit tests
that simulate HTTP calls (I think all HTTP/REST frameworks have a
facility to do that) the Bravado's validation abilities. That's what I
did for Falcon {[}6{]} with bravado-falcon {[}7{]}, thanks to Bravado's
extensibility. Such integration can probably be easily developed for
other web frameworks.

\starttyping
from bravado.client import SwaggerClient
from bravado_falcon import FalconHttpClient
import yaml
import tests # our tests package

# swagger_spec is the same fixture as in the contract/service test.
def test_contract_unit(swagger_spec):
    # Client doesn't need an URL now, but it needs
    # the alternative HTTP client.
    client = SwaggerClient.from_spec(
        swagger_spec,
        http_client=FalconHttpClient(tests.service.api))

    resp_object = client.v1.submitOperation(
        body={'name': 'make_sandwich', 'repeats': 3},
        worker='Mom').result()

    assert resp_object.status == 'whatever'
\stoptyping

\subsection[conclusion]{Conclusion}

Solutions presented here will help keep you sane when working on a
microservice-based system, but are far from being everything you need to
know to about microservice development. Further things to consider are:

\startitemize[packed]
\item
  end-to-end tests,
\item
  performance tests,
\item
  operations automation (deployment, data recovery, service scaling,
  etc.),
\item
  monitoring of services and infrastructure,
\item
  and more.
\stopitemize

\subsection[references]{References}

\startitemize[n,packed][stopper=.,width=2.0em]
\item
  Mountepy. https://github.com/butla/mountepy
\item
  PyDAS. https://github.com/butla/pydas
\item
  Tox. https://tox.readthedocs.io
\item
  Swagger. http://swagger.io/
\item
  Bravado. https://github.com/Yelp/bravado
\item
  Falcon. https://falconframework.org/
\item
  bravado-falcon. https://github.com/butla/bravado-falcon
\item
  Gary Bernhardt. Fast Test, Slow Test. https://youtu.be/RAxiiRPHS9k
\item
  Sam Newman. Building Microservices. O'Reilly Media, Inc., February 10,
  2015
\item
  Harry J.W. Percival. Test-Driven Development with Python. O'Reilly
  Media, Inc., June 19, 2014
\item
  Tobias Clemson. Microservice Testing.
  http://martinfowler.com/articles/\crlf microservice-testing/
\item
  Testing in production comes out of the shadows.
  http://sdtimes.com/testing-production-comes-shadows/
\stopitemize


\stoptext
