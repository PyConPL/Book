\usemodule[pycon-2016]
\starttext

\Title{Finding a business model for Open Source Applications - case study of Splash}
\Author{Paweł Miech}
\MakeTitlePage

\useURL[url1][http://splash.readthedocs.io/en/stable/][][Splash]\from[url1]
is an open source scriptable headless browser with HTTP API, written in
Python 3 with Twisted \type{&} PyQt. I'd like to present the project and
discuss a couple of social and technical phenomena that I noticed while
taking part in the development of this application. I'd like to discuss
how the application is structured, what libraries we used and how they
served us, and finally discuss how open source works for us and how I
perceive it.

My presentation will have two parts, one technical and other
\quotation{social/psychological}. In the first one I'll outline the
application architecture. In the second one I will try to discuss some
general issues that I noticed around Open Source.

\subsection[what-is-splash-and-how-it-works]{What is Splash and how it
works}

Splash is a headless browser based on WebKit engine. Its main job is
rendering HTML and JavaScript. You can interact with Splash by sending
HTTP requests with some parameters. For example, you can make request to
the \type{render.png} endpoint with \type{url} parameter and it will
send request to the given url, load it into the browser window and
return screenshot in .png format for you. You can also send a request to
the \type{render.HTML} endpoint and it will return HTML.

Splash is scriptable, which means you can write Lua scripts, that tell
Splash what to do. For example, you can write a script that will tell
the browser to click on some link, wait a minute, hover the cursor over
some page area and finally return screenshot of a webpage after
performing those actions.

Why did we decide to create Splash? The main reason is that we needed
some rendering for Scrapy. Scrapinghub was founded by the founders of
Scrapy - a framework for creating web crawlers. Today the main challenge
for all crawlers is JavaScript. If you want to know how important
JavaScript is for modern web try disabling JavaScript in the browser,
and try browsing your favorite webpages. I can guarantee you that in
many cases you won't see anything. If you do manage to browse some
content you'll see terribly broken layouts, HTML elements completely out
of order, divs and headings flying all over the place, missing data. In
general you'll see chaos and mayhem. Long time ago people believed in a
thing called \quotation{graceful degradation}, meaning that if a user
didn't have JavaScript, the page should still work, but in a limited
way. These days no one really does graceful degradation. If you don't
have a modern browser with JavaScript enabled, you're not able to
benefit from all the good content created by web authors.

Now you can of course bypass these limitations, usually web pages are
doing plain HTTP requests in the background using AJAX, and AJAX is just
a normal HTTP request that you can imitate easily from a spider. But
most people don't know this, or are lazy and don't want to spend hours
of their time reverse engineering web pages by looking into developer
tools, so they decide they prefer to render the page. There are also
some use cases when you really need to render JavaScript (e.g.~you need
a screenshot of the page or you need to deal with some bot protections
that require you to solve a JS challenge).

Why we decide to create Splash and not go with another solution? In
other words why not Selenium or phantom JS? Main reason is that we
really needed to have headless browser with an HTTP API. We didn't want
to add JavaScript rendering into Scrapy, but we thought that having a
separate service doing the rendering will be better. So it was crucial
to have an HTTP API. None of the existing JS rendering services provided
this out of the box. Now you can probably add an HTTP API to existing
rendering frameworks, but once you start doing this you realize it's
hard work and it's not that easy to tie it to the existing software. On
the other hand, getting a library for rendering web pages is not
difficult in Python - look no further than PyQt. In fact, in our case
the rendering is done by Qt Web Kit.

Under the hood Splash architecture is simple. The HTTP API part is
written in Python Twisted. It makes a heavy use of Twisted Deferred and
Twisted.Resources. The main benefit it gives us is that the application
is asynchronous, which is absolutely crucial for this type of task.
After a request is received by the application, it is forwarded to the
BrowserTab object. Every browser tab object gets a web page. Web page
inherits from Qt QWebView and does all the rendering, it executes page
JavaScript, renders the DOM and returns a result.

\subsection[how-open-source-works-in-case-of-splash]{How open source
works in case of Splash}

There is a popular misconception that treats open source projects as
some sort of charity work. In this perception people doing open source
are mostly volunteers, who just try to \quotation{help others} by making
useful software. They don't gain anything from their work, they just
contribute to projects because of their good will. Doing open source
from this point of view is similar to helping older people cross the
street or helping starving kids in Africa.

This perception of open source is not entirely untrue, for example Bram
Moolenaar, author of Vim does encourage people to donate to kids in
Africa, but it might be sometimes misleading. I'd like to try to
challenge this myth and try to talk about various forms of business you
can make around your open source. I'd argue that having some form of
business organization around your open source projects is good both for
you, and for your projects. By having some business model, you can stop
relying on completely unreliable and erratic contributions of strangers.
You are able to somehow finance the development time and the efforts of
the core team.

Splash a is nice example of how you can implement these ideas in
practice, so in my presentation I'd like to demonstrate how its business
model works, what we gain from open source and why, in our specific
case, it's not really charity but rational choice.


\stoptext
