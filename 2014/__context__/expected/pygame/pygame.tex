\usemodule[pycon-2014]
\starttext

\Title{Pygame - gry w świecie Pythona}
\Author{Łukasz Jagodziński}
\MakeTitlePage

\subsection[jak-zacząć]{Jak zacząć?}

Od najmłodszych lat, każdy z nas ma do czynienia z grami planszowymi,
komputerowymi, konsolowymi czy innym rodzajem gier. W późniejszym wieku
lub nawet w młodszych latach, niektórzy z nas, bądź większość, chciała
stworzyć własną grę, w którą gracze chcieliby grać. W trakcie nauki
programowania nasze możliwości stworzenia gry komputerowej znacznie się
zwiększały. Na zajęciach postaramy się napisać grę w naszym ulubionym
języku, przy użyciu biblioteki pygame {[}1{]}. Ale zanim to nastąpi,
zaczniemy krótkim teoretycznym wstępem, abyśmy mogli podejść do tematu
jak najbardziej profesjonalnie.

\subsection[czym-tak-naprawdę-jest-gra]{Czym tak naprawdę jest gra?}

Gra jest to zabawa, która ma swoje zasady, dzięki którym możemy
regulować jej stan, rozgrywkę oraz wygrać lub przegrać {[}2{]}. Po
zapoznaniu się z terminem słowa gra, możemy usiąść i zabrać się za
przygotowanie rodzaju gry, zasad, świata czy postaci.

\subsection[i-co-dalej]{I co dalej?}

Jak zapewne wiecie, nie ma \quotation{przepisu} na dobrą grę, która
osiągnie niesamowity sukces. Większość z nas podczas rozgrywki chciała
zmienić reguły gry czy dodać coś, co wzbogaciłoby rozgrywkę. Dowodzi to,
że każdy z nas ma pomysł, który nic nie kosztuje, a może stworzyć nową
grę, czy też wzbogacić nas o doświadczenie, jak dana rzecz faktycznie
zadziała i czy zadziała poprawnie. W pewien sposób jest to wstęp do
projektowania gier, czy też samej produkcji gry. Jedną z metody
rozpoczęcia pracy nad grą jest stworzenie GDD (Game Design Document)
{[}3{]}, czyli dokumentu opisującego funkcjonalność i opis samej gry.
Następnie można sprawdzić rozpisany dokument i zadać sobie pytanie:
{\em Czy chcielibyśmy sami zagrać w tę grę?}, pozwoli to nam na
określenie grywalności gry według własnego uznania - {\em Jeżeli nie
tworzymy gry, w którą chcielibyśmy zagrać, to po co w ogóle to robić?}.

\subsection[jak-to-ma-się-w-świecie-gier-komputerowych]{Jak to ma się w
świecie gier komputerowych?}

Jak powszechnie wiadomo, w grach komputerowych królują inne języki jako
te, w których się je tworzy, najbardziej popularne to C++, C\#,
JavaScript, HTML5 czy Flash. Natomiast języki takie jak Python czy Lua,
często są wykorzystywane jako języki pomocnicze do pisania szybkich
skryptów, ale niestety samo pisanie gry w tych języka często jest mniej
optymalne niż w wyżej wymienionych. Język C++ jest najczęściej
wykorzystywany do tworzenia silników gry (ang. game engine) {[}4{]}, a
Python jest używany, aby oskryptować odpowiednie narzędzia silnika
graficznego. Oczywiście dla języków Lua czy Python istnieją również
silniki czy biblioteki, aby wspomóc te języki w programowaniu gier.
Istnieją odpowiednie silniki w zależności od naszych potrzeb, tj. do
gier 2D i 3D. Dla języka Python najpopularniejsze silniki zostały
przedstawione na wiki Pythonowym {[}5{]}. A jeżeli chodzi o język Lua,
tutaj mamy trochę gorszą sytuację, gdyż tych silników jest mniej, a
niestety nie ma ich wszystkich spisanych, prawdopodobnie przyczyną jest
mniejsza popularność tego języka niż Pythona {[}6{]}. Natomiast jednym z
najbardziej popularnych silników jest LOVE {[}7{]}.

Na naszych zajęciach wykorzystamy bibliotekę {\em Pygame} {[}1{]}.
Zapoznamy się z funkcjami tej biblioteki oraz spróbujemy stworzyć grę. W
trakcie warsztatów również poruszymy kwestię, jak w łatwy sposób
stworzyć grę przy użyciu mechaniki oraz wykorzystania gotowych rzeczy,
które można znaleźć w sieci tj. assety (dźwięki, grafiki). Wybierzemy
również jeden z rodzajów gier, na którym się skupimy i postaramy się w
ciągu kilku godzin stworzyć pierwsze arcydzieło. :)

\section[rodzaje-gier]{Rodzaje gier:}

\startitemize
\item
  gry rekreacyjne
\item
  gry logiczne
\item
  gry platformowe
\item
  gry zręcznościowe
\item
  gry przygodowe
\item
  gry akcji
\item
  gry sportowe, wyścigi
\item
  gry fabularne (cRPG), MMORPG
\item
  gry strategiczne
\item
  gry symulacyjne
\item
  survival horrory
\item
  gry edukacyjne
\stopitemize

\section[podsumowanie]{Podsumowanie}

Aby napisać grę, wystarczy pomysł, mechanika oraz grafika i dźwięk. W
przypadku, gdy brakuje nam umiejętności do stworzenia grafiki, muzyki
czy dźwięków, istnieją serwisy, gdzie można je pobrać i wykorzystać
całkowicie za darmo. Odpowiednio stworzony dokument GDD oraz
zbalansowanie świata gry, pozwoli na stworzenie gry, która będzie
grywalna. Oczywiście wszystkie elementy, z którymi się zapoznamy, nie
dadzą nam możliwości stworzenia gier takich jak GTA, BattleField czy
nawet mniej znanych tytułów, ale od czegoś trzeba zacząć, a przy okazji
pobawimy się językiem Python i mam nadzieję, że stworzymy w pełni
funkcjonalne i grywalne gry. Być może uda się wspólnie wypracować
dzieło, którym będziemy mogli się pochwalić na następnym PyConie. :)

\section[referencje]{Referencje}

\startitemize
\item
  {[}1{]} \useURL[url1][http://www.pygame.org/][][Oficjalna strona
  biblioteki PyGame]\from[url1]
\item
  {[}2{]} \useURL[url2][http://pl.wikipedia.org/wiki/Gra][][Znaczenie
  słowa \quotation{gra} wg wikipedii]\from[url2]
\item
  {[}3{]}
  \useURL[url3][http://www.gamasutra.com/view/feature/131632/creating_a_great_design_document.php][][Jak
  stworzyć Game Design Document]\from[url3]
\item
  {[}4{]} \useURL[url4][http://en.wikipedia.org/wiki/Game_engine][][Co
  to jest silnik gry]\from[url4]
\item
  {[}5{]}
  \useURL[url5][https://wiki.python.org/moin/PythonGames][][Silniki
  Pythonowe]\from[url5]
\item
  {[}6{]}
  \useURL[url6][http://www.gamefromscratch.com/post/2012/09/21/Battle-of-the-Lua-Game-Engines-Corona-vs-Gideros-vs-Love-vs-Moai.aspx][][Silniki
  dla lua]\from[url6]
\item
  {[}7{]} \useURL[url7][http://love2d.org/][][Oficjalna strona
  Love2D]\from[url7]
\stopitemize


\stoptext
