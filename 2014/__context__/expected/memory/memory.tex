\usemodule[pycon-2014]
\starttext


\section[python-memory-workshop---tomasz-paczkowski-piotr-przymus]{Python
Memory Workshop - Tomasz Paczkowski, Piotr Przymus}

A short \quotation{hands on} workshop on Python memory management. This
workshop will cover basics of CPython memory usage and show essential
differences between cPython and other implementations. We will start
with basics like objects and data structures representation. Then
advanced memory management aspects, such as sharing, segmentation,
preallocation or caching, will be discussed. Finally, memory profiling
tools will be presented.

\subsection[introduction]{Introduction}

Working with Python does not usually involve debugging memory problems:
the interpreter takes care of allocating and releasing system memory and
you get to enjoy working on real world issues. But what if you encounter
such problems?

What if your program: * never releases memory, * allocates way too much
memory, * or you suspect that it is memory inefficient ?

This workshop will introduce you to the basics of Python Memory model,
following with more advanced topics. During this tutorial you may learn
basic memory profiling and debugging tools. Some advanced
(implementation depended) topics will be discussed.

\subsection[workshop-topics]{Workshop topics}

Workshop will be divided into various topics. Each topic will include
series of practical exercises.

\section[basic-memory-model]{Basic memory model}

This topic will cover most of the memory basics: * Basic objects memory
representation - what is the actual size of basic types, how to check
it, various types features * Differences in basic objects
representations among various Python implementations - PyPy, Stackless
Python, Jython * Different data representations and how they affect
memory consumption - this will cover: * Old style classes, New style
classes, New style classes with slots, Named tuples, tuples, lists and
dictionaries

\section[advanced-memory-topics]{Advanced memory topics}

Here some advanced concepts will be introduced, like: * Object and
String interning - this will explain objects are preallocated and are
shared instead of new allocation. And more importantly this also will
cover what are the motivations of this approach. * Mutable Containers
Memory Allocation Strategy - this will explain basics of memory strategy
for containers. Examples for lists and dictionaries will be provided. *
Notes with examples on Python garbage collector, reference count and
cycles. * Differences in garbage collector among various Python
implementations - PyPy, Stackless Python, Jython

\section[basic-memory-profiling-and-debugging-tools]{Basic memory
profiling and debugging tools}

This topic will be entirely divided to various Python memory debugging
and profiling tools: * Memory usage monitoring tools, such as: psutil,
memory\letterunderscore{}profiler,
run\letterunderscore{}snake\letterunderscore{}run * Object reference
counting investigation: objgraph * Python heap analysis: Meliae, Heapy

\section[cpython-memory-allocator-introduction]{CPython memory
allocator introduction}

This topic will go down to the internals of Python memory management,
this will cover very advanced topics: * Memory fixing techniques *
Python memory fragmentation * mallloc replacement considerations *
testing memory problems with CPython 3.4 custom memory allocator


\stoptext
