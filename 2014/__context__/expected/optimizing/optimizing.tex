\usemodule[pycon-2014]
\starttext


\section[designing-api-for-mobile-apps---wojtek-erbetowski]{Designing
API for mobile apps - Wojtek Erbetowski}

Designing an API is an incredibly important moment for a project's
future. Every change made in the future will be accompanied by having to
support multiple versions of an API, modifying docs, server-side code
and the client. Because of this laundry list of changes to keep track
of, we try to solve as many design problems as possible up front.\crlf
There are several different areas, that make a huge impact. The one we
target in this post is latency.

\startblockquote
Latency is a time interval between the stimulation and response
(\ldots{}). Wikipedia, Latency (engineering)
\stopblockquote

What is latency when it comes to mobile devices? Mobile networks are
still far from what we would call high performance. There is huge
overhead required when you make even a single request from your mobile
app. A quick list of the steps needed:

\startitemize
\item
  RRC negotiation
\item
  DNS lookup
\item
  TCP handshake
\item
  TLS handshake
\item
  HTTP request
\stopitemize

Beyond just this one complicated path, we need to keep in mind some
potential problems:

\startitemize
\item
  weak signal
\item
  lost signal
\item
  client is moving and changing base stations
\item
  routing gets mixed up -
  \useURL[url1][http://calendar.perfplanet.com/2012/latency-in-mobile-networks-the-missing-link/][][source]\from[url1]
\item
  different pings originating from the same location(home/office) -
  \useURL[url2][http://blog.cloudflare.com/why-mobile-performance-is-difficult][][source]\from[url2]
\stopitemize

You end up with a latency of 100ms-300ms for 3G networks and up to
1000ms for 2G. Since 2G is becoming obsolete and 4G is just around the
corner for our users (check this
\useURL[url3][http://serverfault.com/a/573815][][thread]\from[url3]),
let's take a look at a forecast from Cisco Systems, Inc.
(\useURL[url4][http://www.cisco.com/c/en/us/solutions/collateral/service-provider/visual-networking-index-vni/white_paper_c11-520862.html][][source]\from[url4]).

\placefigure[here,nonumber]{CISCO mobile connection speed
forecast}{\externalfigure[cisco.png]}

What this shows is that 4G shouldn't be treated as the representative
network connection of our users. Of course we should be able to target
our application well and if it's an app for geeks, let's say a
\useURL[url5][http://stackoverflow.com/][][StackOverflow]\from[url5]
client, the subsequent network connection speed will be drastically
higher than when compared to a general purpose app like a news reader.

We can deal with latency by trying to optimize DNS lookup counts,
shorten the request route, and many many more. We could even go so far
as to spend
\useURL[url6][http://www.extremetech.com/extreme/122989-1-5-billion-the-cost-of-cutting-london-toyko-latency-by-60ms][][25
million dollars]\from[url6] to improve it by another 1ms. However, this
article is addressing a rather specific part connected to API design.
So, how can we influence the latency by API design?

\subsection[merging-responses]{Merging responses}

First, let's make a use of an example. Let's say we have a view that
says user name, age and country. Name and age are a part of
\mono{/users/\letteropenbrace{}id\letterclosebrace{}} resource:

\starttyping
{
  "name": "John Doe",
  "age": 25,
  "links": [
    {
      "rel": "self",
      "href": "/users/123"
    },
    {
      "rel": "account",
      "href": "/accounts/987"
    },
    {
      "rel": "address",
      "href": "/users/123/address"
    }
  ]
}
\stoptyping

To display a country, we need to make another call, for
\mono{/users/\letteropenbrace{}id\letterclosebrace{}/address}. This
means following the entire route to reach the server again:

\starttyping
{
  "street": "Sesame",
  "no": 25,
  "zipCode": "12-321",
  "state": "NY",
  "country": "US"
}
\stoptyping

But since there's no real reason to make both of those calls, let's try
to save one connection and prepare a special endpoint, dedicated for
that screen:

\starttyping
{
  "name": "John Doe",
  "age": 25,
  "country": "US"
}
\stoptyping

Now instead of making two separate calls we may just call a single
endpoint and get all of the required information. But remember that
there is a tradeoff. To get a more robust API we're stepping away from a
REST design pattern and binding together two related entities - user and
address.

\subsection[expansion]{Expansion}

We don't always know the way in which our API will be used. In the last
example we implicitly assumed that we knew what information was being
displayed. But what about cases where we design a service for future
consumers which may present data in any way they decide? Yet we still
want to give the option to optimize answers.

In that case an expansion pattern becomes very handy. Going to our
previous example, what would this look like? We need to prepare a
request that specifies which fields to display (and embed from related
entities). \type{GET /users/123} now becomes
\type{GET /users/123?fields=[name,age,address[country]]}.

\starttyping
{
  "name": "John Doe",
  "age": 25,
  "address": {
    "country": "US"
  }
}
\stoptyping

This is a pretty neat mechanism for merging responses, although it has
it's own limitations as well. If API consumers are using relations, we
won't be able to change the resource hierarchy without breaking clients.

\subsection[keep-alive]{Keep-alive}

The closing trick is to make sure your service supports the HTTP
keep-alive option. By default, all mobile HTTP clients will try to
sustain the HTTP connection between different calls so it makes perfect
sense to utilize this. If the server will agree to keep the connection,
the next requests will omit several handshakes and directly be sent to
server within an opn connection.

\subsection[closing-word]{Closing word}

In Polidea we are developing mobile apps and supporting them with
backend services. The last couple of years has taught us that you may
not achieve a proper user experience without slick HTTP communication.
This article was inspired by our experiences from real-world projects.
As always, we welcome any feedback and would be happy to discuss this
topic as well as our work in further details.


\stoptext
