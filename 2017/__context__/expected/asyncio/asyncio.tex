\usemodule[pycon-yyyy]
\starttext


\section[asyncio-w-praktyce-warsztat]{Asyncio w praktyce (warsztat)}

Podczas tych warsztatów nauczysz się korzystać z {\bf asyncio} {[}1{]}
na tyle, by móc samodzielnie zaprojektować i zakodować aplikację
korzystającą z jego dobrodziejstw.

Nauka oprze się o pisanie w pełni funkcjonalnego czatu. W trakcie
ćwiczeń poznasz część biblioteki asyncio potrzebną do zrealizowania
zadania, nową składnię dla coroutines z async i await, bibliotekę
{\bf aiohttp} {[}1{]} zapewniającą klienta i serwer HTTP oraz tajniki
testowania powyższych. Pisanie testów przed kodem będzie częścią
szkolenia, ale w razie potrzeby może zostać pominięte, by nadal móc coś
z niego wynieść.

Do testowania będziemy używać biblioteki {\bf pytest}{[}4{]}, ale w
minimalnym stopniu - jego nieznajomość w niczym nie będzie przeszkadzać.

Najważniejszą rzeczą, którą wyniesiecie ze szkolenia będzie umiejętność
projektowania aplikacji opartych na asyncio. Poznacie mocne strony
biblioteki i sytuacje w których jest odpowiednim narzędziem. Nauczycie
się także unikania najczęstszych pułapek związanych z asyncio.

\starttyping
import asyncio
import datetime

PYCON_PL_START_DATE = datetime.date(2017, 8, 17)

loop = asyncio.get_event_loop()
asyncio_workshop = asyncio.Future()

def check_workshop_time():
    today = datetime.date.today()
    if today >= PYCON_PL_START_DATE:
        asyncio_workshop.set_result('Przyjdź na warsztat!')
    else:
        days_left = (PYCON_PL_START_DATE - today).days
        print(f'Pozostało {days_left} dni')
        loop.call_later(86400, check_workshop_time)

async def attend_pycon_pl():
    result = await asyncio_workshop
    print(result)


loop.call_later(1, check_workshop_time)
loop.run_until_complete(
    attend_pycon_pl()
)
loop.close()
\stoptyping

\subsection[dla-kogo]{Dla kogo?}

Szkolenie jest adresowane do osób, które programują już chwilę w
Pythonie i chcą poznać jego asynchroniczną stronę. Każdy
średniozaawansowany lub ogarniający początkujący programista Pythona da
sobie radę z ćwiczeniami. Znajomość technologii frontendowych nie jest
wymagana - ta część zostanie zapewniona przez organizatora warsztatów.

\subsection[jak-się-przygotować]{Jak się przygotować?}

Potrzebny będzie Python w wersji 3.5 lub 3.6, umiejętność obsługi pip'a,
virtualenva i podstawowa znajomość git'a - na tyle, by móc sklonować
sobie repozytorium, utworzyć wirtualne środowisko i zainstalować
zależności potrzebne do ćwiczeń:

\starttyping
git clone https://github.com/Enforcer/asyncio-workshop
cd asyncio-workshop
python3.6 -m venv ve
source ve/bin/activate
pip install -r requirements.pip
\stoptyping

I to wszystko.

Edytor kodu nie ma znaczenia, używajcie swojego ulubionego. W razie
wątpliwości - Pycharm Community Edition na pewno się nada.

\subsection[kiedy-warto-iść-w-asyncio-kiedy-nie]{Kiedy warto iść w
asyncio? Kiedy nie?}

Pojawienie się asyncio wraz z wersją 3.4 interpretera wywołało trochę
zamieszania w świecie Pythona. Nie wszyscy programiści rozumieli,
dlaczego zostało ono dodane do biblioteki standardowej oraz czy powinni
się nim zainteresować mając działające projekty napisane chociażby w
Django.

Inni zareagowali entuzjastycznie i zaadoptowali tę bibliotekę do swoich
potrzeb. Trzeba na początek powiedzieć, że asynchroniczność jest w
Pythonie już od długiego czasu. Pierwszy release frameworka Twisted miał
miejsce w 2002 roku, a więc aż 15 lat temu. Inny popularny framework -
Tornado jest z nami już od 8 lat. Te projekty miały swoją niszę na rynku
oprogramowania, udostępniając programowanie sterowane zdarzeniami na
długo zanim stało się to modne za sprawą NodeJS.

Asynchroniczne programowanie jest szczególnie przydatne, gdy mamy do
czynienia z aplikacjami, które przez większość czasu oczekują na
zakończenie operacji I/O, a mniej czasu spędzają na obliczeniach przy
pomocy CPU. Pomysł na podejście asynchroniczne pochodzi z bardzo prostej
obserwacji - gdy program czeka na zakończenie operacji wejścia-wyjścia,
to pozostaje bezczynny.

Mechanizmem sterującym, który wykrywa sytuacje rozpoczęcia oczekiwania
na operację I/O jest pętla zdarzeń. Zawiesza ona wykonywanie kodu i
przenosi się w inne miejsce, którego operacja wejścia-wyjścia zdążyła
się zakończyć.

W ten sposób maksymalizujemy wykorzystanie pojedynczego rdzenia CPU. Tak
więc od razu można odpowiedzieć na pytanie, kiedy NIE UŻYWAĆ asyncio -
gdy nie mamy wiele operacji I/O lub nie zajmują one wiele czasu
wykonania programu. Zbadanie tego faktu może zostać dokonane przy użyciu
profilerów. Zwykle {\em tradycyjna} aplikacja w Django, szczególnie
jeżeli już działa na produkcji, nie odniesie najmniejszej korzyści z
przepisania na asyncio. Bierze się to stąd, że chociaż może mieć ona
wielu klientów na sekundę, to jednak poszczególne żądania będą bardzo
krótkie. Nie ma potrzeby utrzymywania długożyjących połączeń z
przeglądarkami użytkowników.

Co innego w przypadku, gdy zbudowanie odpowiedzi wymaga odpytania kilku
lub kilkunastu usług wokół. W takiej sytuacji asyncio da natychmiastowe
przyspieszenie dzięki zmultipleksowaniu żądań i wykonaniu ich
jednocześnie. Na koniec musimy tylko poskładać ich wyniki. Bez asyncio
naszą jedyną opcją było użycie puli wątków, procesów lub oddelegowanie
do systemu kolejkowego, jak Celery i synchroniczne oczekiwanie na wynik.
Kolejną przesłanką za zastosowaniem asyncio jest konieczność
utrzymywania długo trwających połączeń z klientami. Idealnym przykładem
jest serwer websocketów, na którym można zbudować powiadomienia w czasie
rzeczywistym do naszej aplikacji. Mówimy cały czas o wykorzystaniu
asyncio po stronie serwera, ale oczywiście można też stworzyć
asynchronicznego klienta, który będzie w stanie odbierać strumień danych
płynących po sieci (potencjalnie z kilku źródeł) i odpowiednio na nie
reagować.

Nic nie stoi na przeszkodzie, by w jednym procesie Pythona, przy użyciu
jednej pętli zdarzeń, mieć jednocześnie osadzony tak serwer jak i
klienta, a nawet po kilka z nich. Sprawia to, że asyncio świetnie nadaje
się do pisania {\em glue-services} - tworów, które zapewniają pomost
pomiędzy różnymi niekompatybilnymi ze sobą usługami.

\subsection[forma-szkolenia]{Forma szkolenia}

W trakcie tego warsztatu uczestnicy będą realizować samodzielnie zadania
na podstawie instrukcji dostarczonej w repozytorium z początkowym kodem
do dalszego rozwijania. Prowadzący przeprowadzi krótki wstęp teoretyczny
i będzie wprowadzać uczestników w każdy etap warsztatów. W trakcie
wykonywania ćwiczeń prowadzący będzie sprawdzać postępy uczestników,
odpowiadać na pytania i pomagać w razie problemów.

Instrukcja poza pewnymi informacjami teoretycznymi zawierać też będzie
zadania zorganizowane zgodnie z programem szkolenia i oznaczone trzema
różnymi kolorami:

\startitemize[n,packed][stopper=.]
\item
  zielony - zakres minimum szkolenia, bez ich zrealizowania nie idź
  dalej
\item
  pomarańczowy - rozszerzenie zakresu podstawowego, łącznie z testami
\item
  czerwony - dla chętnych jako zadanie domowe w celu jeszcze lepszego
  poznania asyncio lub do wykonania w trakcie warsztatów dla
  zaawansowanych uczestników
\stopitemize

W treści zadań zawarte są też podpowiedzi, ukryte do momentu aż
uczestnik nie zdecyduje się z nich skorzystać. Mogą okazać się pomocne,
gdyby uczestnik miał problem z podejściem do danego etapu lub chciał
zobaczyć inny sposób rozwiązania.

\subsection[program-szkolenia]{Program szkolenia}

\startitemize[n][stopper=.,width=2.0em]
\item
  Wstęp teoretyczny (10 minut)
\item
  czym jest asyncio
\item
  kiedy warto stosować asyncio?
\item
  przegląd dostępnych narzędzi i bibliotek
\item
  testowanie aplikacji z asyncio - złote zasady
\item
  Wprowadzenie do projektu (5 minut)
\item
  Przedstawienie zawartości repozytorium
\item
  Pożądany zakres funkcjonalności na koniec szkolenia
\item
  Moduł 1 - czat ogólny (1 godzina 15 minut)
\item
  Obsługa przychodzącej wiadomości
\item
  Wysłanie wiadomości do pozostałych użytkowników czatu
\item
  Zabezpieczenie przed sytuacją, gdy jeden z użytkowników rozłączy się w
  trakcie przekazywania wiadomości
\item
  Oznaczanie obecności na czacie
\item
  Statystyki czatu
\item
  Przechowanie historii czatu dla nowych uczestników
\stopitemize

(przerwa - 15 minut)

\startitemize[n][start=4,stopper=.,width=2.0em]
\item
  Moduł 2 - prywatne pokoje (45 minut)
\item
  Zakładanie pokoju
\item
  Zapraszanie do środka uczestników
\item
  Wchodzenie do pokoju przez listę
\item
  Możliwość opuszczania pokoju
\item
  Automatyczne usuwanie pokoju po wyjściu z niego ostatniego uczestnika
\item
  Moduł 3 - prywatne wiadomości (45 minut)
\item
  Możliwość rozpoczęcia konwersacji z innym użytkownikiem
\item
  Dynamiczne tworzenie pokoi z zachowaniem historii konwersacji
\item
  Możliwość dołączenia kolejnego uczestnika do prywatnej rozmowy i
  utworzenie w ten sposób dynamicznie nowego pokoju
\item
  Podsumowanie
\stopitemize

\section[komentarz-do-modułu-i]{Komentarz do modułu I}

W tej części uczestnicy zapewnią minimum funkcjonalności, jaką powinna
mieć aplikacja czatowa - obsługę wspólnego dla wszystkich pokoju
czatowego. Jest to najważniejsza część warsztatu - to tu zostaną
podłożone podwaliny pod kolejne części, uczestnicy zrozumieją czym jest
pętla zdarzeń i jak unikać jej blokowania.

\section[komentarz-do-modułu-ii]{Komentarz do modułu II}

Druga część wzbogaci rozwijany projekt o możliwość rozpoczynania
prywatnych rozmów i dynamicznie zarządzanych pokoi prywatnych. Ujawni to
więcej ciekawych problemów typu {\em race conditions}, a przecież o to
chodzi - by nauczyć się wykrywać i zabezpieczać takie sytuacje.

\section[komentarz-do-modułu-iii]{Komentarz do modułu III}

Ostatnia część skupia się na dodaniu kolejnej, częstej funkcjonalności -
prywatnych wiadomości. Chociaż mierzyć się będziemy z tymi samymi
problemami, co wcześniej, to jest {\em znikający} użytkownicy, to jednak
rozwiązanie jakie chcemy osiągnąć będzie zgoła inne - nie tylko chcemy
pozbyć się wyjątków związanych z błędami przy próbie wysyłania
wiadomości na martwy websocket, ale także zapewnić integralność danych.
Jeżeli użytkownika nie było w momencie wysyłania do niego wiadomości, to
chcielibyśmy aby miał możliwość jej odczytania później.

\subsection[co-robić-dalej-po-szkoleniu]{Co robić dalej po szkoleniu?}

Zdecydowanie polecam zapoznanie się z pierwszymi trzema pozycjami z
bibliografii, począwszy od książki Luciano Ramalho. Powinno to dać
dalszą, solidną porcję wiedzy potrzebną do projektowania aplikacji
opartych na asyncio.

Liczba bibliotek i narzędzi kompatybilnych z asyncio rośnie z czasem.
Większość z nich można znaleźć w dedykowanym repozytorium na
{\bf githubie} {[}5{]}

\subsection[bibliografia]{Bibliografia}

\startitemize[n,packed][stopper=.]
\item
  Dokumentacja asyncio. https://docs.python.org/3/library/asyncio.html
\item
  Dokumentacja aiohttp. https://aiohttp.readthedocs.io/en/stable/
\item
  Fluent Python autorstwa Luciano Ramalho, rozdział 18.
\item
  Dokumentacja pytest. https://docs.pytest.org/
\item
  awesome-asyncio (repozytorium bibliotek i rozszerzeń asyncio)
  https://github.com/timofurrer/awesome-asyncio
\stopitemize


\stoptext
